
\documentclass[a4paper,11pt]{article}
\usepackage[a4paper, margin=8em]{geometry}

% usa i pacchetti per la scrittura in italiano
\usepackage[french,italian]{babel}
\usepackage[T1]{fontenc}
\usepackage[utf8]{inputenc}
\frenchspacing 

% usa i pacchetti per la formattazione matematica
\usepackage{amsmath, amssymb, amsthm, amsfonts}

% usa altri pacchetti
\usepackage{gensymb}
\usepackage{hyperref}
\usepackage{standalone}

% imposta il titolo
\title{Appunti Calcolo Numerico}
\author{Luca Seggiani}
\date{2025}

% disegni
\usepackage{pgfplots}
\pgfplotsset{width=10cm,compat=1.9}

% imposta lo stile
% usa helvetica
\usepackage[scaled]{helvet}
% usa palatino
\usepackage{palatino}
% usa un font monospazio guardabile
\usepackage{lmodern}

% tikz in sans
\tikzset{every picture/.style={/utils/exec={\sffamily}}}

\renewcommand{\rmdefault}{ppl}
\renewcommand{\sfdefault}{phv}
\renewcommand{\ttdefault}{lmtt}

% circuiti
\usepackage{circuitikz}
\usetikzlibrary{babel}

% disponi il titolo
\makeatletter
\renewcommand{\maketitle} {
	\begin{center} 
		\begin{minipage}[t]{.8\textwidth}
			\textsf{\huge\bfseries \@title} 
		\end{minipage}%
		\begin{minipage}[t]{.2\textwidth}
			\raggedleft \vspace{-1.65em}
			\textsf{\small \@author} \vfill
			\textsf{\small \@date}
		\end{minipage}
		\par
	\end{center}

	\thispagestyle{empty}
	\pagestyle{fancy}
}
\makeatother

% disponi teoremi
\usepackage{tcolorbox}
\newtcolorbox[auto counter, number within=section]{theorem}[2][]{%
	colback=blue!10, 
	colframe=blue!40!black, 
	sharp corners=northwest,
	fonttitle=\sffamily\bfseries, 
	title=Teorema~\thetcbcounter: #2, 
	#1
}

% disponi definizioni
\newtcolorbox[auto counter, number within=section]{definition}[2][]{%
	colback=red!10,
	colframe=red!40!black,
	sharp corners=northwest,
	fonttitle=\sffamily\bfseries,
	title=Definizione~\thetcbcounter: #2,
	#1
}

% disponi problemi
\newtcolorbox[auto counter, number within=section]{problem}[2][]{%
	colback=green!10,
	colframe=green!40!black,
	sharp corners=northwest,
	fonttitle=\sffamily\bfseries,
	title=Problema~\thetcbcounter: #2,
	#1
}

% disponi codice
\usepackage{listings}
\usepackage[table]{xcolor}

\definecolor{codegreen}{rgb}{0,0.6,0}
\definecolor{codegray}{rgb}{0.5,0.5,0.5}
\definecolor{codepurple}{rgb}{0.58,0,0.82}
\definecolor{backcolour}{rgb}{0.95,0.95,0.92}

\lstdefinestyle{codestyle}{
		backgroundcolor=\color{black!5}, 
		commentstyle=\color{codegreen},
		keywordstyle=\bfseries\color{magenta},
		numberstyle=\sffamily\tiny\color{black!60},
		stringstyle=\color{green!50!black},
		basicstyle=\ttfamily\footnotesize,
		breakatwhitespace=false,         
		breaklines=true,                 
		captionpos=b,                    
		keepspaces=true,                 
		numbers=left,                    
		numbersep=5pt,                  
		showspaces=false,                
		showstringspaces=false,
		showtabs=false,                  
		tabsize=2
}

\lstdefinestyle{shellstyle}{
		backgroundcolor=\color{black!5}, 
		basicstyle=\ttfamily\footnotesize\color{black}, 
		commentstyle=\color{black}, 
		keywordstyle=\color{black},
		numberstyle=\color{black!5},
		stringstyle=\color{black}, 
		showspaces=false,
		showstringspaces=false, 
		showtabs=false, 
		tabsize=2, 
		numbers=none, 
		breaklines=true
}

\lstdefinelanguage{javascript}{
	keywords={typeof, new, true, false, catch, function, return, null, catch, switch, var, if, in, while, do, else, case, break},
	keywordstyle=\color{blue}\bfseries,
	ndkeywords={class, export, boolean, throw, implements, import, this},
	ndkeywordstyle=\color{darkgray}\bfseries,
	identifierstyle=\color{black},
	sensitive=false,
	comment=[l]{//},
	morecomment=[s]{/*}{*/},
	commentstyle=\color{purple}\ttfamily,
	stringstyle=\color{red}\ttfamily,
	morestring=[b]',
	morestring=[b]"
}

% disponi sezioni
\usepackage{titlesec}

\titleformat{\section}
	{\sffamily\Large\bfseries} 
	{\thesection}{1em}{} 
\titleformat{\subsection}
	{\sffamily\large\bfseries}   
	{\thesubsection}{1em}{} 
\titleformat{\subsubsection}
	{\sffamily\normalsize\bfseries} 
	{\thesubsubsection}{1em}{}

% disponi alberi
\usepackage{forest}

\forestset{
	rectstyle/.style={
		for tree={rectangle,draw,font=\large\sffamily}
	},
	roundstyle/.style={
		for tree={circle,draw,font=\large}
	}
}

% disponi algoritmi
\usepackage{algorithm}
\usepackage{algorithmic}
\makeatletter
\renewcommand{\ALG@name}{Algoritmo}
\makeatother

% disponi numeri di pagina
\usepackage{fancyhdr}
\fancyhf{} 
\fancyfoot[L]{\sffamily{\thepage}}

\makeatletter
\fancyhead[L]{\raisebox{1ex}[0pt][0pt]{\sffamily{\@title \ \@date}}} 
\fancyhead[R]{\raisebox{1ex}[0pt][0pt]{\sffamily{\@author}}}
\makeatother

\begin{document}

% sezione (data)
\section{Lezione del 11-04-25}

% stili pagina
\thispagestyle{empty}
\pagestyle{fancy}

% testo
Riprendiamo il discorso dell'interpolazione polinomiale nella prospsettiva di arrivare all'interpolazione di Newton, a cui abbiamo accennato alla scorsa lezione.

Avevamo detto di avere $(x_0, y_0), (x_1, y_1), ..., (x_k, y_k)$, cioè $k + 1$ punti \textbf{distinti}, cioè con $x_i \neq x_j$, $\forall i \neq j$, tali per cui volevamo interpolarli con un polinomio di grado $k$.
Questo esisteva ed era unico appunto sotto l'ipotesi di punti distinti.

Avevavmo visto le possibili base per la realizzazione di tale polinomio, di \textit{Vandermonde} e di \textit{Lagrange}.

\subsection{Differenze divise}
Avevamo poi introdotto la base di \textbf{Newton}, che definiva il polinomio del tipo:
$$
p_k (x) = \sum_{j = 0}^k f [ x_0, ..., x_j ] m_j(x)
$$
con:
\[
	\begin{cases}
		m_0(x) = 1 \\ 
		m_j(x) = (x - x_0) ... (x - x_{j - 1}), \quad \forall j \geq 1
	\end{cases}
\]
I coefficienti $f[x_0, ..., x_j]$ $\forall j$ si definiscono per \textit{ricorsione}.
Definiamo quindi la formula di ricorrenza:
\begin{definition}{Differenza divisa}
	Data una funzione $f: \mathbb{R} \rightarrow \mathbb{R}$, e dati $x_0, ..., x_{k - 1} \in \mathbb{R}$, con $x_i \neq x_j \, \forall i \neq j$, vogliamo una funzione $f[x]$ che rispetti:
	\[
		\begin{cases}
			f[x] = f(x) \quad k = 0 \\
			f[x_0, x] = \frac{f(x) - f(x_0)}{x - x_0} \quad k = 1 \\
			f[x_0, ... x_{k - 1}, x] = \frac{ f[x_0, ..., x_{k - 2}] - f[x_0, ..., x_{k - 1}] }{ x - x_{k - 1} }, \quad \text{altrimenti}
		\end{cases}
	\]
\end{definition}

Le differenze divise ricordano la forma del \textit{rapporto incrementale}, e anzi corrispondono esattamente a questo per $k = 1$, mentre per $k>1$ si può pensare ad un'approssimazione \textit{"grezza"} delle derivate di ordine $k$, cioè:
\[
	\begin{cases}
		k = 0 : \quad f[x] = f(x)  \\
		k = 1 : f[x_0, x] = \frac{f(x) - f(x_0)}{x - x_0} \approx \frac{df}{dx}(x_0) \\
		f[x_0, ... x_{k - 1}, x] = \frac{ f[x_0, ..., x_{k - 2}] - f[x_0, ..., x_{k - 1}] }{ x - x_{k - 1} } \approx \frac{d^{k-2}f}{dx^{k-2}}(x_0) 
	\end{cases}
\] # mettici caso grado 2 boh

\subsubsection{Proprietà delle differenze divise}
Possiamo trovare le seguenti proprietà delle differenze divise:
\begin{enumerate}
	\item $\forall P$ permutazione $\{i_0, ..., i_k\}$ di $\{0, ..., k \}$ vale:
		$$
			f[x_0, x_1, ..., x_k] = f[x_{i_0}, x_{i_1}, ..., x_{i_k}, x]
		$$ # non so se ci vuole una x lì dentro
		cioè una permutazione dei $0, ..., k$ non cambia la differenza divisa;

	\item $f[x_0, ..., x_{k - 1}, x]$ è una funzione non definita in $x_0, ..., x_{k-1}$.
		Tuttavia, se $f \in C^1(I)$ con $I$ contenente $x_0, ..., x_k$, allora è prolungabile per continuità su tale intervallo;

	\item Se $f$ è non solo $C^1$, ma $f \in C^k(I)$, allora esiste $\epsilon$ tale che:
		$$
		\epsilon \in \left[ \min_{0 \leq i \leq k} x_i, \max_{0 \leq i \leq k} x_i \right]
		$$
		tale che:
		$$
		f[x_0, ..., x_k] = \frac{f^{k} (\epsilon) }{ k! }
		$$ # non so se ci vuole un epsilon lì dentro
		Questa è una conseguenza del teorema di Lagrange.
\end{enumerate}

\subsubsection{Calcolo delle differenze divise}
Esiste un metodo piuttosto schematico per il calcolo delle differenze divise.
Dati $k + 1$ punti di interpolazione, si possono calcolare le differenze divise fino all'ordine $k$ secondo questo schema (posto $k = 3$), detto \textbf{quadro delle differenze divise}:

\begin{table}[H]
	\center 
	\begin{tabular} { c | c | p{2cm}  p{2cm} p{2cm} }
		$x$ & $f(x)$ & DD1 & DD2 & DD3 \\
		\hline
		$x_0$ & $f(x_0)$ & // & // & // \\
		$x_1$ & $f(x_1)$ & $f[x_0, x_1]$ & // & // \\
		$x_2$ & $f(x_2)$ & $f[x_0, x_2]$ & $f[x_0, x_1, x_2]$ & // \\
		$x_3$ & $f(x_3)$ & $f[x_0, x_3]$ & $f[x_0, x_2, x_3]$ & $f[x_0, x_1, x_2, x_3]$ \\
	\end{tabular}
\end{table}

Cioè a ogni passaggio completiamo una riga, da sinistra verso destra, con le differenze divise note a quel passaggio.
Notiamo che in verità le differenze divise che compaiono sulla \textit{diagonale} sono le sole che compaiono effettivamente nel polinomio di Newton.


Supponendo di scrivere gli elementi della tabella come le entrate di una matrice $A = a{ij}$ triangolare inferiore, per calcolare una sua entrata $a_{ij}$ la formula da applicare è la seguente:
$$
a_{ij} = \frac{a_{i, j - 1} - a_{j - 1, j - 1}}{x_i - x}
$$

Poniamo infatti $A$:
$$
A = 
\begin{pmatrix}
	f(x_0) & & ... & & 0 \\
	f(x_1) & f[x_0, x_1] & & & & \\
	\vdots & f[x_0, x_2] & f[x_0, x_1, x_2] & & \vdots \\
	\vdots & \vdots & \vdots & & & \\
	f(x_k) &  f[x_0, x_k] & f[x_0, x_1, x_k] & ... & f[x_0, x_1, ..., x_k] \\
\end{pmatrix}
$$

\subsubsection{Esempio: quadro delle differneze divise}
Vediamo un caso pratico di quadro risolto:
\begin{table}[H]
	\center 
	\begin{tabular} { c | c | p{1cm}  p{1cm} p{1cm} }
		$x$ & $f(x)$ & DD1 & DD2 & DD3 \\
		\hline
		0 & 3 & // & // & // \\
		1 & 8 & 5 & // & // \\
		2 & 15 & 6 & 1 & // \\
		3 & 17 & 11 & 2 & $\frac{1}{2}$ \\
	\end{tabular}
\end{table}

A questo punto il polinomio sarà:
$$
p_3(x)
$$
$$
= f[x_0] + f[x_0, x_1] (x - x_0) + f[x_0, x_1, x_2] (x - x_0) (x - x_1) + f[x_0, x_1, x_2, x_3] (x - x_0) (x - x_1) (x - x_2) 
$$
$$
= 3 + 5 ( x - 0 ) + 1 ( x - 0 ) ( x - 1 ) + \frac{1}{2} ( x - 0 ) ( x - 1 ) ( x - 2 ) = \frac{1}{2}x^3 - \frac{1}{2}x^2 + 5x + 3
$$

Osserviamo che, nel caso in cui gli elementi di una colonna delle differenze divisie sono tutti uguali, gli elementi delle colonne successive sono tutti nulli, e in particolare il grado del polinomio di interpolazione $p_k$ sarà $p_i$ con $i$ l'ultima colonna non nulla (cioè l'ordine dell'ultima differenza divisa non nulla). 

\subsubsection{Esempio: colonne nulle nel quadro}
Vediamo ad esempio:
\begin{table}[H]
	\center 
	\begin{tabular} { c | c | p{1cm}  p{1cm} p{1cm} p{1cm} }
		$x$ & $f(x)$ & DD1 & DD2 & DD3 & DD4\\
		\hline
		0 & 5 & // & // & // & // \\
		-1 & 3 & 2 & // & // & //\\
		2 & 3 & -1 & -1 & // & // \\
		-2 & -9 & 7 &-5 & 1 & // \\
		3 & 11 & 2 & 0 & 1 & 0 
	\end{tabular}
\end{table}

Come vediamo, alla colonna DD3 le differenze divise sono 1 e 1, ergo la differenza divisa in DD4 è nulla.
Questo significherà che il polinomio interpolante, sebbene sono stati presi $k + 1 = 5$ punti, da cui ci aspetteremo un polinomio di grado $k = 4$, ha invece grado $3$, cioè è:
$$
p_2(x) = 5 + 2 (x - 0) - 1 (x - 0)(x + 1) + 1 (x - 0)(x + 1)(x - 2) = x^3 - 2x^2 - x + 5
$$

\par\smallskip

Osserviamo infine che data una tabella di valori $(x_0, y_0), ..., (x_k, y_k)$ da interpolare, possiamo riordinarli come vogliamo nel quadro delle differenze divise.
Questo può essere utile a semplificare i calcoli.

\subsubsection{Esempio: riordinamento dei valori}
Poniamo di avere i seguenti dati:
\begin{table}[H]
	\center 
	\begin{tabular} { c | c }
		$x$ & $f(x)$ \\
		\hline
		0 & 2 \\
		1 & 1 \\
		$\alpha$ & 4 \\
		-1 & $3 \alpha + 1$ \\
		3 & 11 
	\end{tabular}
\end{table}
con $\alpha \in \mathbb{R} \setminus \{ 0, 1, -1, 3 \}$ e di chiederci quale deve essere il valore di $\alpha$ perché il polinomio interpolante abbia grado minimo.

Vediamo quindi di riordinare i dati in modo che i termini dipendenti da $\alpha$ vadano a finire in fondo alla tabella:
\begin{table}[H]
	\center 
	\begin{tabular} { c | c | p{1cm} p{1cm} p{1cm} p{1cm} }
		$x$ & $f(x)$ & DD1 & DD2 & DD3 & DD4\\
		\hline
		0 & 2 & // & // & // & // \\
		1 & 1 & -1 & // & // & // \\
		3 & 11 & 3 & 2 & ...\\
		-1 & $3 \alpha + 1$ & $1 - 3 \alpha$ & $\frac{3 \alpha - 2}{2}$ & ... \\
		$\alpha$ & 4 & $\frac{2}{\alpha}$ & $\frac{2 + \alpha}{\alpha(\alpha - 1)}$ & ...
	\end{tabular}
\end{table}

Alla colonna DD1 non abbiamo speranza di soddisfare la condizione di colonna tutta uguale (ci sono -1 e 3), mentre alla colonna DD2 possiamo provare ad imporre:
\[
	\begin{cases}
		\frac{3 \alpha - 2}{2} = 2 \implies \alpha = 2 \\
		\frac{2 + \alpha}{\alpha(\alpha - 1)} = 2 \implies \alpha = 2
	\end{cases}
\]
che quindi, per pura fortuna, corrisponde.
Possiamo allora prendere $\alpha = 2$ e completare il quadro:
\begin{table}[H]
	\center 
	\begin{tabular} { c | c | p{1cm} p{1cm} p{1cm} p{1cm} }
		$x$ & $f(x)$ & DD1 & DD2 & DD3 & DD4\\
		\hline
		0 & 2 & // & // & // & // \\
		1 & 1 & -1 & // & // & // \\
		3 & 11 & 3 & 2 & // & //\\
		-1 & $3 \alpha + 1$ & $1 - 3 \alpha$ & 2 & 0 & // \\
		$\alpha$ & 4 & $\frac{2}{\alpha}$ & 2 & 0 & 0
	\end{tabular}
\end{table}
da cui il polinomio al solo grado 2:
$$
p_2(x) = 2 - x + 2x(x - 1) = 2x^2 - 3x + 2
$$

\subsection{Errore nell'approssimazione polinomiale}
Veniamo quindi alla valutazione dell'errore.
Vale il seguente teorema:
\begin{theorem}{Errore nell'approssimazione polinomiale}
	Presi $k + 1$ punti $x_0, ..., x_k$ con $x_i \neq x_j \, \forall i \neq j$.
	Sia $f: \mathbb{R} \rightarrow \mathbb{R}$ tale che $f \in C^{k + 1}(I)$ con $I \subseteq \mathbb{R}$ contenente $x_0, ..., x_k$.
	Se $p_k(x)$ è il polinomio di interpolazione di grado al più $k$ che interpola i punti $(x_0, f(x_0), ..., (x_k, f(x_k)))$ allora riguardo all'errore vale:
$$
f(x) - p_k(x) = \Pi(x) \frac{f^{(k + 1)}(\xi)}{(k + 1)!}
$$
con:
$$
\Pi(x) = \prod_{j = 0}^k (x - x_j)
$$
per un certo $\xi$ tale che:
$$
\xi \in \left[ \min_{0 \leq i \leq k} x_i \cup x, \max_{0 \leq i \leq k} x_i \cup x \right]
$$
\end{theorem}

Osserviamo quindi che conoscendo una stima di $|f^{k + 1}|$ e  di $|\Pi(x)|$ si può provare a stimare l'errore del polinomio di interpolazione quando si valuta $p_k(x)$ fuori dai nodi $\{ x_j \}_{j=0}^k$.

\subsection{Riassunto fra base di Newton e base di Lagrange}
Facciamo quindi un riassunto confrontando la base di Newton e la base di Lagrange.
\begin{table}[h!]
	\center 
	\begin{tabular} { c p{6cm} | p{6cm} }
		& \bfseries Newton & \bfseries Lagrange \\
		\hline
		\bfseries Base & $m_j(x) = (x - x_0) ... (x - x_{j - 1}) $ & $e_j(x) = \prod_{i \neq j}^k \frac{x - x_i}{x_j - x_i}$ \\
		\bfseries Coef. & $f[x_0, ... x_j]$ & $f(x_j) = y_j$ \\
		& Con la base di Newton, se si è calcolato $p_k(x)$ e si vuole $p_{k + 1}(x)$, basta calcolare $m_{k + 1}$ e $f[x_0, ..., x_{k}]$ da $f[x_0, ..., x_{k - 1}]$ &
		Con la base di Lagrange, per fare la stessa cosa bisogna ricalcolare tutti gli $e_j(x)$ \\
		& Il grado del polinomio si nota facilmente & Per trovare il grado del polinomio bisogna espandere nella base dei monomi
	\end{tabular}
\end{table}

\subsection{Interpolazione osculatoria di Hermite}
Vediamo un altro metodo per interpolare polinomi.
Supponiamo di conoscere non soltanto ai nodi il valore della funzione, ma anche il valore della sua derivata.
In questo caso potremo costruire un polinomio in cui le valutazioni e quelle derivata equivalongono rispettivamente ai valori di $f$ e $f'$.
Questo potrebbe essere utile nel caso in cui conosciamo sia la posizione che la velocità di un corpo in certi istanti e vogliamo approssimarne la traiettoria.

Definiamo quindi:
\begin{definition}{Polinomio interpolante di Hermite }
	Dati $x_0, ..., x_k \in \I \subseteq \mathbb{R}$ e $2k + 2$ valori $\{f(x_0), ..., f(x_k) \}$ e $\{ f'(x_0), ..., f'(x_k) \}$ di una certa funzione $f \in C^{1}(I)$, si dice polinomio interpolante di Hermite per $f$ cui nodi $x_0, ..., x_k$ il polinomio $H_{2k + 1}(x)$ di grado al più $2k + 1$ tale che:
	$$
		H_{2k + 1}(x_j) = f(x_j), \quad H'_{2k + 1}(x_j) = f'(x_j), \quad \forall j = 0, ..., k
	$$
\end{definition}

Varrà quindi il seguente teorema:
\begin{theorem}{Esistenza del polinomio interpolante di Hermite}
	Se $x_i \neq x_j$ $\forall i \neq j$ allora esiste ed è unico il polinomio di Hermite $H_{2k + 1}$ per $f \in C^1(I)$. 
\end{theorem}

# c'è una dimostrazione DA SAPERE sulle dispense 

Fondamentalmente l'argomentazione per la dimostrazione è simile a quella della matrice di Vandermonde.

Si potrebbe formulare un sistema lineare (# fallo) come è stato fatto nel casi Vandermonde, ma questo non conviene in quanto si incontrano gli stessi problemi di malcondizionamento che avevamo incontrato anche in Vandermonde.

Tipicamente allora si procede in maniera analoga all'interpolazione di Lagrange, ma generalizzata per ottenere le condizioni sulle derivate.
Più precisamente, cerchiamo un polinomio della forma:
$$
H_{2k + 1}(x) = \sum_{j = 0}^k h_{0j}(x) f(x_j) + \sum_{j = 0}^k h_{1j}(x) + f'(x_j)
$$

Avremo che $f(x_j)$ e $f'(x_j)$ sono chiaramente i dati del problema, che farano da coefficienti, mentre $h_{0j}(x)$ e $h_{1j}(x)$ saranno polinomi di grado $2k + 1$ tali che:
\begin{equation}
h_{0j}(x_i) =
\begin{cases}
	1, \quad i = j \\
	0, \quad i \neq j
\end{cases}, \quad
h'_{0j}(x_i) = 0  \quad \forall \, 0 \leq i \leq k
$$
$$
h_{1j}(x_i) = 0, \quad
h'_{1j}(x_i) =
\begin{cases}
	1, \quad i = j \\
	0, \quad i \neq j
\end{cases} \quad \forall \, 0 \leq i \leq k
\end{equation}
che possiamo interpretare come le condizioni imposte alle basi di Lagrange generalizzata al caso con la derivata prima.

Se valgono queste proprietà, infatti, si ha che:
$$
H_{2k + 1}(x_i) = \sum_{j = 0}^k h_{0j}(x_j) f(x_i) + \sum_{j = 0}^k h_{1j}(x_i) f'(x_j) = f(x_j) + 0 
$$
e rispetto alla derivata prima:
$$
H'_{2k + 1}(x_i) = \sum_{j = 0}^k h_{0j}'(x_i) f(x_j) + \sum_{j = 0}^k h'_{1j}f'(x_j) = 0 + f'(x_i)
$$
con in entrambi i casi, $\forall \, i = 0, ..., k$.

Osserviamo che come nel caso dell'interpolazione standard, stiamo rinunciando a cercare una rappresentazione di $H_{2k + 1}$ nella base dei monomi $\{ 1, x, ..., x^{2k + 1} \}$ e cerchiamo invece un'espressione rispetto alla base:
$$
	B_\text{Hermite} = \{ h_{01}(x), ..., h_{0k}(x), h_{10}(x), ..., h_{1k}(x) \}
$$
in cui i coefficienti delle combinazioni lineari sono:
$$
f(x_0), ..., f(x_k), f'(x_0), ..., f'(x_k)
$$

Il problema si riconduce quindi a trovare $h_{0j}(x)$, $h_{1j}(x)$ $\forall j$ che soddisfino le proprietà (3).

L'idea parte dall'\textit{ansatz} di cercare $h_{0j}$ e $h_{1j}$ della forma:
$$
h_{0j} = (Ax + B) e_j^2(x), \quad h_{1j} = (Cx + D) e_j^2(x)
$$
con $e_j(x)$ $j$-esimo polinomio in base di Lagrange, cioè:
$$
e_j(x) = \prod_{i \neq j}^k \frac{x - x_i}{x_j - x_i}
$$

Imponiamo quindi le proprietà (3) per ricavare $A, B, C, D$ di ogni $h_{0j}$ e $h_{1j}$, per ottenere: # fai calcoli
\[
	\begin{cases}
		h_{0j}(x) = \left( 1 - 2 e_j'(x_j) (x - x_j) \right) e_j^2(x) \\
		h_{1j}(x) = (x - x_j) e^2_j(x)
	\end{cases}
\]
ossia:
$$
H_{2k + 1}(x) = \sum_{j = 0}^k f(x_j) \left( 1 - 2 e_j'(x_j)(x - x_j) \right) e_j^2(x) + \sum_{j = 0}^k f'(x_j) (x - x_j) e_j^2(x)
$$

\subsubsection{Esempio: trenitalia fallisce}
Supponiamo di avere 2 binari e di volerli unire in maniera $C^1$, ovvero vogliamo trovare un polinomio $H_3(x)$ tale che:
\[
	\begin{cases}
		H_3(0) = 0 \\
		H_3(d) = L
	\end{cases}, \quad
	\begin{cases}
		H_3'(0) = 0 \\
		H_3'(d) = s
	\end{cases} 
\]

Il polinomio dovrà essere di grado 3 (le due condizioni alle derivate contano nel calcolo del grado).
Dobbiamo calcolare, prima di tutto, le basi di Lagrange:
$$
e_0(x) = \frac{d - x}{d}, \quad e_1(x) = \frac{x}{d}
$$
e sostituire quindi nelle formule per $h_{0j}(x)$ e $h_{1j}(x)$:
$$
h_{00}(x) = \left( 1 + 2\frac{x}{d} \right) \frac{(x - d)^2}{d^2}, \quad h_{01}(x) = \left( 1 - 2 \frac{x - d}{d} \right) \frac{x^2}{ð^2}
$$
$$
h_{10}(x) = x \frac{(x - d)^2}{d^2}, \quad h_{11} = (x - d) \frac{x^2}{d^2}
$$
per cui otterremo:
$$
H_3(x) = 0 \cdot h_{00}(x) + L \cdot h_{01}(x) + 0 \cdot h_{10}(x) + S \cdot h_{11}(x) = L \left( 1 - 2 \frac{x - d}{d} \right) \frac{x^2}{d^2} + S (x - d) \frac{x^2}{d^2}
$$

\subsubsection{Errore nell'approssimazione di Hermite}
Valutiamo quindi l'errore:
\begin{theorem}{Errore nell'approssimazione di Hermite}
	Data $f \in C^{2k + 2}(I)$ allora:
$$
f(x) - H_{2k + 1}(x) = \Pi^2 (x) \frac{f^{(2k + 1)(\xi)}}{(2k + 2)!}
$$
con le stesse definizioni del 14.1.
\end{theorem}

\end{document}
