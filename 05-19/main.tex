
\documentclass[a4paper,11pt]{article}
\usepackage[a4paper, margin=8em]{geometry}

% usa i pacchetti per la scrittura in italiano
\usepackage[french,italian]{babel}
\usepackage[T1]{fontenc}
\usepackage[utf8]{inputenc}
\frenchspacing 

% usa i pacchetti per la formattazione matematica
\usepackage{amsmath, amssymb, amsthm, amsfonts}

% usa altri pacchetti
\usepackage{gensymb}
\usepackage{hyperref}
\usepackage{standalone}

% imposta il titolo
\title{Appunti Calcolo Numerico}
\author{Luca Seggiani}
\date{2025}

% disegni
\usepackage{pgfplots}
\pgfplotsset{width=10cm,compat=1.9}

% imposta lo stile
% usa helvetica
\usepackage[scaled]{helvet}
% usa palatino
\usepackage{palatino}
% usa un font monospazio guardabile
\usepackage{lmodern}

% tikz in sans
\tikzset{every picture/.style={/utils/exec={\sffamily}}}

\renewcommand{\rmdefault}{ppl}
\renewcommand{\sfdefault}{phv}
\renewcommand{\ttdefault}{lmtt}

% circuiti
\usepackage{circuitikz}
\usetikzlibrary{babel}

% disponi il titolo
\makeatletter
\renewcommand{\maketitle} {
	\begin{center} 
		\begin{minipage}[t]{.8\textwidth}
			\textsf{\huge\bfseries \@title} 
		\end{minipage}%
		\begin{minipage}[t]{.2\textwidth}
			\raggedleft \vspace{-1.65em}
			\textsf{\small \@author} \vfill
			\textsf{\small \@date}
		\end{minipage}
		\par
	\end{center}

	\thispagestyle{empty}
	\pagestyle{fancy}
}
\makeatother

% disponi teoremi
\usepackage{tcolorbox}
\newtcolorbox[auto counter, number within=section]{theorem}[2][]{%
	colback=blue!10, 
	colframe=blue!40!black, 
	sharp corners=northwest,
	fonttitle=\sffamily\bfseries, 
	title=Teorema~\thetcbcounter: #2, 
	#1
}

% disponi definizioni
\newtcolorbox[auto counter, number within=section]{definition}[2][]{%
	colback=red!10,
	colframe=red!40!black,
	sharp corners=northwest,
	fonttitle=\sffamily\bfseries,
	title=Definizione~\thetcbcounter: #2,
	#1
}

% disponi problemi
\newtcolorbox[auto counter, number within=section]{problem}[2][]{%
	colback=green!10,
	colframe=green!40!black,
	sharp corners=northwest,
	fonttitle=\sffamily\bfseries,
	title=Problema~\thetcbcounter: #2,
	#1
}

% disponi codice
\usepackage{listings}
\usepackage[table]{xcolor}

\definecolor{codegreen}{rgb}{0,0.6,0}
\definecolor{codegray}{rgb}{0.5,0.5,0.5}
\definecolor{codepurple}{rgb}{0.58,0,0.82}
\definecolor{backcolour}{rgb}{0.95,0.95,0.92}

\lstdefinestyle{codestyle}{
		backgroundcolor=\color{black!5}, 
		commentstyle=\color{codegreen},
		keywordstyle=\bfseries\color{magenta},
		numberstyle=\sffamily\tiny\color{black!60},
		stringstyle=\color{green!50!black},
		basicstyle=\ttfamily\footnotesize,
		breakatwhitespace=false,         
		breaklines=true,                 
		captionpos=b,                    
		keepspaces=true,                 
		numbers=left,                    
		numbersep=5pt,                  
		showspaces=false,                
		showstringspaces=false,
		showtabs=false,                  
		tabsize=2
}

\lstdefinestyle{shellstyle}{
		backgroundcolor=\color{black!5}, 
		basicstyle=\ttfamily\footnotesize\color{black}, 
		commentstyle=\color{black}, 
		keywordstyle=\color{black},
		numberstyle=\color{black!5},
		stringstyle=\color{black}, 
		showspaces=false,
		showstringspaces=false, 
		showtabs=false, 
		tabsize=2, 
		numbers=none, 
		breaklines=true
}

\lstdefinelanguage{javascript}{
	keywords={typeof, new, true, false, catch, function, return, null, catch, switch, var, if, in, while, do, else, case, break},
	keywordstyle=\color{blue}\bfseries,
	ndkeywords={class, export, boolean, throw, implements, import, this},
	ndkeywordstyle=\color{darkgray}\bfseries,
	identifierstyle=\color{black},
	sensitive=false,
	comment=[l]{//},
	morecomment=[s]{/*}{*/},
	commentstyle=\color{purple}\ttfamily,
	stringstyle=\color{red}\ttfamily,
	morestring=[b]',
	morestring=[b]"
}

% disponi sezioni
\usepackage{titlesec}

\titleformat{\section}
	{\sffamily\Large\bfseries} 
	{\thesection}{1em}{} 
\titleformat{\subsection}
	{\sffamily\large\bfseries}   
	{\thesubsection}{1em}{} 
\titleformat{\subsubsection}
	{\sffamily\normalsize\bfseries} 
	{\thesubsubsection}{1em}{}

% disponi alberi
\usepackage{forest}

\forestset{
	rectstyle/.style={
		for tree={rectangle,draw,font=\large\sffamily}
	},
	roundstyle/.style={
		for tree={circle,draw,font=\large}
	}
}

% disponi algoritmi
\usepackage{algorithm}
\usepackage{algorithmic}
\makeatletter
\renewcommand{\ALG@name}{Algoritmo}
\makeatother

% disponi numeri di pagina
\usepackage{fancyhdr}
\fancyhf{} 
\fancyfoot[L]{\sffamily{\thepage}}

\makeatletter
\fancyhead[L]{\raisebox{1ex}[0pt][0pt]{\sffamily{\@title \ \@date}}} 
\fancyhead[R]{\raisebox{1ex}[0pt][0pt]{\sffamily{\@author}}}
\makeatother

\begin{document}

% sezione (data)
\section{Lezione del 19-05-25}

% stili pagina
\thispagestyle{empty}
\pagestyle{fancy}

% testo
\subsection{Calcolo di autovalori e autovettori}
Vediamo quindi il problema di, data una matrice $A \in \mathbb{C}^{n \times n}$, trovare autocoppie $(\lambda, v)$, con $\lambda \in \mathbb{C}$, $v \in \mathbb{C}^n \setminus \{0\}$ che verificano:
$$
A v = \lambda v
$$

Vedremo 2 metodi iterativi che approcciano 2 versioni di questo problema:
\begin{enumerate}
	\item Calcolo di una singola autocoppia $(\lambda, v)$, ad esempio una coppia in cui $\lambda$ è l'autovalore di modulo massimo;
	\item Calcolo di tutte le autocoppie $(\lambda, v)$, ovvero una autocoppia per ogni autovalore distinto. Non approfondiremo questo metodo per questioni di complessità.
\end{enumerate}

Osserviamo che gli autovalori sono necessariamente $n$ contate le le loro molteplicità $\mu$, in quanto rappresentano le soluzioni di un polinomio di grado $n$, e altro non possono fare per via del teorema fondamentale dell'algebra.

Osserviamo poi che gli autovettori sono definiti a meno di moltiplicazione per scalare, cioè sono sempre infiniti gli autovettori associati ad un singolo autovalore $\lambda$ (tutti quelli nell'autopazio).

\subsubsection{Metodo delle potenze}
Vediamo il metodo 1) per ottenere una singola autocoppia, e in particolare quella di modulo (autovalore) massimo.
Questo può essere utile per:
\begin{itemize}
	\item Trovare il raggio spettrale di una matrice;
	\item Trovare la norma 2 di una matrice come:
		$$
			|A|_2 = \sqrt{\rho(A^H A)}
		$$
\end{itemize}

Supponiamo che $A$ sia diagonalizzabile.
Questa è l'ipotesi che ci assicura che ogni autovalore ha un autovettore associato linearmente indipendente.
Prendiamo quindi gli autovalori $\lambda_1, ..., \lambda_n$ posti che soddisfano:
$$
|\lambda_1| > |\lambda_2| \geq |\lambda_3| \geq ... \geq |\lambda_n|
$$
dove il primo è in maggiorazione stretta (altrimenti non si distinguerebbe un unico autovalore di modulo massimo).

Dato che $A$ è diagonalizzabile, abbiamo dai fondamenti di algebra lineare che esiste una base di $\mathbb{C}^n$ fatta da autovettori di $A$, ovvero $\exists v^{(1)}, ..., v^{(n)} \in \mathbb{C}^n$ tali che:
$$
A v^{(j)} = \lambda_j v^{(j)}, \quad j = 1, ..., n
$$
e per ogni $z \in \mathbb{C}^n$ vale che $\exists ! c_1, .., c_n \in \mathbb{C}$ tali che:
$$
z = \sum_{j = 1}^n c_j v^{(j)} = c_1 v^{(1)} + ... c_n v^{(n)}
$$

\par\smallskip

L'idea fondamentale del metodo è che, preso $z \in \mathbb{C}^n$ a caso, abbiamo:
$$
A z = A (c_1 v^{(1)} + ... + c_n v^{(n)}) = c_1 A v^{(1)} + ... + c_n A v^{(n)} = c_1 \lambda_1 v^{(1)} + ... + c_n \lambda_n v^{(n)}
$$
Continuando ad iterare si ha:
$$
A^2 z = A (Az) = A (c_1 \lambda_1 v^{(1)} + ... + c_n \lambda_n v^{(n)})
$$
$$
= c_1 \lambda_1 A v^{(1)} + ... + c_n \lambda_b A v^{(n)} = c_1 \lambda_1^2 v^{(1)} + ... + c_n \lambda_n^2 v^{(n)}
$$
e quindi continuando ad iterare si ha che:
$$
A^k z = c_1 \lambda_1^k v^{(1)} + ... + c_n \lambda_n^k v^{(n)}
$$

Il vantaggio è che l'autovalore di modulo massimo compare chiaramente con contribuzioni sempre più grandi, in quanto:
$$
= \lambda_1^k \left( c_1 v^{(1)} + c_2 \left( \frac{\lambda_2}{\lambda_1} \right)^k v^{(2)} + ... + c_n \left( \frac{\lambda_n}{\lambda_1} \right)^k v^{(n)} \right) \sim \lambda_1^k, \quad k \rightarrow + \infty
$$

Più formalmente, abbiamo che vale il teorema:
\begin{theorem}{Convergenza del metodo delle potenze}
	Sia considerata la successione:
	\[
		\begin{cases}
			z^{(0)} = z \\
			z^{(k + 1)} = Az^{(k)} = A^{k + 1} z^{(0)} = A^{k + 1} z
		\end{cases}
	\]
	con $A \in \mathbb{C}^{n \times n}$ hermitiana ($A = A^H$), con autovalori:
	$$
	|\lambda_1| > |\lambda_2| \geq |\lambda_3| \geq ... \geq |\lambda_n| > 0
	$$
	e sia $h \in \{1, ..., n\}$ tale per cui $v_h^{(1)} \neq 0$.
	Se $z^{(0)} \in \mathbb{C}^n$ è tale che:
	$$
	\left[V^{(1)}\right]^H z^{(0)} \neq 0
	$$
	allora la successione è tale che:
	$$
	\lim_{k \rightarrow +\infty} \frac{z^{(k)}}{z_h^{(k)}} = \tilde{v}^{(1)}
	$$
	multiplo scalare di $v^{(1)}$ (in particolare $\tilde{v}^{(1)} = \frac{v^{(1)}}{v_h^{(1)}}$).
	Inoltre, per l'autovalore vale:
	$$
	\lim_{k \rightarrow +\infty} \frac{\left[ z^{(k)} \right]^H A z^{(k)}}{\left[ z^{(k)} \right]^H z^{(k)}} = \lambda_1
	$$
\end{theorem}

Osserviamo che, dato $x \in \mathbb{C}^n$ e considerato il \textit{quoziente di Rayleigh}:
$$
R(x) = \frac{x^H A x}{x^H x}
$$
Se $x$ verifica $Ax = \lambda x$ (cioè è autovettore) allora:
$$
R(x) = \frac{x^H A x}{x^H x} = \frac{x^H \lambda x}{x^H x} = \lambda \frac{x^H x}{x^H x} = \lambda
$$

Notiamo la motivazione dell'assunto di matrice hermitiana.
Di base, per il teorema spettrale si ha che:
$$
A = A^H \implies A \text{ diagonalizzabile}
$$
e in particolare si può scegliere una base ortonormale di autovettori, cioè trovare $v^{(1)}, ..., v^{(n)} \in \mathbb{C}^n$ tali che:
$$
A v^{(j)} = \lambda_j v^{(j)}, \quad \left[ v^{(i)} \right]^H v^{(j)} = \, < v^{(i)}, v^{(j)} > \, =
	\begin{cases}
		0, \quad i \neq j \\
		1, \quad i = j
	\end{cases}
$$
cioè equivalentemente:
$$
A = V D V^H
$$
con $D$ diagonalizzabile e $V$ normale data dagli autovettori in colonna.

\par\smallskip

Dimostriamo quindi il teorema.
Dato che $A = A^H$ hermitiana si ha che:
$$
z^{(0)} = \sum_{j = 1}^n c_j v^{(j)} 
$$
ed inoltre abbiamo:
$$
0 \neq \left[ v^{(1)} \right]^H z^{(0)} = \sum_{j = 1}^n c_j \left[ v^{(1)} \right]^H v^{(j)} = c_1 \left[ v^{(1)} \right]^H v^{(1)} = c_1
$$
e quindi $c_1 \neq 0$.

Dimostriamo allora le due tesi, su autovettore (1) a autovalore massimo (2):
\begin{enumerate}
	\item 
Come fatto in precedenza prendiamo: 
$$
z^{(k)} = A^k z^{(0)} = \lambda_1^k \left( c_1 v^{(1)} + c_2 \left( \frac{\lambda_2}{\lambda_1} \right)^k v^{(2)} + ... + c_n \left( \frac{\lambda_n}{\lambda_1} \right)^k v^{(n)} \right)
$$
$$
= \lambda_1^k \left( c_1 v^{(1)} + \sum_{j = 2}^n \left(\frac{\lambda_j}{\lambda_i}\right)^k v^{(j)} c_j \right)
$$
dove vorremo dimostrare che il termine a destra scompare per $k \rightarrow +\infty$.

Prendiamo quindi il limite:
$$
\lim_{k \rightarrow +\infty} \frac{ \lambda_1^k \left( c_1 v^{(1)} + \sum_{j = 2}^n \left(\frac{\lambda_j}{\lambda_i}\right)^k v^{(j)} c_j \right) }{ \lambda_1^k \left( c_1 v^{(1)}_h + \sum_{j = 2}^n \left(\frac{\lambda_j}{\lambda_i}\right)^k v^{(j)}_h c_j \right) }
$$
$$
\lim_{k \rightarrow +\infty} \frac{ c_1 v^{(1)} + \sum_{j = 2}^n \left(\frac{\lambda_j}{\lambda_i}\right)^k v^{(j)} c_j ) }{ c_1 v^{(1)}_h + \sum_{j = 2}^n \left(\frac{\lambda_j}{\lambda_i}\right)^k v^{(j)}_h c_j }
= \lim_{n \rightarrow + \infty} \frac{c_1 v^{(1)}}{c_1 v_h^{(1)}} = \frac{v^{(1)}}{v_h^{1}}
$$
che è la prima tesi.

	\item
Prendiamo quindi:
$$
\lim_{k \rightarrow + \infty} \frac{\left[ z^{(k)} \right]^H A z^{(k)}}{\left[ z^{(k)} \right]^H z^{(k)}} 
= \lim_{k \rightarrow + \infty} \frac{ \left[ \frac{z^{(k)}}{z_h^{(k)}} \right]^H A \frac{z^{(k)}}{z_h^{(k)}} }{ \left[ \frac{z^{(k)}}{z^{(k)}_h} \right]^H A \frac{z^{(k)}}{z^{(k)}_h} }
= \frac{ \left[ \frac{v^{(1)}}{v_h^{(1)}} \right]^H A \frac{v^{(1)}}{v_h^{(1)}} }{ \left[ \frac{v^{(1)}}{v^{(1)}_h} \right]^H A \frac{v^{(1)}}{v^{(1)}_h} } 
$$
che moltiplicando sopra e sotto per $\left( v_h^{(1)} \right)^2$ dà:
$$
= \frac{ [v^{(1)} ]^H A v^{(1)} }{ [v^{(1)}]^H v^{(1)} } = \lambda_1
$$
da cui la seconda tesi. \qed
\end{enumerate}

Osserviamo che $h$ è stato usato perché dividiamo per $z_h^{(k)}$ che per $k$ abbastanza grandi è $\neq 0$, in quanto vicino a $v_h^{(1)}$.

Osserviamo poi che nell'implementazione pratica il metodo delle potenze $z^{(k)}$ viene diviso per la sua norma, così da ottenere solo vettori di norma $1$ ed evitare problemi di overflow o underflow.

\subsubsection{Criterio di stop}
Vediamo la condizione di stop del metodo delle potenze.
In genere si usa:
$$
|R(z^{(k)} - R(z^{(k - 1)}))| < \epsilon
$$
con ad esempio $\epsilon = 10^{-8}$.

Testiamo gli autovalori con $R()$ e non gli autovettori con:
$$
|z^{(k)} - z^{(k + 1)}|_2 < \epsilon
$$
in quanto non sappiamo verso quale multiplo scalare degli autovettori ci stiamo dirigendo, per cui la differenza ad ogni passaggio potrebbe essere molto grande.

\subsubsection{Implementazione MATLAB del metodo delle potenze}
Vediamo quindi, da quello che abbiamo detto, come si può implementare una funzione per il calcolo dell'autocoppia massima in MATLAB:
\lstset{style=codestyle, language=MATLAB}
\lstinputlisting{../code/matlab/max_eigen.m}

Il costo è quello di una norma ($O(n)$) e di un prodotto matrice vettore ($O(n^2)$ )ad ogni iterazione, cioè $O(k \cdot n^2)$.

\subsubsection{Commenti sull'ipotesi di convergenza}
Possiamo fare un commento sulle ipotesi del teorema 22.1.
\begin{itemize}
	\item Non è strettamente necessario che $A = A^H$ hermitiana, in quanto si può dimostrare che il metodo funziona anche con $A$ solo diagonalizzabile;
	\item L'assunzione $\left[ v^{(1)} \right]^H z^{(0)} \neq 0$ è virtualmente vera (o meglio vera con probabilità $1$) per qualsiasi $z^{(0)} \in \mathbb{C}^n$ scelto a caso.
		Inoltre, nel caso di calcolo a macchina si ha che gli errori di arrotondamento stessi aiutano ad uscire dall'ortogonalità, anche se si parte da tale condizione.
		\item L'assunzione $|\lambda_1| > |\lambda_2|$ non è invece rinunciabile, in quanto si possono costruire esempi di successioni non convergenti nel caso di più autovalori dominanti.

			Prendiamo l'esempio:
			$$
				A =
				\begin{pmatrix}
					0 & 1 \\ 
					1 & 0
				\end{pmatrix}
			$$
			con 2 autovalori di modulo 1.
			Prendiamo:
			$$
			z^{(0)} = 
			\begin{pmatrix}
				2 \\ 3
			\end{pmatrix}
			$$
			Si trova che l'aggiornamento scambia semplicemente le componenti:
			$$
			A z^{(0)} =
			\begin{pmatrix}
				3 \\ 2
			\end{pmatrix}
			$$
			e cosi via all'infinito.

			Gli autovettori, calcolati a mano, sono semplicemente:
			$$
				v^{(1)} =
				\begin{pmatrix}
					1 \\ 1
				\end{pmatrix}, \quad
				v^{(2)} =
				\begin{pmatrix}
					1 \\ -1
				\end{pmatrix}
			$$
\end{itemize}

Osserviamo quindi che per eseguire il metodo delle potenze, serve soltanto essere in grado di valutare il prodotto matrice vettore $A z$.
Questo può essere utile nel caso sopra citato, quando si ha una matrice definita come risultato di operazioni aritmetiche, magari costose.

Ad esempio, se si vuole stimare:
$$
|A|_2 = \sqrt{\rho(A^H A)}
$$
si può pensare di applicare il metodo delle potenze su $A^H A$, senza nemmeno dover calcolare $A^H A$ ma applicando direttamente i prodotti matrici vettore $A^H A z$.

\subsection{Metodi per più autovalori}
\subsubsection{Autovalori minimi}
Supponiamo di essere interessati all'autocoppia $(\lambda_n, v^{(n)})$, associata all'autovalore $\lambda_n$ di modulo minimo, e supponiamo:
$$
|\lambda_1| \geq ... \geq |\lambda_{n - 2}| \geq |\lambda_{n - 1}| > |\lambda_n|
$$

L'idea è di applicare il metodo delle potenze ad $A^{-1}$, infatti $A^{-1}$ ha autovalori:
$$
\left| \frac{1}{\lambda_n} \right| > \left| \frac{1}{\lambda_{n - 1}} \right| \geq \left| \frac{1}{\lambda_{n - 2}} \right| \geq ... \geq \left| \frac{1}{\lambda_1} \right|
$$
ed inoltre ha come autovettori $v^{(1)}, ..., v^{(n)}$.
Quidi le successioni generate dal metodo delle potenze su $A^{-1}$ convergono all'autocoppia:
$$
\left( \frac{1}{\lambda_n}, v^{(n)} \right)
$$

Chiamiamo questo metodo anche \textbf{metodo delle potenze inverse}.

Dove chiaramente non vale la pena di calcolare l'inversa ma piuttosto conviene risolvere il sistema:
$$
A y^{(k)} = z^{(k - 1)}
$$
ad esempio sfruttando l'eliminazione di Gauss con pivoting.
Per rendere tutto più efficiente, conviene anzi precalcolare una fattorizzazione $LU$ di $A$ prima di iniziare il metodo.

In questo caso avremo:
$$
A = \Pi LU
$$
cioè si fa una fattorizzazione $LU$ con pivoting dato dalla matrice di permutazione $\Pi$, così che ad ogni passaggio il prodotto $A^{-1} z$ diventa equivalente a risolvere un sistema lineare triangolare.
In questo caso il costo diventa $O(n^3)$ per la fattorizzazione più $O(k \cdot n^2)$ per la risoluzione dei sistemi triangolari.

\subsubsection{Implementazione MATLAB del metodo delle potenze inverse}
Vediamo quindi come si può implementare una funzione per il calcolo dell'autocoppia minima in MATLAB:
\lstset{style=codestyle, language=MATLAB}
\lstinputlisting{../code/matlab/min_eigen.m}

\subsubsection{Autovalori intermedi}
Vediamo quindi come modificare il metodo delle potenze per trovare un autovalore intermedio, cioè preso $\sigma \in \mathbb{C}$ scalare arbitrario calcolare l'autocoppia $(\lambda, v)$ con $\lambda$ più vicino a $\sigma$.

In questo caso si applica il metodo delle potenze alla matrice:
$$
A \sim (A - \sigma \cdot I)^{-1}
$$
che ha gli stessi autovettori di $A$ e come autovalori ha quantità della forma:
$$
\lambda_j' = \frac{1}{\lambda_j - \sigma}
$$
In questo caso l'autovalore di modulo più grande diventa quello più vicino a $\sigma$, per cui basta applicare il metodo delle potenze.


\end{document}
