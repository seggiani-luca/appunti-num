
\documentclass[a4paper,11pt]{article}
\usepackage[a4paper, margin=8em]{geometry}

% usa i pacchetti per la scrittura in italiano
\usepackage[french,italian]{babel}
\usepackage[T1]{fontenc}
\usepackage[utf8]{inputenc}
\frenchspacing 

% usa i pacchetti per la formattazione matematica
\usepackage{amsmath, amssymb, amsthm, amsfonts}

% usa altri pacchetti
\usepackage{gensymb}
\usepackage{hyperref}
\usepackage{standalone}

% imposta il titolo
\title{Appunti Calcolo Numerico}
\author{Luca Seggiani}
\date{2025}

% disegni
\usepackage{pgfplots}
\pgfplotsset{width=10cm,compat=1.9}

% imposta lo stile
% usa helvetica
\usepackage[scaled]{helvet}
% usa palatino
\usepackage{palatino}
% usa un font monospazio guardabile
\usepackage{lmodern}

% tikz in sans
\tikzset{every picture/.style={/utils/exec={\sffamily}}}

\renewcommand{\rmdefault}{ppl}
\renewcommand{\sfdefault}{phv}
\renewcommand{\ttdefault}{lmtt}

% circuiti
\usepackage{circuitikz}
\usetikzlibrary{babel}

% disponi il titolo
\makeatletter
\renewcommand{\maketitle} {
	\begin{center} 
		\begin{minipage}[t]{.8\textwidth}
			\textsf{\huge\bfseries \@title} 
		\end{minipage}%
		\begin{minipage}[t]{.2\textwidth}
			\raggedleft \vspace{-1.65em}
			\textsf{\small \@author} \vfill
			\textsf{\small \@date}
		\end{minipage}
		\par
	\end{center}

	\thispagestyle{empty}
	\pagestyle{fancy}
}
\makeatother

% disponi teoremi
\usepackage{tcolorbox}
\newtcolorbox[auto counter, number within=section]{theorem}[2][]{%
	colback=blue!10, 
	colframe=blue!40!black, 
	sharp corners=northwest,
	fonttitle=\sffamily\bfseries, 
	title=Teorema~\thetcbcounter: #2, 
	#1
}

% disponi definizioni
\newtcolorbox[auto counter, number within=section]{definition}[2][]{%
	colback=red!10,
	colframe=red!40!black,
	sharp corners=northwest,
	fonttitle=\sffamily\bfseries,
	title=Definizione~\thetcbcounter: #2,
	#1
}

% disponi problemi
\newtcolorbox[auto counter, number within=section]{problem}[2][]{%
	colback=green!10,
	colframe=green!40!black,
	sharp corners=northwest,
	fonttitle=\sffamily\bfseries,
	title=Problema~\thetcbcounter: #2,
	#1
}

% disponi codice
\usepackage{listings}
\usepackage[table]{xcolor}

\definecolor{codegreen}{rgb}{0,0.6,0}
\definecolor{codegray}{rgb}{0.5,0.5,0.5}
\definecolor{codepurple}{rgb}{0.58,0,0.82}
\definecolor{backcolour}{rgb}{0.95,0.95,0.92}

\lstdefinestyle{codestyle}{
		backgroundcolor=\color{black!5}, 
		commentstyle=\color{codegreen},
		keywordstyle=\bfseries\color{magenta},
		numberstyle=\sffamily\tiny\color{black!60},
		stringstyle=\color{green!50!black},
		basicstyle=\ttfamily\footnotesize,
		breakatwhitespace=false,         
		breaklines=true,                 
		captionpos=b,                    
		keepspaces=true,                 
		numbers=left,                    
		numbersep=5pt,                  
		showspaces=false,                
		showstringspaces=false,
		showtabs=false,                  
		tabsize=2
}

\lstdefinestyle{shellstyle}{
		backgroundcolor=\color{black!5}, 
		basicstyle=\ttfamily\footnotesize\color{black}, 
		commentstyle=\color{black}, 
		keywordstyle=\color{black},
		numberstyle=\color{black!5},
		stringstyle=\color{black}, 
		showspaces=false,
		showstringspaces=false, 
		showtabs=false, 
		tabsize=2, 
		numbers=none, 
		breaklines=true
}

\lstdefinelanguage{javascript}{
	keywords={typeof, new, true, false, catch, function, return, null, catch, switch, var, if, in, while, do, else, case, break},
	keywordstyle=\color{blue}\bfseries,
	ndkeywords={class, export, boolean, throw, implements, import, this},
	ndkeywordstyle=\color{darkgray}\bfseries,
	identifierstyle=\color{black},
	sensitive=false,
	comment=[l]{//},
	morecomment=[s]{/*}{*/},
	commentstyle=\color{purple}\ttfamily,
	stringstyle=\color{red}\ttfamily,
	morestring=[b]',
	morestring=[b]"
}

% disponi sezioni
\usepackage{titlesec}

\titleformat{\section}
	{\sffamily\Large\bfseries} 
	{\thesection}{1em}{} 
\titleformat{\subsection}
	{\sffamily\large\bfseries}   
	{\thesubsection}{1em}{} 
\titleformat{\subsubsection}
	{\sffamily\normalsize\bfseries} 
	{\thesubsubsection}{1em}{}

% disponi alberi
\usepackage{forest}

\forestset{
	rectstyle/.style={
		for tree={rectangle,draw,font=\large\sffamily}
	},
	roundstyle/.style={
		for tree={circle,draw,font=\large}
	}
}

% disponi algoritmi
\usepackage{algorithm}
\usepackage{algorithmic}
\makeatletter
\renewcommand{\ALG@name}{Algoritmo}
\makeatother

% disponi numeri di pagina
\usepackage{fancyhdr}
\fancyhf{} 
\fancyfoot[L]{\sffamily{\thepage}}

\makeatletter
\fancyhead[L]{\raisebox{1ex}[0pt][0pt]{\sffamily{\@title \ \@date}}} 
\fancyhead[R]{\raisebox{1ex}[0pt][0pt]{\sffamily{\@author}}}
\makeatother

\begin{document}

% sezione (data)
\section{Lezione del 24-03-25}

% stili pagina
\thispagestyle{empty}
\pagestyle{fancy}

% testo
\subsection{Pivoting}
L'algoritmo di eliminazione di Gauss che abbiamo definito alla scorsa lezione ha un punto di fallimento nel caso uno degli elementi $a_{ii}^{(i - 1)}$ sia $= 0$, o comunque $\approx 0$, in quanto vorremmo a quel punto calcolare un moltiplicatore $l_{ji} = \frac{a_{ji}}{a_{ii}} \rightarrow$ non ben definito.

In tal caso si può modificare l'algoritmo sfruttando una matrice di permutazione che porti un elemento diverso da zero nella stessa posizione di $a_{ii}$.
Vorremo quindi cercare un indice $h$ tal per cui $a_{hi}^{(i - 1)}$ sia di modulo massimo nella sua colonna al di sotto di $i$, cioè:
$$
a_{hi}^{(i - 1)} \geq \max_{j = i, ..., n} | a_{ji}^{(i - 1)} |
$$
e scambiare la riga $i$ con la riga $h$.
Infatti, se $\det(A) \neq 0$, allora necessariamente esiste un $a_{ji}^{(i - 1)} \neq 0$ (altrimenti si ha uno 0 obbligato sulla diagonale, che con la matrice triangolare a blocchi dà $\det(A^{(i - 1)}) = 0$). \qed

Un altra conseguenza di questo approccio è che tutti i moltiplicatori $l_{ji}$ diventeranno $\leq 1$.
L'algoritmo di Gauss con questa modifica si chiama \textbf{eliminazione di Gauss con pivoting parziale} (\textit{parziale} perché ne esistono versioni più sofisticate, che non vedremo).

Osserviamo che ogni scambio di righe equivale a moltiplicare a sinistra per una matrice di permutazione $\Pi_i$.
Quindi il metodo di Gauss con pivoting può essere rappresentato come:
\begin{algorithm}
\caption{Eliminazione di Gauss con pivoting parziale}
\begin{algorithmic}
	\STATE \textbf{Input:} un sistema lineare qualsiasi $Ax = b$ % input
	\STATE \textbf{Output:} un sistema lineare triangolare superiore $Ux = c$ % output
	% body
	\FOR{$i = 1$ to $n$}
		\STATE Trova la matrice $\Pi_i$ che porta l'elemento di modulo massimo in testa
		\STATE $A \leftarrow \Pi_i A$
		\FOR{$j = i$ to $n$}
			\STATE Calcola il \textbf{moltiplicatore} $l_{ji} = \frac{a_{ji}^{(i - 1)}}{a_{ii}^{(i - 1)}}$
			\STATE Aggiungi alla riga $j$ la riga $i$ moltiplicata per $l_{ij}$
		\ENDFOR
	\ENDFOR
\end{algorithmic}
\end{algorithm}

\par\smallskip

Vediamo un esempio pratico dell'algoritmo prima di procedere all'implementazione MATLAB.
Prendiamo la matrice $A$ e il vettore $b$:
$$
A = \begin{pmatrix}
	1 & 2 & 3 \\ 
	4 & 5 & 6 \\ 
	7 & 8 & 0
\end{pmatrix}, \quad
b = \begin{pmatrix}
	1 \\ 2 \\ 3
\end{pmatrix}
$$
Nel ridurre la matrice aumentata $Ab$:
$$
\left(\begin{array}{@{}c c c | c@{}}
	1 & 2 & 3 & 1 \\ 
	4 & 5 & 6 & 2 \\ 
	7 & 8 & 0 & 3
\end{array}\right)
$$
ci accorgiamo che alla prima colonna l'entrata di modulo massimo è 7, di indice 3. Si permutano quindi la prima e la terza riga:
$$
\xrightarrow{\Pi_1}
\left(\begin{array}{@{}c c c | c@{}}
	7 & 8 & 0 & 3 \\
	4 & 5 & 6 & 2 \\ 
	1 & 2 & 3 & 1
\end{array}\right)
\xrightarrow{H_1 \Pi_1}
\left(\begin{array}{@{}c c c | c@{}}
		7 & 8 & 0 & 3 \\ 
		0 & \frac{3}{7} & 6 & \frac{2}{7} \\ 
		0 & \frac{6}{7} & 3 & \frac{4}{7}
\end{array}\right)
$$
nuovamente, l'entrata di modulo massimo è all'indice 3. Si permutano quindi la seconda e la terza riga:
$$
\xrightarrow{\Pi_2 H_1 \Pi_1}
\left(\begin{array}{@{}c c c | c@{}}
		7 & 8 & 0 & 3 \\ 
		0 & \frac{6}{7} & 3 & \frac{4}{7} \\
		0 & \frac{3}{7} & 6 & \frac{2}{7} 
\end{array}\right)
\xrightarrow{H_2 \Pi_2 H_1 \Pi_1}
\left(\begin{array}{@{}c c c | c@{}}
		7 & 8 & 0 & 3 \\ 
		0 & \frac{6}{7} & 3 & \frac{4}{7} \\
		0 & 0 & \frac{9}{2} & 0
\end{array}\right)
$$

\subsubsection{Implementazione MATLAB del metodo di eliminazione di Gauss con pivoting}
Modifichiamo quindi la funzione \lstinline|gauss_decomp()| per introdurre il meccanismo di pivoting appena visto:
# pivoting

\subsubsection{Fattorizzazione LU con pivoting}
Vediamo come ricavare una fattorizzazione LU dal metodo di Gauss modificato con il pivoting.
Si ha quindi che la matrice $U$ si evolve come:
$$
A \rightarrow \Pi_1 A \rightarrow H_1 \Pi_1 A \rightarrow ... \rightarrow H_{n - 1} \Pi_{n - 1} ... H_1 \Pi_1 A = U 
$$
mentre per la $L$ dovremo notare che:
$$
LU = \Pi A
$$
dove la matrice $\Pi$ rappresenta tutte le permutazioni fatte sulle righe di $A$.

Si ha quindi che:
\begin{itemize}
	\item $U$ è la matrice triangolare superiore trovata alla fine del metodo di Gauss con pivoting;
	\item $L$ è la matrice dei moltiplicatori, a cui però si defono applicare gli scambi delle righe, come segue: se al passo $i$ applico la matrice $\Pi_i$, devo applicare lo stesso cambio nelle prime $i - 1$ colonne di $L$.
\end{itemize}

\par\smalskip 

Vediamo un esempio numerico spieghi il processo di formazione della matrice $L$ e della matrice di permutazione $\Pi$.
Presa la stessa matrice dell'esempio precedente:
$$
A = \begin{pmatrix}
	1 & 2 & 3 \\ 
	4 & 5 & 6 \\ 
	7 & 8 & 0
\end{pmatrix}
$$
abbiamo che le permutazioni sono, in sequenza:
$$
\Pi_1 : 
\begin{pmatrix}
	1 \\ 2 \\ 3
\end{pmatrix}
\rightarrow
\begin{pmatrix}
	3 \\ 2 \\ 1
\end{pmatrix}, \quad 
\Pi_2 :
\begin{pmatrix}
	3 \\ 2 \\ 1
\end{pmatrix}
\rightarrow
\begin{pmatrix}
	3 \\ 1 \\ 2
\end{pmatrix}
$$
o, in forma matriciale:
$$
\Pi_1 = \begin{pmatrix}
	0 & 0 & 1 \\ 
	0 & 1 & 0 \\ 
	1 & 0 & 0
\end{pmatrix}, \quad
\Pi_2 = \begin{pmatrix}
	1 & 0 & 0 \\ 
	0 & 0 & 1 \\ 
	0 & 1 & 0
\end{pmatrix}
$$

Calcoliamo quindi $L$.
$\Pi_1$ è irrilevante al calcolo di $L$, quindi la ignoriamo.
Vediamo che i primi due moltiplicatori sono $l_{21} = \frac{4}{7}$ e $l_{31} = \frac{1}{7}$, da cui si imposta $L^{(1)}$:
$$
L^{(1)} = \begin{pmatrix}
	1 & 0 & 0 \\ 
	\frac{4}{7} & 1 & 0 \\ 
	\frac{1}{7} & 0 & 1
\end{pmatrix}
$$

Notiamo quindi che dalla $\Pi_2$ dobbiamo scambiare gli elementi sotto la diagonale della prima colonna, quindi:
$$
L^{(1)} = \begin{pmatrix}
	1 & 0 & 0 \\ 
	\frac{1}{7} & 1 & 0 \\ 
	\frac{4}{7} & 0 & 1
\end{pmatrix}
$$

Infine, l'ultimo moltiplicatore $l_{32} = \frac{1}{2}$ non ha ambiguità:
$$
L^{(2)} = L = \begin{pmatrix}
	1 & 0 & 0 \\ 
	\frac{1}{7} & 1 & 0 \\ 
	\frac{4}{7} & \frac{1}{2} & 1
\end{pmatrix}
$$

Il calcolo di $\Pi$ deriva invece direttamente studiando la permutazione complessiva data dalle $\Pi_1, ..., \Pi_{n - 1}$, in questo caso:
$$
\Pi : \begin{pmatrix}
	1 \\ 2 \\ 3
\end{pmatrix}
\xrightarrow{\Pi_1}
\begin{pmatrix}
	3 \\ 2 \\ 1
\end{pmatrix}
\xrightarrow{\Pi_2}
\begin{pmatrix}
	3 \\ 1 \\ 2
\end{pmatrix}
$$
da cui:
$$
\Pi = \begin{pmatrix}
	0 & 0 & 1 \\ 
	1 & 0 & 0 \\ 
	0 & 1 & 0
\end{pmatrix}
$$

Con brevi calcoli si verifica che:
$$
LU = \Pi A
$$

\par\medskip 

\subsubsection{Implementazione MATLAB completa del metodo di eliminazione di Gauss con pivoting}
Vediamo quindil'implementazione completa, che calcola anche la matrice $L$ e la matrice $\Pi$.
Notiamo inoltre l'argomento condizionale $b$, che viene ignorato se non fornito abbiamo (constatato che spesso è così).

# algo

\subsubsection{Determinante con pivoting}
Possiamo sfruttare la matrice $\Pi$ per il calcolo del determinante.
Si ha infatti dal teorema di Binet-Cauchy che:
$$
\det(\Pi) \det(A) = \det(\Pi A) = \det(L U) = \det(L) \det(U)
$$
e quindi:
$$
\det(A) = \det(\Pi)^{-1} \det(L) \det(U)
$$
dove $\det(L) = 1$ (triangolare inferiore con diagonale di 1).
Si nota poi che $\det(\Pi)^{-1}$ è $(-1)^s$ è il numero di pivot che effettuiamo.
A questo punto $\det(U)$ è semplicemente il prodotto degli elementi sulla diagonale (triangolare superiore), cioè:
$$
\prod_{i = 1}^n a_{ii}^{(i - 1)} = \prod_{i = 1}^n u_{ii}
$$
e quindi:
$$
\det(A) = (-1)^s \prod_{i = 1}^n a_{ii}^{(i - 1)} = (-1)^s \prod_{i = 1}^n u_{ii}
$$
Si può quindi usare il metodo di Gauss, ancora una volta, per il calcolo del determinante di una matrice, con costo pari al costo dell'eliminazione di Gauss ($O(\frac{2}{3}n^3)$), molto meglio dello sviluppo di Laplace! ($O(n!)$).

# implementa 

\subsection{Condizionamento di un sistema lineare}
Date $A \in \mathbb{C}^{n \times n}$ e $b \in \mathbb{C}^n$, supponiamo di voler trovare $Ax = b$ ma a causa di errori nei dati o errori di arrotondamento troviamo (con un qualunque metodo) un vettore perturbato $x + \delta x \in \mathbb{C}^n$ che risolve un sistema lineare di per sé perturbato:
$$
(A + \delta A) (x + \delta x) = (b + \delta b)
$$
con $\delta A$ e $\delta B$ perturbazioni "piccole" della matrice $A$ e del vettore $b$, quindi $A + \delta A \in \mathbb{C}^{n \times n}$ e $b + \delta b \in \mathbb{C}^n$

La domanda è, se $\delta A$ e $\delta b$ sono piccole, posso concludere che anche $\delta x$ è relativamente piccolo?
Si scopre che la risposta a questa domanda è generalmente no.
Prendiamo ad esempio il sistema $2 \times 2$:
$$
\begin{pmatrix}
	1 & -1 \\
	1 & 1.000001
\end{pmatrix}
\begin{pmatrix}
	x_1 \\ x_2
\end{pmatrix}
=
\begin{pmatrix}
	1 \\ 1
\end{pmatrix}
$$
da cui $x$ esatto è $\begin{pmatrix}
	1 & 0
\end{pmatrix}$.
Perturbando $b$ a $\begin{pmatrix}
	0.999999 & 1
\end{pmatrix}$, si ha $x$ perturbato a $\begin{pmatrix}
	-10^{-6} & -1
\end{pmatrix}$, che è chiaramente un cambiamento drastico. # riguarda
Viene da sé che agendo sulla matrice $A$ potremo ottenere effetti anche più drammatici.

Riprendiamo quindi la definizione di errore relativo:
$$
\epsilon = \frac{|\delta x|}{|x|}
$$
assumendo $\delta A = 0$ con $\det(A) \neq 0$, come nell'esempio precedente, e quindi perturbazioni solo del termine noto, si ha:
$$
A (x + \delta x) = (b + \delta b) \ \Leftrightarrow \ A\delta x = \delta b
$$
visto che $Ax = b$. Passando alle norme, si ha che:
$$
| \delta x | \leq |A^{-1}| \cdot |\delta b|
$$
e inoltre:
$$
|Ax| = |b| \implies |A| |x| \geq |b| \implies |x| \geq \frac{|b|}{|A|}
$$
quindi:
$$
\frac{|\delta x|}{|x|} \leq \frac{|A^{-1}| \cdot |\delta b| \cdot |A|}{|b|} = \frac{|\delta b|}{|b|} \cdot |A| \cdot |A^{-1}|
$$
dove ci interessa il valore $|A| \cdot |A^{-1}|$, l'unico che non dipende dall'errore assoluto $\delta b$.
\begin{definition}{Numero di condizionamento}
	Chiamiamo numero di condizionamento di una matrice $A$, data una certa norma $|\cdot|$, il valore:
	$$
		\mu(A) = |A| \cdot |A^{-1}|
	$$
\end{definition}

Abbiamo quindi che se $\mu(A) >> 1$, allora l'errore relativo può essere molto più grande dell'errore relativo dei dati e il problema si dice \textit{mal condizionato}.

\end{document}
