
\documentclass[a4paper,11pt]{article}
\usepackage[a4paper, margin=8em]{geometry}

% usa i pacchetti per la scrittura in italiano
\usepackage[french,italian]{babel}
\usepackage[T1]{fontenc}
\usepackage[utf8]{inputenc}
\frenchspacing 

% usa i pacchetti per la formattazione matematica
\usepackage{amsmath, amssymb, amsthm, amsfonts}

% usa altri pacchetti
\usepackage{gensymb}
\usepackage{hyperref}
\usepackage{standalone}

% imposta il titolo
\title{Appunti Calcolo Numerico}
\author{Luca Seggiani}
\date{2025}

% disegni
\usepackage{pgfplots}
\pgfplotsset{width=10cm,compat=1.9}

% imposta lo stile
% usa helvetica
\usepackage[scaled]{helvet}
% usa palatino
\usepackage{palatino}
% usa un font monospazio guardabile
\usepackage{lmodern}

% tikz in sans
\tikzset{every picture/.style={/utils/exec={\sffamily}}}

\renewcommand{\rmdefault}{ppl}
\renewcommand{\sfdefault}{phv}
\renewcommand{\ttdefault}{lmtt}

% circuiti
\usepackage{circuitikz}
\usetikzlibrary{babel}

% disponi il titolo
\makeatletter
\renewcommand{\maketitle} {
	\begin{center} 
		\begin{minipage}[t]{.8\textwidth}
			\textsf{\huge\bfseries \@title} 
		\end{minipage}%
		\begin{minipage}[t]{.2\textwidth}
			\raggedleft \vspace{-1.65em}
			\textsf{\small \@author} \vfill
			\textsf{\small \@date}
		\end{minipage}
		\par
	\end{center}

	\thispagestyle{empty}
	\pagestyle{fancy}
}
\makeatother

% disponi teoremi
\usepackage{tcolorbox}
\newtcolorbox[auto counter, number within=section]{theorem}[2][]{%
	colback=blue!10, 
	colframe=blue!40!black, 
	sharp corners=northwest,
	fonttitle=\sffamily\bfseries, 
	title=Teorema~\thetcbcounter: #2, 
	#1
}

% disponi definizioni
\newtcolorbox[auto counter, number within=section]{definition}[2][]{%
	colback=red!10,
	colframe=red!40!black,
	sharp corners=northwest,
	fonttitle=\sffamily\bfseries,
	title=Definizione~\thetcbcounter: #2,
	#1
}

% disponi problemi
\newtcolorbox[auto counter, number within=section]{problem}[2][]{%
	colback=green!10,
	colframe=green!40!black,
	sharp corners=northwest,
	fonttitle=\sffamily\bfseries,
	title=Problema~\thetcbcounter: #2,
	#1
}

% disponi codice
\usepackage{listings}
\usepackage[table]{xcolor}

\definecolor{codegreen}{rgb}{0,0.6,0}
\definecolor{codegray}{rgb}{0.5,0.5,0.5}
\definecolor{codepurple}{rgb}{0.58,0,0.82}
\definecolor{backcolour}{rgb}{0.95,0.95,0.92}

\lstdefinestyle{codestyle}{
		backgroundcolor=\color{black!5}, 
		commentstyle=\color{codegreen},
		keywordstyle=\bfseries\color{magenta},
		numberstyle=\sffamily\tiny\color{black!60},
		stringstyle=\color{green!50!black},
		basicstyle=\ttfamily\footnotesize,
		breakatwhitespace=false,         
		breaklines=true,                 
		captionpos=b,                    
		keepspaces=true,                 
		numbers=left,                    
		numbersep=5pt,                  
		showspaces=false,                
		showstringspaces=false,
		showtabs=false,                  
		tabsize=2
}

\lstdefinestyle{shellstyle}{
		backgroundcolor=\color{black!5}, 
		basicstyle=\ttfamily\footnotesize\color{black}, 
		commentstyle=\color{black}, 
		keywordstyle=\color{black},
		numberstyle=\color{black!5},
		stringstyle=\color{black}, 
		showspaces=false,
		showstringspaces=false, 
		showtabs=false, 
		tabsize=2, 
		numbers=none, 
		breaklines=true
}

\lstdefinelanguage{javascript}{
	keywords={typeof, new, true, false, catch, function, return, null, catch, switch, var, if, in, while, do, else, case, break},
	keywordstyle=\color{blue}\bfseries,
	ndkeywords={class, export, boolean, throw, implements, import, this},
	ndkeywordstyle=\color{darkgray}\bfseries,
	identifierstyle=\color{black},
	sensitive=false,
	comment=[l]{//},
	morecomment=[s]{/*}{*/},
	commentstyle=\color{purple}\ttfamily,
	stringstyle=\color{red}\ttfamily,
	morestring=[b]',
	morestring=[b]"
}

% disponi sezioni
\usepackage{titlesec}

\titleformat{\section}
	{\sffamily\Large\bfseries} 
	{\thesection}{1em}{} 
\titleformat{\subsection}
	{\sffamily\large\bfseries}   
	{\thesubsection}{1em}{} 
\titleformat{\subsubsection}
	{\sffamily\normalsize\bfseries} 
	{\thesubsubsection}{1em}{}

% disponi alberi
\usepackage{forest}

\forestset{
	rectstyle/.style={
		for tree={rectangle,draw,font=\large\sffamily}
	},
	roundstyle/.style={
		for tree={circle,draw,font=\large}
	}
}

% disponi algoritmi
\usepackage{algorithm}
\usepackage{algorithmic}
\makeatletter
\renewcommand{\ALG@name}{Algoritmo}
\makeatother

% disponi numeri di pagina
\usepackage{fancyhdr}
\fancyhf{} 
\fancyfoot[L]{\sffamily{\thepage}}

\makeatletter
\fancyhead[L]{\raisebox{1ex}[0pt][0pt]{\sffamily{\@title \ \@date}}} 
\fancyhead[R]{\raisebox{1ex}[0pt][0pt]{\sffamily{\@author}}}
\makeatother

\begin{document}

% sezione (data)
\section{Lezione del 05-05-25}

% stili pagina
\thispagestyle{empty}
\pagestyle{fancy}

% testo
Continuiamo la trattazione dell'approssimazione di integrali attraverso le formule di quadratura interpolatorie.
Un problema che avevamo fino adesso è che se $h$ è grande, allora $E_n(f)$ è grande per qualche termine proporzionale a $h^d$, dove $d$ è il grado dell'errore.

Un'idea potrebbe essere di aumentare $n$, quindi aumentare i punti campionati per ridurre il passo $h$, ma come abbiamo visto questo è instabile (risente di fenomeni simili al fenomeno di Runge).

\subsubsection{Formule di Newton-Cotes generalizzate}
Decidiamo quindi di spezzare l'integrale su sottointervalli, e in ciascuno di essi applicare una formula di quadratura.
Questo metodo ci porterà alle formule di Newton-Cotes \textbf{generalizzate}, anche dette \textit{composite}.

Quindi, posto ad esempio un certo intervallo $[a, b]$ contenente i punti ordinati $c, d, e$, dividiamo l'integrale:
$$
\int_a^b f(x) \, dx = \int_a^c f(x) \, dx + \int_c^d f(x) \, dx + \int_d^e f(x) \, dx + \int_e^b f(x) \, dx
$$
e si applica la formula di quadratura ad ogni sottointervallo, sommando.

Più formalmente, avremo che si divide $[a, b]$ in $L$ sottointervalli equispaziati, con:
$$
x_0 = a, \quad x_1 = x_0 + \frac{b - a}{L}, \quad ..., \quad x_L = b
$$
e si divide l'integrale come:
$$
\int_a^b f(x) \, dx = \sum_{i = 1}^L \int_{x_{i - 1}}^{x_i} f(x) \, dx
$$
In ogni intervallo $[x_{i - 1}, x_i]$ si applica quindi una formula di Newton Cotes con $n + 1$ nodi.

Avremo quindi bisogno, in ogni sottointervallo, di:
$$
n + 1 - 2 = n - 1
$$ 
nodi aggiuntivi oltre agli estremi dell'intervallo per poter applicare la formula di quadratura $J_n(f)$.
Il numero di nodi totali dovrà quindi essere:
$$
L + 1 + (n - 1) \cdot L = nL + 1
$$

Osserviamo che il numero di nodi corrisponde al numero di valutazioni di $f$, e quindi in genere è un'indicazione del costo computazionale di una formula. 

\subsubsection{Esempio: formula dei trapezi generalizzata}
Prendiamo la forma che otteniamo per $n = 1$:
$$
\int_a^b f(x) \, dx \approx \sum_{i = 1}^L \frac{b - a}{2L} \left( f(x_{i - 1}) + f(x_i) \right) = \frac{b - a}{2L} \sum_{i = 1}^L \left( f(x_{i - 1}) + f(x_i) \right)
$$
$$
= \frac{b - a}{2L} \left( f(x_0) + 2 \sum_{i = 1}^{L - 1} f(x_i) + f(x_L) \right) = J_1^{(G)} (f)
$$

\subsubsection{Esempio: formula di Cavalieri-Simspon}
Vediamo quindi la generalizzazione della formula di Cavalieri, cioè ciò che otteniamo per $n = 2$:
$$
\int_a^b f(x) \, dx \approx \sum_{i = 1}^L \frac{b - a}{6L} \left( f(x_{i - 1}) + 4 f \left( \frac{x_{i - 1} + x_i}{2} \right) + f(x_i) \right)
$$
$$
= \frac{b - a}{6L} \left( f(x_0) + 2 \sum_{i = 1}^{L-1} f(x_i) + 4 \sum_{i = 1}^{L} f\left( \frac{x_{i - 1} + x_i}{2} \right) + f(x_L) \right) = J_2^{(G)} (f)
$$

\subsubsection{Errore nelle formule generalizzate}
In ogni intervallo della forma $[x_{i - 1}, x_i]$ conosciamo l'errore, che è quello della formula di Newton-Cotes che stiamo utilizzando:
$$
e \cdot h^d \cdot f^{(d - 1)} (\varepsilon), \quad f \in C^{(d - 1)} ([a, b])
$$
L'errore sulla formula totale $J_n^{(G)}(f)$ sarà quindi:
$$
E_n^{(G)} (f) = I(f) - J_n^{(G)} (f) = \sum_{i = 1}^L c \cdot h^d \cdot f^{(d - 1)}(\varepsilon_i), \quad \varepsilon_i \in [x_{i - 1}, x_i]
$$
Vediamo quindi che l'unico termine dipendente dall'intervallo è $f^{(d - 1)} (\varepsilon)$, per cui possiamo dire:
$$
= \sum_{i = 1}^L c \cdot h^d \cdot L \frac{f^{(d - 1)}(\varepsilon_i)}{L} = c h^dL \cdot f^{(d - 1)} (\varepsilon), \quad \varepsilon \in [a, b]
$$
sfruttando il teorema della media integrale.

Abbiamo quindi gli errori:
\begin{itemize}
	\item \textbf{Formula dei trapezi:}
		$$
		E_1^{(G)}(f) = -\frac{(b - a)^3}{12L^2} \cdot f''(\varepsilon)
		$$
	\item \textbf{Formula di Cavalieri-Simpson:}
		$$
		E_2^{(G)}(f) = -\frac{(b - a)^5}{2880L^4} \cdot f^{(4)} (\varepsilon), \quad \varepsilon \in [a, b]
		$$
\end{itemize}
e il caso generale:
$$
E_n^{(G)} (f) =
\begin{cases}
	c_n \cdot \frac{(b - a)^{n + 2}}{L^{n + 1}} \cdot f^{(n + 1)} (\varepsilon), \quad n \text{ dispari} \\
	c_n \cdot \frac{(b - a)^{n + 3}}{L^{n + 2}} \cdot f^{(n + 2)} (\varepsilon), \quad n \text{ pari} \\
\end{cases}
$$

\subsection{Formule di quadratura sulle derivate}
Potremmo considerare formule di quadratura che coinvolgono le derivate di $f$.
Si possono quindi generalizzare i concetti di grado di precisione, scelta dei nodi e pesi ottimali anche per formule di quadratura del tipo:
$$
\int_a^b f(x) \rho(x) \, dx = \sum_{i = 0}^{n_1} f(x_i) + \sum_{i = 0}^{n_2} f'(y_i) + \sum_{i = 0}^{n_3} f''(z_i)
$$
chiaramente restringendo l'applicazione delle formule a funzioni sufficientemente derivabili.

Vediamo ad esempio come trovare i pesi ottimi per l'approssimazione con:
$$
\int_0^1 f(x) \, dx = I(f) \approx a_1 f(0) + a_2 f' \left( \frac{1}{3} \right) + a_3 f' \left( \frac{2}{3} \right) + f(4)
$$
sull'intervallo $[a, b] = [0, 1]$.

Imponendo errore nullo ai primi 4 gradi si otterrà:
\[
	\begin{cases}
		E_1(1)= 0 \\
		E_1(x)= 0 \\
		E_1(x^2)= 0 \\
		E_1(x^3)= 0 
	\end{cases}
	\implies
	\begin{cases}
		\int_0^1 1 \, dx = 1 = a_1 + a_4 \\ 
		\int_0^1 x \, dx = \frac{1}{2} = a_2 + a_3 + a_4 \\
		\int_0^1 x^2 \, dx = \frac{1}{3} = \frac{2}{3} a_2 + \frac{4}{3} a_3 + a_4 \\
		\int_0^1 x^3 \, dx = \frac{1}{4} = \frac{1}{3} a_2 + \frac{4}{3} a_3 + a_4
	\end{cases}
\]
Ricaviamo allora dalle prime 3:
\[
	\begin{cases}
		a_4 = 1 - a_1 \\
		a_2 = \frac{1}{2} - a_3 - a_4 \\ 
		a_3 = \frac{3}{4} \left( \frac{1}{3} - \frac{2}{3} a_2 - a_4 \right) = \frac{1}{4} - \frac{1}{2} a_2 - \frac{3}{4} a_4
\end{cases}
\]
da cui sostituendo ulteriormente:
\[
	\begin{cases}
		a_4 = 1 - a_1 \\ 
		a_2 = \frac{1}{2} + \frac{1}{2} a_4 - a_4 = \frac{1}{2} - \frac{1}{2}a_4 \\ 
		a_3 = \frac{1}{4} - \frac{1}{2} \left( \frac{1}{2} - a_3 - a_4 \right) - \frac{3}{4} a_4 = \frac{1}{2} a_3 - \frac{1}{4}a_4
	\end{cases}
\]
sostituendo tutto nella quarta si ha:
$$
\frac{1}{4} = \frac{1}{3} \left( \frac{1}{2} - \frac{1}{2} a_4 \right) + \frac{4}{3} \cdot - \frac{1}{2} a_4 + a_4 = \frac{1}{6} + \frac{1}{6} a_4 \implies a_4 \frac{1}{2} 
$$
da cui immediatamente:
$$
a_1 = \frac{1}{2}, \quad
a_2 = \frac{1}{4}, \quad
a_3 = -\frac{1}{4}, \quad
a_4 = \frac{1}{2}, \quad
$$
da cui:
$$
J_3(f) = \frac{f(0)}{2} + \frac{f' \left( \frac{1}{3} \right)}{4} - \frac{f'' \left( \frac{2}{3} \right)}{4} + \frac{f(1)}{2}
$$

Per determinare il grado di precisione vediamo l'errore per $x^4$:
$$
E_3(x^4) = \int_0^1 x^4 \, dx - \left( \frac{1}{2} \cdot 0 + \frac{1}{4} \cdot 4 \cdot \left( \frac{1}{3} \right)^3 - \frac{1}{4} \cdot 4 \cdot \left( \frac{2}{3} \right)^3 + \frac{1}{2} \right) = \frac{1}{5} - \left( \frac{1}{27} - \frac{8}{27} + \frac{1}{2} \right) = \frac{1}{5} - \frac{13}{54} \neq 0
$$
da cui il grado di precisione è esattamente 3.

\end{document}
