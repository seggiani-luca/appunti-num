
\documentclass[a4paper,11pt]{article}
\usepackage[a4paper, margin=8em]{geometry}

% usa i pacchetti per la scrittura in italiano
\usepackage[french,italian]{babel}
\usepackage[T1]{fontenc}
\usepackage[utf8]{inputenc}
\frenchspacing 

% usa i pacchetti per la formattazione matematica
\usepackage{amsmath, amssymb, amsthm, amsfonts}

% usa altri pacchetti
\usepackage{gensymb}
\usepackage{hyperref}
\usepackage{standalone}

% imposta il titolo
\title{Appunti Calcolo Numerico}
\author{Luca Seggiani}
\date{2025}

% disegni
\usepackage{pgfplots}
\pgfplotsset{width=10cm,compat=1.9}

% imposta lo stile
% usa helvetica
\usepackage[scaled]{helvet}
% usa palatino
\usepackage{palatino}
% usa un font monospazio guardabile
\usepackage{lmodern}

% tikz in sans
\tikzset{every picture/.style={/utils/exec={\sffamily}}}

\renewcommand{\rmdefault}{ppl}
\renewcommand{\sfdefault}{phv}
\renewcommand{\ttdefault}{lmtt}

% circuiti
\usepackage{circuitikz}
\usetikzlibrary{babel}

% disponi il titolo
\makeatletter
\renewcommand{\maketitle} {
	\begin{center} 
		\begin{minipage}[t]{.8\textwidth}
			\textsf{\huge\bfseries \@title} 
		\end{minipage}%
		\begin{minipage}[t]{.2\textwidth}
			\raggedleft \vspace{-1.65em}
			\textsf{\small \@author} \vfill
			\textsf{\small \@date}
		\end{minipage}
		\par
	\end{center}

	\thispagestyle{empty}
	\pagestyle{fancy}
}
\makeatother

% disponi teoremi
\usepackage{tcolorbox}
\newtcolorbox[auto counter, number within=section]{theorem}[2][]{%
	colback=blue!10, 
	colframe=blue!40!black, 
	sharp corners=northwest,
	fonttitle=\sffamily\bfseries, 
	title=Teorema~\thetcbcounter: #2, 
	#1
}

% disponi definizioni
\newtcolorbox[auto counter, number within=section]{definition}[2][]{%
	colback=red!10,
	colframe=red!40!black,
	sharp corners=northwest,
	fonttitle=\sffamily\bfseries,
	title=Definizione~\thetcbcounter: #2,
	#1
}

% disponi problemi
\newtcolorbox[auto counter, number within=section]{problem}[2][]{%
	colback=green!10,
	colframe=green!40!black,
	sharp corners=northwest,
	fonttitle=\sffamily\bfseries,
	title=Problema~\thetcbcounter: #2,
	#1
}

% disponi codice
\usepackage{listings}
\usepackage[table]{xcolor}

\definecolor{codegreen}{rgb}{0,0.6,0}
\definecolor{codegray}{rgb}{0.5,0.5,0.5}
\definecolor{codepurple}{rgb}{0.58,0,0.82}
\definecolor{backcolour}{rgb}{0.95,0.95,0.92}

\lstdefinestyle{codestyle}{
		backgroundcolor=\color{black!5}, 
		commentstyle=\color{codegreen},
		keywordstyle=\bfseries\color{magenta},
		numberstyle=\sffamily\tiny\color{black!60},
		stringstyle=\color{green!50!black},
		basicstyle=\ttfamily\footnotesize,
		breakatwhitespace=false,         
		breaklines=true,                 
		captionpos=b,                    
		keepspaces=true,                 
		numbers=left,                    
		numbersep=5pt,                  
		showspaces=false,                
		showstringspaces=false,
		showtabs=false,                  
		tabsize=2
}

\lstdefinestyle{shellstyle}{
		backgroundcolor=\color{black!5}, 
		basicstyle=\ttfamily\footnotesize\color{black}, 
		commentstyle=\color{black}, 
		keywordstyle=\color{black},
		numberstyle=\color{black!5},
		stringstyle=\color{black}, 
		showspaces=false,
		showstringspaces=false, 
		showtabs=false, 
		tabsize=2, 
		numbers=none, 
		breaklines=true
}

\lstdefinelanguage{javascript}{
	keywords={typeof, new, true, false, catch, function, return, null, catch, switch, var, if, in, while, do, else, case, break},
	keywordstyle=\color{blue}\bfseries,
	ndkeywords={class, export, boolean, throw, implements, import, this},
	ndkeywordstyle=\color{darkgray}\bfseries,
	identifierstyle=\color{black},
	sensitive=false,
	comment=[l]{//},
	morecomment=[s]{/*}{*/},
	commentstyle=\color{purple}\ttfamily,
	stringstyle=\color{red}\ttfamily,
	morestring=[b]',
	morestring=[b]"
}

% disponi sezioni
\usepackage{titlesec}

\titleformat{\section}
	{\sffamily\Large\bfseries} 
	{\thesection}{1em}{} 
\titleformat{\subsection}
	{\sffamily\large\bfseries}   
	{\thesubsection}{1em}{} 
\titleformat{\subsubsection}
	{\sffamily\normalsize\bfseries} 
	{\thesubsubsection}{1em}{}

% disponi alberi
\usepackage{forest}

\forestset{
	rectstyle/.style={
		for tree={rectangle,draw,font=\large\sffamily}
	},
	roundstyle/.style={
		for tree={circle,draw,font=\large}
	}
}

% disponi algoritmi
\usepackage{algorithm}
\usepackage{algorithmic}
\makeatletter
\renewcommand{\ALG@name}{Algoritmo}
\makeatother

% disponi numeri di pagina
\usepackage{fancyhdr}
\fancyhf{} 
\fancyfoot[L]{\sffamily{\thepage}}

\makeatletter
\fancyhead[L]{\raisebox{1ex}[0pt][0pt]{\sffamily{\@title \ \@date}}} 
\fancyhead[R]{\raisebox{1ex}[0pt][0pt]{\sffamily{\@author}}}
\makeatother

\begin{document}

% sezione (data)
\section{Lezione del 04-04-25}

% stili pagina
\thispagestyle{empty}
\pagestyle{fancy}

% testo
Riprendiamo il discorso dei problemi ai minimi quadrati.

\subsection{Esistenza delle soluzioni al sistema delle equazioni normali}
Avevamo detto che di sistemi $Ax = b$ con $A \in \mathbb{C}^{m \times n}$, $m > n$, cioè \textit{sovradeterminati}, che se la soluzione di $Ax = b$ non esiste, ergo $\mathrm{rank}(A|b) > \mathrm{rank}(A)$, potrebbe essere interessante cercare di risolvere il problema di ottimizzazione:
$$
\min_{x \in \mathbb{C}^n} |Ax - b|_2 = \min_{x \in \mathbb{C}^n} |b - Ax|_2
$$
dove il punto $p_1$ di minimo è lo stesso di quello del problema:
$$
\min_{x \in \mathbb{C}^n} |Ax - b|_2^2
$$
Avevamo quindi definito la funzione $\phi$, assunta per semplicità $\psi:\mathbb{R}^n \rightarrow [0, +\infty)$:
$$
\psi(x) = |b - Ax|_2^2 = |r(x)|_2^2 = \sum_{i = 1}^n r_i(x)^2
$$
A questo punto il minimo sarà semplicemente nel punto:
$$
\begin{pmatrix}
	\frac{\partial \psi}{\partial x_1}(x) \\
	\vdots \\
	\frac{\partial \psi}{\partial x_n}(x)
\end{pmatrix}
=
\begin{pmatrix}
	0 \\ 
	\vdots \\
	0
\end{pmatrix}
$$
da cui avevamo visto ricavamo il sistema \textbf{lineare} (in quanto $\psi$ è quadratica e quindi deriva ad una lineare):
$$
A^T A x = A^T b
$$

Non lo abbiamo dimostrato formalmente, ma possiamo dire che lo stesso vale nel campo complesso, cioè con $x \in \mathbb{C}^n$:
$$
A^H A x = A^H b
$$
prese le hermitiane invece delle trasposte.

Vediamo quindi che vale il seguente teorema:
\begin{theorem}{Esistenza delle soluzioni dei problemi ai minimi quadrati}
	Il sistema $A^T A x = A^T b$ ammette soluzione $\forall A \in \mathbb{R}^{m \times n}$ e $\forall b \in \mathbb{C}^m$ (1).
	Inoltre, la soluzione è unica se e solo se $\mathrm{rank}(A) = n$, $(m > n)$ (2).
\end{theorem}

Dimostriamo le due tesi in ordine.
\begin{enumerate}
	\item 
La prima tesi si dimostra verificando che $A^T b \in I_m(A^T A)$ con $I_m$ \textit{immagine} o \textbf{spazio delle colonne} di $A^T A$, cioè verificando che è rispettata la condizione:
$$
\mathrm{rank}(A^T A | A^T b) = \mathrm{rank}(A^T A)
$$
di Rouché-Capelli, per cui esiste una soluzione.
Abbiamo quindi:
$$
I_m (A^T A) = \{ y \in \mathbb{C}^n \ : \ y = A^T A z, \quad z \in \mathbb{C}^n \}
$$
Il caso buono è $A^T b \in I_m(A)$, cioè:
$$
I_m (A) = \{ y \in \mathbb{C}^n \ : \ y = A z, \quad z \in \mathbb{C}^n \}
$$
così che:
$$
A^T b = A^T A z \in I_m (A)
$$
Questo però non è sempre vero.
Possiamo allora dire che uno spazio vettoriale è sempre dato dalla somma dello spazio delle colonne di una trasformazione $A$ e il corrispettivo spazio perpendicolare: 
$$
\mathbb{C}^m = I_m(A) \oplus I_m(A)^\perp
$$
con lo spazio perpendicolare allo spazio delle colonne definito come:
$$
I_m(A)^\perp = \{ y : y^H z = 0, \quad \forall z \in I_m(A) \}
$$
Potremo allora riscrivere $b$ come la somma delle componenti \textit{parallela} e perpendicolare:
$$
b \in \mathbb{C}^m \implies b = b_1 + b_2, \quad
	\begin{cases}
		b_1 \in I_m(A) \\ 
		b_2 \in I_m(A)^\perp \\ 
	\end{cases}
$$
Prendiamo allora $A^T b$ come:
$$
A^T b = A^T b_1 + A^T b_2
$$
Si ha che $A^T b_1 \in I_m (A^T A)$ da quanto avevamo detto prima, mentre riguardo ad $A^T b_2$ si può dire:
$$
A^T b_2 = \begin{pmatrix}
	a_1^T b_2 \\
	\vdots \\
	a_m^T b_2
\end{pmatrix} = 0
$$
perchè $b_2 \in I_m(A)^\perp$ e le colonne di $A$ stanno in $I_m(A)$, quindi la condizione di Rouché-Capelli è soddisfatta e la soluzione esiste.
\qed

\item 
Per l'unicità, vogliamo verificare che $\det(A^T A) \neq 0$ soddisfatta la proprietà $\mathrm{rank}(A) = n$.
Abbiamo allora che dal teorema di Binet-Cauchy \textit{generalizzato}:
$$
\det(A^T A) = \sum_{[j]} \det(A_{[j]}^T) \cdot \det(A_{[j]}) = \sum_{[j]} |\det(A_{[j]})|^2
$$
dove la sommatoria varia su tutte le possibili scelte di $n$ indici in $\{ 1, ..., m \}$.
Vogliamo allora mostrare che $\det(A^T A) \geq 0$, cioè che $\exists$ almeno una sottomatrice $A_{[j]}$ tale che $\det(A_{[j]}) \neq 0$.
Questa esisterà sempre se $\mathrm{rank}(A) = n$, e quindi anche l'ultima tesi del teorema è soddisfatta. \qed 
\end{enumerate}

Vediamo due esempi pratici per osservare il teorema in azione.
\begin{enumerate}
	\item Prendiamo il sistema:
\[
	\begin{cases}
		2x_1 - x_2 = 1 \\ 
		-x_1 + x_2 = 0 \\ 
		x_1 + 2 x_2 = 1
	\end{cases}
\]
con $m = 3$ equazioni in $n = 2$ incognite.
Avremo che le matrici $A$ e $b$, e l'incognita $x$, saranno:
$$
A = 
\begin{pmatrix}
	2 & -1 \\
	-1 & 1 \\ 
	1 & 2
\end{pmatrix}, \quad
b =
\begin{pmatrix}
	1 \\ 
	0 \\ 
	1
\end{pmatrix}, \quad
x = 
\begin{pmatrix}
	x_1 \\ 
	x_2
\end{pmatrix}
$$
per il sistema $Ax = b$.
Vediamo che il rango di $A$ è 2, quindi dal teorema appena dimostrato la soluzione sarà unica. 
Calcoliamo allora le matrici $A^T A$ e $A^t b$ per il sistema $A^T A x = A^T b$:
$$
A^T A = 
\begin{pmatrix}
	6 & -1 \\ 
	-1 & 6
\end{pmatrix}, \quad
A^T b =
\begin{pmatrix}
	3 \\ 
	1
\end{pmatrix}
$$
Avremo quindi il sistema:
\[
	\begin{cases}
		6 x_1 - x_2 = 3 \\
		-x_1 + 6 x_2 = 1
	\end{cases}
\]
da cui:
$$
x^* =
\begin{pmatrix}
	x_1 \\ x_2
\end{pmatrix}
=
\begin{pmatrix}
	\frac{19}{35} \\ \frac{9}{35}
\end{pmatrix}
$$
Che è la soluzione ottima.
Possiamo quindi calcolare la norma di residuo:
$$
r = b - A x^* = b - A \cdot
\begin{pmatrix}
	\frac{19}{35} \\ 
	\frac{9}{35}
\end{pmatrix}
=
\begin{pmatrix}
	\frac{6}{35} \\ 
	\frac{10}{35} \\
	-\frac{2}{35}
\end{pmatrix}
$$
da cui ottieniamo uno scarto di $\approx 0.3381$ dal valore ideale.
\item Un problema simmetrico al precedente potrebbe essere quello di trovare per quali parametri una matrice parametrizzata rispetta il teorema, cioè ammette una soluzione unica.
	Prendiamo ad esempio il sistema:
$$
\underbrace{
\begin{pmatrix}
	1 & 1 & 1 \\ 
	1 & \alpha & 1 \\
	0 & 0 & \alpha \\ 
	\beta & 0 & 0
\end{pmatrix}
}_A
\begin{pmatrix}
	x_1 \\ 
	x_2 \\
	x_3
\end{pmatrix}
=
\underbrace{
\begin{pmatrix}
	1 \\ 
	-1 \\
	1 \\
	-1
\end{pmatrix}
}_b
$$
con $\alpha, \beta \in \mathbb{R}$.
Iniziamo imponendo l'antitesi del teorema cioè:
$$
\mathrm{rank}(A) < n \ \Leftrightarrow \ \exists A_{[j]} \in \mathbb{R}^{3 \times 3} \text{ non invertibile}
$$
con $A_{[j]}$ le sottomatrici quadrate $3 \times 3$ prese dal teorema di Binet-Cauchy.
Consideriamo allora tutte le possibili $A_{[j]}$ coi loro determinanti:

\noindent
\begin{minipage}{0.22\textwidth}
$$
\begin{array}{c}
\begin{pmatrix}
	1 & 1 & 1 \\
	1 & \alpha & 1 \\
	0 & 0 & 1
\end{pmatrix} \\ 
\downarrow \\
\alpha(\alpha - 1)
\end{array}
$$
\end{minipage}
\begin{minipage}{0.22\textwidth}
$$
\begin{array}{c}
\begin{pmatrix}
	1 & 1 & 1 \\
	1 & \alpha & 1 \\
	\beta & 0 & 0
\end{pmatrix} \\ 
\downarrow \\
\beta(1 - \alpha)
\end{array}
$$
\end{minipage}
\begin{minipage}{0.22\textwidth}
$$
\begin{array}{c}
\begin{pmatrix}
	1 & 1 & 1 \\
	0 & 0 & \alpha \\
	\beta & 0 & 0
\end{pmatrix} \\ 
\downarrow \\
\alpha \beta
\end{array}
$$
\end{minipage}
\begin{minipage}{0.22\textwidth}
$$
\begin{array}{c}
\begin{pmatrix}
	1 & \alpha & 1 \\
	0 & 0 & \alpha \\
	\beta & 0 & 0
\end{pmatrix} \\ 
\downarrow \\
\alpha^2 \beta
\end{array}
$$
\end{minipage}
\par\bigskip
da cui avere tutti determinanti nulli singifica soddisfare il sistema (non lineare):
\[
	\begin{cases}
		\alpha(\alpha - 1) = 0 \\
		\beta(1 - \alpha) = 0 \\ 
		\alpha \beta = 0 \\
		\alpha^2 \beta = 0
	\end{cases}
\]
con soluzioni $(\alpha, \beta) = (0, 0)$ o $(\alpha, \beta) = (1, 0)$.
Si ha quindi che per queste due combinazioni di valori di $\alpha$ e $\beta$ il sistema di partenza non ammette soluzione ottima unica.
\end{enumerate}

\subsubsection{Considerazioni di complessità del sistema delle equazioni normali}
Valutiamo la complessità del sistema visto finora.
Dobbiamo:
\begin{enumerate}
	\item Calcolare $A^H A$, che ha complessità $O(mn^2)$ (da una $(n \times m) (m \times n)$);
	\item Calcolare $A^H b$, che ha complessità $O(mn)$ (da una $(n \times m)(m \times 1)$);
	\item Risolvere il sistema $A^H A x = A^H b$, che ha complessità $O(mn^2 + n^3) \sim O(n^3)$.
\end{enumerate}
Vediamo che il problema non è tanto la complessità, ma il fatto che $A^T A$ ha spesso un numero di condizionamento piuttosto alto, ad esempio nel caso $m = n$ si ha:
$$
\mu(A^T A) = \mu(A)^2
$$
Se $\mu(A^T A)$ ($< 10^k$ con $k$ moderato), questo è fastidioso in quanto abbatte sostanzialmente la precisione.
Conviene quindi applicare un metodo alternativo.

\subsection{Metodo QR per problemi ai minimi quadrati}
Vediamo quindi un metodo basato sulla fattorizzazione QR.

\subsubsection{Fattorizzazione QR}
La \textbf{fattorizzazione QR} è un tipo di fattorizzazione matriciale, come lo era la \textit{fattorizzazione LU} che abbiamo già visto (8.3.4).

Diamone quindi la definizione:
\begin{definition}{Fattorizzazione LU}
	Data $A \in \mathbb{C}^{m \times n}$, con $m \geq n$, una fattorizzazione QR di $A$ è una coppia di matrici $(Q, R)$ tali che: $A = Q R$, con $Q \in \mathbb{C}^{m \times m}$ unitaria ($Q^H Q = Q Q^H = I$), e $R \in \mathbb{C}^{m \times m}$ della forma triangolare superiore rettangolare (con un certo numero di righe a zero sotto il triangolo), cioè esprimibile come:
	$$
R = \begin{pmatrix}
R_1 \\ 0
\end{pmatrix}
	$$
	con $R_1 \in \mathbb{C}^{n \times n}$ triangolare superiore.
\end{definition}

Varrà allora il teorema:
\begin{theorem}{Esistenza della fattorizzazione QR}
	Per ogni matrice $\exists$ sempre una fattorizzazione QR, ma non è mai unica.
\end{theorem}

Questo viene dal fatto che data $A = QR$, si potrebbe prendere $D \in \mathbb{R}^{m \times m}$ diagonale con $|d_j| = 1$ per $j = 1, ..., m$, e dire:
$$
A = Q_2 R_2, \quad
	\begin{cases}
		Q_2 = QD \\
		R_2 = D^{-1} R
	\end{cases}
$$
in quanto:
$$
A = QR =  \underbrace{QD}_{Q_2} \cdot \underbrace{D^{-1}R}_{R_2}
$$

\par\smallskip

Una proprietà di rilievo è che la norma 2 e la norma di Frobenius sono invarianti per moltiplicazioni con matrici unitarie, ovvero:
$$
|A| = |QA| = |AQ| = |Q_1 A Q_2|
$$
con norma $|\cdot |$ 2 o di Frobenius.

Dimostriamo per le due norme:
\begin{itemize}
	\item \textbf{Norma 2:} 
questo si ha prendendo la definizione di norma 2:
$$
|A|_2 = \sqrt{\rho(A^H A)}
$$
Allora la norma di $AQ$, ad esempio, sarà:
$$
|AQ|_2 = \sqrt{\rho( (AQ)^H AQ )} = \sqrt{\rho(Q^H A^H A Q)}
$$
$Q^H A^H A Q$ è simile a $A^H A$ in quanto dalla definizione di unarietà di $Q$, si può intendere $Q^H = Q^{-1}$ e quindi prendere la forma come un applicazione di matrice di trasformazione, da cui la tesi per la norma 2. \qed

	\item \textbf{Norma di Frobenius:}
questo si ha prendendo la definizione di norma di Frobenius:
$$
|A|_F = \sqrt{\mathrm{trace}(A^H A)}
$$
Allora la norma di $AQ$, ad esempio, sarà:
$$
|AQ|_F = \sqrt{\mathrm{trace}(Q^H A^H A Q)} 
$$
da cui come sopra. \qed
\end{itemize}

\par\smallskip

Nei problemi ai minimi quadrati, la fattorizzazione QR è utile in quanto se $A = QR$:
$$
\min_{x \in \mathbb{C}^n} | b - Ax |_2 = \min_{x \in \mathbb{C}^n} |b - QR x|_2 = \min_{x \in \mathbb{C}} |Q^H b - Rx|_2
$$
moltiplicando per $Q^H$ all'ultimo passaggio (che preserva la norma, come abbiamo appena dimostrato).

Chiamato $Q^H b = c$, osserviamo quindi di poter prendere:
$$
\psi(x) = |cx - Rx|_2^2
$$
che preso:
	$$
R = \begin{pmatrix}
R_1 \\ 0
\end{pmatrix}
	$$
diventa:
$$
\psi(x) = \left| \begin{pmatrix}
	c_1 \\ c_2
\end{pmatrix} - \begin{pmatrix}
	R_1 \\ 0
\end{pmatrix} x \right|^2_2
=
\left|
\begin{pmatrix}
	c_1 - R_1 x \\ 
	c_2
\end{pmatrix}
\right|_2^2
=
\left|
	c_1 - R_1 x
\right|_2^2
+
|c_2|_2^2
$$
l'unico termine su cui abbiamo controllo è il primo, e possiamo quindi trascurare $|c_2|_2^2$.
Prendiamo allora:
$$
\psi'(x) = |c_1 - R_1 x|_2^2
$$
che rappresenta un sistema lineare $n \times n$, dove bastera scegliere $x$ tale che:
$$
R_1 x = c
$$
così da trovare 0 e ottenere il valore di $\psi'$ (e quindi di $\psi$) minore possibile.
Il vantaggio aggiunto è che $R_1$ è triangolare superiore, e quindi si può calcolare:
$$
x = R_1^{-1} c
$$
per sostituzione all'indietro.

Riassumendo, quindi, abbiamo ottenuto il metodo:
\begin{enumerate}
	\item Si calcola la fattorizzazione $A = QR$;
	\item Si calcola $Q^H b = \begin{pmatrix}
		c_1 \\ c_2
	\end{pmatrix}$, che ha costo $O(nm)$ (non è $O(m^2 n)$, per via della struttura di $Q$);
\item Si risolve $R_1 x = c_1$, che è un sistema lineare triangolare superiore $n \times n$, quindi ha costo $O(n^2)$ per sostituzione all'indietro.
\end{enumerate}

L'unica incognita è allora il costo della fattorizzazione, che anticipiamo è $O(mn^2)$.

\subsubsection{Matrici di Householder}
Una proprietà rilevante è che dato un vettore $v \in \mathbb{C}^m$ esiste sempre una matrice unitaria $H_v$ tale che:
$$
H_v \cdot v = c \cdot e_1, \quad c \in \mathbb{C}
$$
per $e_1$ primo elemento della base canonica.
Dato che la moltiplicazione per matrici unitarie non cambia la norma si ha:
$$
|c| = |v|_2
$$
Per creare tale matrice, scegliamo $\tilde{v}$:
$$
\tilde{v} = v \pm |v|_2 \cdot e_1 =
\begin{pmatrix}
	v_1 \pm |v|_2 \\ 
	v_2 \\ 
	\vdots \\ 
	v_m
\end{pmatrix}
$$
$H_v$ dovrà quindi essere scelta come:
$$
H_v = I - 2  \frac{\tilde{v} \tilde{v}^H}{|\tilde{v}|_2^2}
$$
dove il termine $\frac{\tilde{v} \tilde{v}^H}{|\tilde{v}|_2^2}$ rappresenta una proiezione sulla retta tracciata dal vettore $\tilde{v}$, e quindi $H_v$ uguale alla \textit{matrice di riflessione} sull'iperpiano ortogonale a $\tilde{v}$.   
Visto che abbiamo scelto $\tilde{v}$ perchè fosse effettivamente \textit{"equidistante"} sia da $v$ che da $e_1$, otteniamo una matrice che porta $v$ esattamente sulla base $e_1$.

Le matrici costruite in questo modo vengono anche dette \textbf{Matrici di Householder} associate al vettore $v$.

Possiamo scrivere una funzione MATLAB, che ci torneà utile a breve, per il calcolo della matrice di Householder relativa a un certo vettore:
\lstset{language=MATLAB,style=codestyle}
\lstinputlisting{../code/matlab/householder.m}

Vediamo inoltre come si è scelto il segno della norma aggiunta a \lstinline|vt| ($\tilde{v}$) in modo da ridurre gli errori di cancellazione.

\subsubsection{Calcolo della fattorizzazione QR}
Analogamente al metodo di Gauss, per la fattorizzazione QR si moltiplica a sinistra per matrici unitarie che \textit{"sistemano"} gli elementi sotto la diagonale.

Sfruttando la proprietà delle matrici di Householder, agiamo come segue:
$$
H_1 = H_{a_1} = \text{matrice di Householder associata ad $a_1$, prima colonna di $A$}
$$
diciamo:
$$
H_1 \begin{pmatrix}
	a_1 & a_2 & ... & a_n
\end{pmatrix} =
\begin{pmatrix}
	? & ... \\
	0 & ... \\ 
	\vdots & ... \\ 
	0 & ...
\end{pmatrix}
$$
a questo punto basterà prendere $H_2$ per \textit{"sistemare"} gli elementi della sottomatrice di coda successiva:
$$
H_2 = \left( \begin{array}{c| c c c}
	1 & 0 & ... & 0 \\
	\hline
	0 & & & \\
	\vdots & & \tilde{H}_2 & \\
	0 & & & \\
\end{array} \right)
$$
e così via, introducendo di volta in volta più elementi della diagonale in testa.
Si potrà quindi definire la matrice $R$ come:
$$
R = \underbrace{H_1^H H_2^H ... H_n^H}_Q A
$$
dove il prodotto delle unitarie rappresenterà la matrice $Q$.

\subsubsection{Ottimizzazioni della fattorizzazione QR}
Osserviamo che per avere costo $O(mn^2)$ (ed anche costo $O(mn)$ per $Q^H b$) si sfrutta il fatto che $H_j$ è del tipo \textit{identità + rango 1}, cioè:
$$
H_j = I - b b^T
$$
dove gli $a$ e $b$ vengono presi qui a scopo di esempio, risulterebbero dalla forma della matrice di Householder cha abbiamo visto alla 12.2.2.
La parte importante è che il prodotto per una matrice generica $B$ assume la forma:
$$
H_j \cdot B = B - b \cdot b^T \cdot B
$$
dove l'ultimo termine è il prodotto di una colonna per il prodotto di una riga per $B$.
Quindi, nel caso $B$ sia un vettore $b$, si hanno solo prodotti scalari di costo $m$, ergo:
$$
H_j b \rightarrow O(m)
$$
Se prendiamo il prodotto per tutte le matrici $H_j$, cioè per $Q$, abbiamo che questa cosa si ripete $n$ volte e quindi il costo è $O(mn)$.
$$
\implies H_n \cdot H_{n - 1} \cdot ... \cdot H_1 b \rightarrow O(mn)
$$
il costo $O(mn^2)$ a questo punto è immediato dal fatto che abbiamo semplicemente una dimensione $n$ in più dalla matrice $A$.

\subsubsection{Osservazioni ulteriori sulla fattorizzazione QR}
Osserviamo poi che nel caso $m = n$ si può sfruttare la fattorizzazione QR per risolvere sistemi lineari, con costo $O(\frac{4}{3}n^3)$ per la fattorizzazione, il doppio rispetto a Gauss, ma solitamente più stabile.

Osserviamo infine che se $A \in \mathbb{R}^{m \times n}$ si può restare nei reali ed avere $Q \in \mathbb{R}^{m \times m}$, $Q^T Q = I$, ecc...

\subsubsection{Implementazione MATLAB della decomposizione QR}
Riportiamo l'implementazione MATLAB di questo algoritmo:
\lstinputlisting{../code/matlab/qr_decomp.m}
Dove la funzione \lstinline|blkdiag| ci permette di realizzare la forma a blocchi diagonale, con l'identità in testa, che la $H_j$ assume ad ogni passaggio dell'algoritmo.

\subsubsection{Implementazione MATLAB ottimizzata della decomposizione QR}
Osserviamo quindi la versione della fattorizzazione QR, implementata in MATLAB, che effettua le ottimizzazioni riportate nella sezione 12.2.4:
\lstinputlisting{../code/matlab/opt_qr_decomp.m}
dove la matrice di Householder viene mantenuta in due vettori, e si sono definite le funzioni \lstinline|rank_one_l()| e \lstinline|rank_one_r()| per l'aggiornamento di rango uno, rispettivamente a sinistra e a destra (che nient'altro sarebbe se non la moltiplicazione per la matrice di Householder).

\subsubsection{Implementazione C++ della decomposizione QR}
L'ultima versione ottimizzata della decomposizione QR si presta particolarmente bene all'implementazione in linguaggi di livello più basso. 

Potremmo infatti avere la funzione, del tutto analoga alla scorsa implementazione MATLAB (se non per il fatto che si assume la matrice $A$ sia quadrata):
\begin{lstlisting}[language=C++, style=codestyle]	
void qr_decomp(double** A, double** Q, unsigned n) {
	// inizializza vettori a, b
	double* a, *b;
	a = new double[n];
	b = new double[n];

	for(unsigned i = 0; i < n; i++) {
		std::cout << "Passaggio " << i << std::endl;

		householder_vec(A, i, n, a, b);

		std::cout << "I vettori di Householder sono:" << std::endl;
		std::cout << "a:" << std::endl;
		print_vector(a, n - i);
		std::cout << "b:" << std::endl;
		print_vector(b, n - i);

		rank_one_l(A, i, n, a, b);
		rank_one_r(Q, i, n, a, b);

		std::cout << std::endl;
	}

	// ripulisci
	delete[] a;
	delete[] b;
}
\end{lstlisting}
dove la \lstinline|householder_vec()| è anch'essa simile alla versione MATLAB:
\begin{lstlisting}[language=C++, style=codestyle]	
void householder_vec(double** A, unsigned i, unsigned n, double* a, double* b) {
	double sqr_norm = 0;

	// prendi la i-esima colonna di A
	for(unsigned j = i; j < n; j++) {
		a[j - i] = A[j][i];
		sqr_norm += a[j - i] * a[j - i];
	}

	double norm = std::sqrt(sqr_norm);
	sqr_norm -= a[0] * a[0];

	if(a[0] < 0) {
		a[0] -= norm;
	} else {
		a[0] += norm;
	}

	sqr_norm += a[0] * a[0];

	for(unsigned j = 0; j < n - i; j++) {
		b[j] = 2 * a[j] / sqr_norm;
	}
}
\end{lstlisting}
se non per una probabile ottimizzazione che si potrebbe fare riguardo alle radici, che però ignoriamo.

A questo punto si implementano le \lstinline|rank_one_l()| e \lstinline|rank_one_r()| srotolando le moltiplicazioni, ed evidenziando quindi il costo delle applicazioni delle matrici di rango uno, che è effettivamente $O(mn^2)$:
\begin{lstlisting}[language=C++, style=codestyle]	
// applicazione a sinistra
void rank_one_l(double** A, unsigned i, unsigned n, double* a, double* b) {
	// A(i:end, :) = A(i:end, :) - a * b' * A(i:end, :);
	for(unsigned c = 0; c < n; c++) {
		double sum = 0;

		for(unsigned r = i; r < n; r++) {
			sum += b[r - i] * A[r][c]; 
		}

		for(unsigned r = i; r < n; r++) {
			A[r][c] -= a[r - i] * sum;
		}
	}
}

// applicazione a destra
void rank_one_r(double** A, unsigned i, unsigned n, double* a, double* b) {
	// A(i:end, :) = A(i:end, :) - (A(i:end, :) * a) * b';
	for(unsigned r = 0; r < n; r++) {
		double sum = 0;

		for(unsigned c = i; c < n; c++) {
			sum += A[r][c] * a[c - i]; 
		}

		for(unsigned c = i; c < n; c++) {
			A[r][c] -= sum * b[c - i];
		}
	}
}
\end{lstlisting}

L'algoritmo è dunque completo, e realizza usando solo semplici cicli di complessità al massimo quadratica/cubica la decomposizione QR.
Ricordiamo che il codice visto finora è disponibile nella directory \lstinline|/code| degli appunti del corso, e in particolare queste implementazioni di livello più basso in C++ sono disponibili in \lstinline|/code/cpp|.

\subsubsection{Implementazione MATLAB di un risolutore ai minimi quadrati}
Abbiamo quindi tutti gli elementi necessari per realizzare un semplice risolutore che implementi ciò che MATLAB fa già naturalmente quando gli si pone l'istruzione \lstinline|A \ b| con il numero di righe di $A$ maggiore del numero di colonne, cioè risolvere un sistema sovradeterminato usando i minimi quadrati.

Prendiamo quindi la serie di passaggi descritta in 12.2.1 ed implementiamo un algoritmo che li effettui:
\lstinputlisting{../code/matlab/least_squares.m}
dove notiamo che dobbiamo troncare sia la matrice $R$ che il vettore $c$ ai soli elementi non nulli (cosa fatta controllando gli elementi non nulli della diagonale), in quanto la \lstinline|bck_subst()| non si comporta bene altrimenti.

Potremo quindi usare l'algoritmo per risolvere problemi ai minimi quadrati come segue:
\begin{lstlisting}[language=matlab, style=codestyle]	
% dati A, b noti
x = least_squares(A, b);
% x minimizza lo scarto quadratico
\end{lstlisting}

\end{document}
