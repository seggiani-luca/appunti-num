
\documentclass[a4paper,11pt]{article}
\usepackage[a4paper, margin=8em]{geometry}

% usa i pacchetti per la scrittura in italiano
\usepackage[french,italian]{babel}
\usepackage[T1]{fontenc}
\usepackage[utf8]{inputenc}
\frenchspacing 

% usa i pacchetti per la formattazione matematica
\usepackage{amsmath, amssymb, amsthm, amsfonts}

% usa altri pacchetti
\usepackage{gensymb}
\usepackage{hyperref}
\usepackage{standalone}

% imposta il titolo
\title{Appunti Calcolo Numerico}
\author{Luca Seggiani}
\date{2025}

% disegni
\usepackage{pgfplots}
\pgfplotsset{width=10cm,compat=1.9}

% imposta lo stile
% usa helvetica
\usepackage[scaled]{helvet}
% usa palatino
\usepackage{palatino}
% usa un font monospazio guardabile
\usepackage{lmodern}

% tikz in sans
\tikzset{every picture/.style={/utils/exec={\sffamily}}}

\renewcommand{\rmdefault}{ppl}
\renewcommand{\sfdefault}{phv}
\renewcommand{\ttdefault}{lmtt}

% circuiti
\usepackage{circuitikz}
\usetikzlibrary{babel}

% disponi il titolo
\makeatletter
\renewcommand{\maketitle} {
	\begin{center} 
		\begin{minipage}[t]{.8\textwidth}
			\textsf{\huge\bfseries \@title} 
		\end{minipage}%
		\begin{minipage}[t]{.2\textwidth}
			\raggedleft \vspace{-1.65em}
			\textsf{\small \@author} \vfill
			\textsf{\small \@date}
		\end{minipage}
		\par
	\end{center}

	\thispagestyle{empty}
	\pagestyle{fancy}
}
\makeatother

% disponi teoremi
\usepackage{tcolorbox}
\newtcolorbox[auto counter, number within=section]{theorem}[2][]{%
	colback=blue!10, 
	colframe=blue!40!black, 
	sharp corners=northwest,
	fonttitle=\sffamily\bfseries, 
	title=Teorema~\thetcbcounter: #2, 
	#1
}

% disponi definizioni
\newtcolorbox[auto counter, number within=section]{definition}[2][]{%
	colback=red!10,
	colframe=red!40!black,
	sharp corners=northwest,
	fonttitle=\sffamily\bfseries,
	title=Definizione~\thetcbcounter: #2,
	#1
}

% disponi problemi
\newtcolorbox[auto counter, number within=section]{problem}[2][]{%
	colback=green!10,
	colframe=green!40!black,
	sharp corners=northwest,
	fonttitle=\sffamily\bfseries,
	title=Problema~\thetcbcounter: #2,
	#1
}

% disponi codice
\usepackage{listings}
\usepackage[table]{xcolor}

\lstdefinestyle{codestyle}{
		backgroundcolor=\color{black!5}, 
		commentstyle=\color{codegreen},
		keywordstyle=\bfseries\color{magenta},
		numberstyle=\sffamily\tiny\color{black!60},
		stringstyle=\color{green!50!black},
		basicstyle=\ttfamily\footnotesize,
		breakatwhitespace=false,         
		breaklines=true,                 
		captionpos=b,                    
		keepspaces=true,                 
		numbers=left,                    
		numbersep=5pt,                  
		showspaces=false,                
		showstringspaces=false,
		showtabs=false,                  
		tabsize=2
}

\lstdefinestyle{shellstyle}{
		backgroundcolor=\color{black!5}, 
		basicstyle=\ttfamily\footnotesize\color{black}, 
		commentstyle=\color{black}, 
		keywordstyle=\color{black},
		numberstyle=\color{black!5},
		stringstyle=\color{black}, 
		showspaces=false,
		showstringspaces=false, 
		showtabs=false, 
		tabsize=2, 
		numbers=none, 
		breaklines=true
}

\lstdefinelanguage{javascript}{
	keywords={typeof, new, true, false, catch, function, return, null, catch, switch, var, if, in, while, do, else, case, break},
	keywordstyle=\color{blue}\bfseries,
	ndkeywords={class, export, boolean, throw, implements, import, this},
	ndkeywordstyle=\color{darkgray}\bfseries,
	identifierstyle=\color{black},
	sensitive=false,
	comment=[l]{//},
	morecomment=[s]{/*}{*/},
	commentstyle=\color{purple}\ttfamily,
	stringstyle=\color{red}\ttfamily,
	morestring=[b]',
	morestring=[b]"
}

% disponi sezioni
\usepackage{titlesec}

\titleformat{\section}
	{\sffamily\Large\bfseries} 
	{\thesection}{1em}{} 
\titleformat{\subsection}
	{\sffamily\large\bfseries}   
	{\thesubsection}{1em}{} 
\titleformat{\subsubsection}
	{\sffamily\normalsize\bfseries} 
	{\thesubsubsection}{1em}{}

% disponi alberi
\usepackage{forest}

\forestset{
	rectstyle/.style={
		for tree={rectangle,draw,font=\large\sffamily}
	},
	roundstyle/.style={
		for tree={circle,draw,font=\large}
	}
}

% disponi algoritmi
\usepackage{algorithm}
\usepackage{algorithmic}
\makeatletter
\renewcommand{\ALG@name}{Algoritmo}
\makeatother

% disponi numeri di pagina
\usepackage{fancyhdr}
\fancyhf{} 
\fancyfoot[L]{\sffamily{\thepage}}

\makeatletter
\fancyhead[L]{\raisebox{1ex}[0pt][0pt]{\sffamily{\@title \ \@date}}} 
\fancyhead[R]{\raisebox{1ex}[0pt][0pt]{\sffamily{\@author}}}
\makeatother

\begin{document}

% sezione (data)
\section{Lezione del 17-03-25}

% stili pagina
\thispagestyle{empty}
\pagestyle{fancy}

% testo
Avevamo visto i primi 2 teoremi di Gershgorin, cioè:
\begin{itemize}
	\item \textbf{Primo teorema di Gershgorin:} ogni autovalore $\lambda$ è contenuto in:
		$$
			\lambda \in \bigcup_{i=1}^n \mathcal{F}_i(A)
		$$
		definiti i \textit{cerchi di Gershgorin}:
		$$
			\mathcal{F}_i(A) = \left\{ z \in \mathbb{C}, \quad |z - a_{ii}| < \sum_{j \neq i} a_{ij} \right\}
		$$
		
	\item \textbf{Secondo teorema di Gershgorin:} un $M_1$ è unione di $s$ cerchi e $M_2$ unione dei restanti $n - s$, allora:
		$
			M_1 \cap M_2 = \emptyset \implies
		$
		$M_1$ contiene $s$ autovalori e $M_2$ i restanti $n - s$.
\end{itemize}

Vediamo quindi il terzo teorema di Gershgorin.
Per fare ciò, introduciamo il concetto di \textbf{matrice irriducibile}, e ancor prima di \textbf{grafo} associato alla matrice.

\subsection{Matrici irriducibili}
Diamo la definizione:
\begin{definition}{Grafo associato ad una matrice}
	Data una matrice $A \in \mathbb{C}^{n \times n}$ si dice grafo associato ad $A$, indicato con $G(A) = (V, E)$, il grafo che ha come vertici l'insieme $V = \{ 1, ..., n \}$ e come archi l'insieme $E$ così definito:
	$$
		(i, j) \in E \ \Leftrightarrow \ a_{ij} \neq 0
	$$
\end{definition}

Solitamente si ignora la diagonale (notiamo che in caso di entrate $\neq 0$ risultano semplicemente in anelli aperti e chiusi sulllo stesso nodo).

A noi interesseranno grafi \textbf{fortemente connessi}:
\begin{definition}{Grafo fortemente connesso}
	Un grafo si dice fortemente connesso se per ogni scelta di vertici $i, j \in 1, ..., n$ esiste un cammino \textit{orientato} sul grafo che parte da $i$ e arriva a $j$.
\end{definition}

A questo punto possiamo definire le matrici \textbf{irriducibili}:
\begin{definition}{Matrice irriducibile}
	Una matrice $A \in \mathbb{C}^{n \times n}$ si dice irriducibile se il grafo $G(A)$ è fortemente connesso.
\end{definition}

Possiamo quindi enunciare il terzo teorema di Gershgorin:
\begin{theorem}{Terzo teorema di Gershgorin}
	Data una matrice $A \in \mathbb{C}^{n \times n}$ irriducibile si ha che, se $\lambda$ è un autovalore tale che:
	$$
		\lambda \in \partial  \mathcal{F}_i(A) 
	$$
	dove con $\partial$ si indica il bordo, cioè esiste un indice, $\exists k \in \{ 1, ..., n \}$ tale che:
	$
	|\lambda - a_{kk}| = \sum_{j \neq k} a_{kj}
	$,
	allora dovrà essere che:
	$$
		\lambda \in \bigcap_{i = 1}^n \partial \left( \mathcal{F}_i(A) \right)
	$$
	cioè per \textit{tutti} gli indici, $\forall k \in \{ 1, ..., n \}$ vale che:
	$
	|\lambda - a_{kk}| = \sum_{j \neq k} a_{kj}
	$.

\end{theorem}

Questo significa che, nel caso di matrici irriducibili, un autovalore che sta sul bordo di un disco di Gershgorin dovrà necessariamente stare sul bordo di \textit{tutti} i dischi di Gershgorin.

Avevamo visto la definizione di matrici a predominanza diagonale forte.
Esistono anche matrici a predominanza diagonale \textit{debole} (o semplicemente \textit{diagonali deboli}):
\begin{definition}{Matrice a predominanza diagonale debole}
	Una matrice $A$ si dice a predominanza diagonale debole se vale:
	$$
	|a_{ii}| \geq \sum_{j \neq i} |a{ij}|
	$$
	ed $\exists k \in \{ 1, ..., n \}$ tale per cui:
	$$
	|a_{kk}| > \sum_{j \neq k} |a{kj}|
	$$
\end{definition}

Un \textbf{corollario} è che se $A$ è a predominanza diagonale debole ed è irriducibile, allora $A$ non è singolare.

Questo si dimostra direttamente dal fatto che il valore $0$ appartiene ai cerchi per cui vale $|a_{ii}| \geq \sum_{j \neq i} |a_{ij}$, ma esiste almeno un cerchio per cui vale $|a_{kk}| > \sum_{j \neq k} |a{kj}|$, cioè non contiene lo $0$, e quindi per il terzo teorema di Gershgorin $0$ non può essere autovalore (matrice non singolare). \qed

\subsection{Matrici a blocchi}
Talvolta è conveniente definire una matrice in termini di \textit{sottomatrici}, o \textbf{blocchi}, al posto delle entrate scalari che la compongono, cioè si può scrivere:
$$
A \in \mathbb{C}^{m \times n}, \quad A = \begin{pmatrix}
	A_{11} & ... & A_{1r} \\
	... & & ... \\
	A_{s1} & ... & A_{sr} \\
\end{pmatrix}
$$
Si può sempre scrivere una matrice a blocchi, purché le sottomatrici abbiano dimensioni compatibili (sia le somme delle dimensioni devono corrispondere alle dimensioni effettive della matrice, sia le dimensioni delle matrici adiacenti devono corrispondere), cioè si deve avere:
$$
A_{ij} \in \mathbb{C}^{m_i \times n_j}, \quad \sum_{i=1}^s m_i = m, \quad \sum_{j=1}^r n_j = n
$$

Notiamo che per noi sarà il importante il caso di matrici quadrate poste come sottomatrici quadrate, cioè $m = n$, $s = r$ e $m_i = n_i$ $\forall i = 1, ..., r \cdot s$.

Vediamo quindi la definizione di matrici \textbf{triangolari a blocchi}:
\begin{definition}{Matrice triangolare a blocchi}
	Nelle condizioni di cui sopra (in particolare, matrici quadrate) una matrice si dice triangolare a blocchi se sono $\neq 0$ tutti i blocchi lungo la diagonale, e sopra o sotto la diagonale (dove rispettivamente si parla di matrice triangolare superiore o inferiore), cioè:
$$
A \in \mathbb{C}^{m \times n}, \quad A = \begin{pmatrix}
	A_{11} & ... & A_{1r} \\
	0 & ... & ... \\
	0 & 0 & A_{sr} \\
\end{pmatrix} \text{ (superiore) o }
A = \begin{pmatrix}
	A_{11} & 0 & 0 \\
	... & ... & 0 \\
	A_{s1} & ... & A_{sr} \\
\end{pmatrix} \text{ (inferiore)}
$$
\end{definition}

Vediamo anche le matrici \textbf{diagonali a blocchi}:
\begin{definition}{Matrice diagonale a blocchi}
	Nelle condizoni della definizione 7.5, una matrice si dice diagonale a blocchi se sono $\neq 0$ i soli blocchi lungo la diagonale, cioè:
$$
A \in \mathbb{C}^{m \times n}, \quad A = \begin{pmatrix}
	A_{11} & 0 & 0 \\
	0 & ... & 0 \\
	0 & 0 & A_{sr} \\
\end{pmatrix} \text{ (diagonale)}
$$
\end{definition}

\subsubsection{Proprietà delle matrici a blocchi}
Vediamo alcune proprietà delle matrici a blocchi:
\begin{enumerate}
	\item Se $A$ è triangolare a blocchi, allora l'insieme degli autovalori di $A$ corrisponde all'unione degli insiemi di autovalori associati ai blocchi sulla diagonale $A_{jj}$;

	\item Dalla scorsa proprietà valgono, a cascata, tutte le proprietà che avevamo visto su autovalori e determinanti.
		Ad esempio il determinante sarà:
		$$
		\det(A) = \prod_{j=1}^s \det(A_{jj})
		$$
		ricordando che:
		$$
		\det(A_{jj}) = \prod_{i=1}^{\frac{n}{s}} \lambda_i
		$$
		con i $\lambda_i$ autovalori di $A_{jj}$;

	\item Se $A$ è diagonale, a blocchi, vale:
		$$
		A^{-1} = \begin{pmatrix}
			A_{11}^{-1} & 0 & 0 \\
	0 & ... & 0 \\
	0 & 0 & A_{sr}^{-1} \\
\end{pmatrix} 
		$$

	\item Se si hanno 2 matrici a blcchi con lo stesso partizionamento, allora calcolare $A + B$ e $A\cdot B$ si può fare trattando i sottoblocchi come entrate scalari, cioè date:
		$$
		 A = \begin{pmatrix}
		 	A_{11} & A_{12} \\
		 	A_{21} & A_{22} \\
		 \end{pmatrix}, \quad B = \begin{pmatrix}
		 	B_{11} & B_{12} \\
		 	B_{21} & B_{22} \\
		 \end{pmatrix}
		$$
		con $A_{11}$ corrispondente in dimensioni a $B_{11}$ e via dicendo, si ha che:
		$$
			A + B = \begin{pmatrix}
				A_{11} + B_{11} & A_{12} + B_{12} \\
				A_{21} + B_{21} & A_{22} + B_{22} \\
			\end{pmatrix}
		$$
		riguardo alla somma, e considerazioni simili (che non riportiamo) riguardo al prodotto.
\end{enumerate}

\subsection{Matrici riducibili}
Vediamo ora una definizione che è effettivamente \textit{duale} a quella di matrice irriducibile: quella di \textbf{matrice riducibile}:
\begin{definition}{Matrice riducibile}
	Una matrice $A \in \mathbb{C}^{n \times n}$, $n \geq 2$, si dice riducibile se $\exists \Pi$ matrice di permutazione tale che:
	$$
	\Pi A \Pi^T = \begin{pmatrix}
		A_{11} & A_{12} \\
		0 & A_{22}
	\end{pmatrix}
	$$
	a blocchi, con $A_{11} \in \mathbb{C}^{k \times k}$, $A_{22} \in \mathbb{C}^{(n - k) \times (n - k)}$, $k \in \{ 1, ..., n \}$.
\end{definition}

Osserviamo che ci possono essere più di una matrice di permutazione $\Pi$ che portano $A$ in una forma triangolare a blocchi, con diversi valori di $k$.

Osserviamo poi che la forma triangolare superiore non è imperativa, cioè se $\exists \Pi_1$ tale che: 
$$\Pi_1 A \Pi_1^T = \begin{pmatrix}
	A_{11} & A_{12} \\
	0 & A_{22}
\end{pmatrix}$$
triangolare superiore, allora vale anche che $\exists \Pi_2$ tale che: 
$$\Pi_2 A \Pi_2^T = \begin{pmatrix}
	A_{11} & 0 \\
	A_{21} & A_{22}
\end{pmatrix}$$
triangolare inferiore.
Vediamo anzi che in verità queste due forme sono equivalenti.

\subsubsection{Sistemi lineari con matrici riducibili}
Supponiamo di voler risolvere $Ax = b$ con $A \in \mathbb{C}^n \times n$, con $A$ riducibile, e conoscendo $T$.
In questo caso conviene procedere sfruttando:
$$
Ax = b \ \Leftrightarrow \ \Pi A x = \Pi b \ \Leftrightarrow \ \Pi A \Pi^T \Pi x = \Pi b \ \Leftrightarrow \ By = c
$$
dove si è detto $B = \Pi A \Pi^T$, $c = \Pi b$, e $y = \Pi x$.
$By = c$ sarà in forma:
$$
\begin{pmatrix}
	A_{11} & A_{12} \\
	0 & A_{22}
\end{pmatrix}
\begin{pmatrix}
	y_1 \\ y_2
\end{pmatrix} =
\begin{pmatrix}
	c_1 \\
	c_2
\end{pmatrix}
$$
con $c_1, y_1 \in \mathbb{C}^k$ e $c_2, y_2 \in \mathbb{C}^{n - k}$.

Una volta risolto $By =C$, si ritrova $x$ mediante la relazione:
$$
\Pi x = y \ \rightarrow \ x = \Pi^T y
$$

Vediamo come risolvere effettivamente questo sistema.
Scrivendo il sistema $2 \times 2$ a blocchi per esteso si ottiene:
\[
	\begin{cases}
		A_{11} y_1 + A_{12} y_2 = C_1 \\
		A_{22} y_2 = c_2
	\end{cases}
\]

L'ultima equazione rappresenta un sistema $(n - k) \times (n - k)$, che risolviamo per trovare $y_2$ (notiamo non c'è dipendenza da $y_1$, dalla definizione come triangolare superiore a blocchi della matrice $B$).
A questo punto si ottiene un altro sistema lineare:
$$
A_{11} y_1 = c_1 - A_{12} y_2
$$
con $y_2$ stavolta noto.

Il \textit{guadagno} di questo metodo è che risolvere un sistema $n \times n$ costa generalmente $O(n^3)$.
Effettuare il \textit{preprocessing} reso disponibile dalla riducibilità della matrice ci permette di rendere la complessità uguale a:
$$
O(k^3 + (n - k)^3 + k(n - k)) = O(k^3 + (n - k)^3)
$$
dove $k(n - k)$ alla prima equazione rappresenta il costo della moltiplicazione matrice-vettore $A_{11} y_{2}$ nella seconda equazione di risoluzione del sistema a blocchi.

Se si hanno molti $0$ da sfruttare, e quindi si può scegliere $k = \frac{n}{2}$, si ottiene un costo:
$$
O\left(\left(\frac{n}{2}\right)^3 + \left( n - \frac{n}{2}\right)^3\right) = O\left(2 \left( \frac{n}{2} \right)^3\right) = O\left( \frac{n^3}{4} \right)
$$
cioè si guadagna di un fattore di 4 in termini di tempo di esecuzione.

Osserviamo inoltre che se i blocchi $A_{11}$, $A_{22}$ sono a loro volta matrici riducibili, il processo si può applicare \textbf{ricorsivamente} sulle matrici della diagonale, in modo da ridurre ulteriormente la complessità del sistema (a patto di dover effettuare più passi di preprocessing).

\end{document}
