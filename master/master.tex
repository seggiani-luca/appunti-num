
\documentclass[a4paper,11pt]{article}
\usepackage[a4paper, margin=8em]{geometry}

% usa i pacchetti per la scrittura in italiano
\usepackage[french,italian]{babel}
\usepackage[T1]{fontenc}
\usepackage[utf8]{inputenc}
\frenchspacing 

% usa i pacchetti per la formattazione matematica
\usepackage{amsmath, amssymb, amsthm, amsfonts}

% usa altri pacchetti
\usepackage{gensymb}
\usepackage{hyperref}
\usepackage{standalone}

% cose fluttuanti
\usepackage{float}

% imposta il titolo
\title{Appunti Calcolo Numerico}
\author{Luca Seggiani}
\date{2025}

% disegni
\usepackage{pgfplots}
\pgfplotsset{width=10cm,compat=1.9}

% imposta lo stile
% usa helvetica
\usepackage[scaled]{helvet}
% usa palatino
\usepackage{palatino}
% usa un font monospazio guardabile
\usepackage{lmodern}

\renewcommand{\rmdefault}{ppl}
\renewcommand{\sfdefault}{phv}
\renewcommand{\ttdefault}{lmtt}

% disponi il titolo
\makeatletter
\renewcommand{\maketitle} {
	\begin{center} 
		\begin{minipage}[t]{.8\textwidth}
			\textsf{\huge\bfseries \@title} 
		\end{minipage}%
		\begin{minipage}[t]{.2\textwidth}
			\raggedleft \vspace{-1.65em}
			\textsf{\small \@author} \vfill
			\textsf{\small \@date}
		\end{minipage}
		\par
	\end{center}

	\thispagestyle{empty}
	\pagestyle{fancy}
}
\makeatother

% disponi teoremi
\usepackage{tcolorbox}
\newtcolorbox[auto counter, number within=section]{theorem}[2][]{%
	colback=blue!10, 
	colframe=blue!40!black, 
	sharp corners=northwest,
	fonttitle=\sffamily\bfseries, 
	title=Teorema~\thetcbcounter: #2, 
	#1
}

% disponi definizioni
\newtcolorbox[auto counter, number within=section]{definition}[2][]{%
	colback=red!10,
	colframe=red!40!black,
	sharp corners=northwest,
	fonttitle=\sffamily\bfseries,
	title=Definizione~\thetcbcounter: #2,
	#1
}

% disponi problemi
\newtcolorbox[auto counter, number within=section]{problem}[2][]{%
	colback=green!10,
	colframe=green!40!black,
	sharp corners=northwest,
	fonttitle=\sffamily\bfseries,
	title=Problema~\thetcbcounter: #2,
	#1
}

% disponi codice
\usepackage{listings}
\usepackage[table]{xcolor}

\lstdefinestyle{codestyle}{
		backgroundcolor=\color{black!5}, 
		commentstyle=\color{codegreen},
		keywordstyle=\bfseries\color{magenta},
		numberstyle=\sffamily\tiny\color{black!60},
		stringstyle=\color{green!50!black},
		basicstyle=\ttfamily\footnotesize,
		breakatwhitespace=false,         
		breaklines=true,                 
		captionpos=b,                    
		keepspaces=true,                 
		numbers=left,                    
		numbersep=5pt,                  
		showspaces=false,                
		showstringspaces=false,
		showtabs=false,                  
		tabsize=2
}

\lstdefinestyle{shellstyle}{
		backgroundcolor=\color{black!5}, 
		basicstyle=\ttfamily\footnotesize\color{black}, 
		commentstyle=\color{black}, 
		keywordstyle=\color{black},
		numberstyle=\color{black!5},
		stringstyle=\color{black}, 
		showspaces=false,
		showstringspaces=false, 
		showtabs=false, 
		tabsize=2, 
		numbers=none, 
		breaklines=true
}

\lstdefinelanguage{javascript}{
	keywords={typeof, new, true, false, catch, function, return, null, catch, switch, var, if, in, while, do, else, case, break},
	keywordstyle=\color{blue}\bfseries,
	ndkeywords={class, export, boolean, throw, implements, import, this},
	ndkeywordstyle=\color{darkgray}\bfseries,
	identifierstyle=\color{black},
	sensitive=false,
	comment=[l]{//},
	morecomment=[s]{/*}{*/},
	commentstyle=\color{purple}\ttfamily,
	stringstyle=\color{red}\ttfamily,
	morestring=[b]',
	morestring=[b]"
}

% disponi sezioni
\usepackage{titlesec}

\titleformat{\section}
	{\sffamily\Large\bfseries} 
	{\thesection}{1em}{} 
\titleformat{\subsection}
	{\sffamily\large\bfseries}   
	{\thesubsection}{1em}{} 
\titleformat{\subsubsection}
	{\sffamily\normalsize\bfseries} 
	{\thesubsubsection}{1em}{}

% disponi alberi
\usepackage{forest}

\forestset{
	rectstyle/.style={
		for tree={rectangle,draw,font=\large\sffamily}
	},
	roundstyle/.style={
		for tree={circle,draw,font=\large}
	}
}

% disponi algoritmi
\usepackage{algorithm}
\usepackage{algorithmic}
\makeatletter
\renewcommand{\ALG@name}{Algoritmo}
\makeatother

% disponi numeri di pagina
\usepackage{fancyhdr}
\fancyhf{} 
\fancyfoot[L]{\sffamily{\thepage}}

\makeatletter
\fancyhead[L]{\raisebox{1ex}[0pt][0pt]{\sffamily{\@title \ \@date}}} 
\fancyhead[R]{\raisebox{1ex}[0pt][0pt]{\sffamily{\@author}}}
\makeatother

\begin{document}

\pagestyle{fancy}
\thispagestyle{empty}
\renewcommand{\thispagestyle}[1]{}

\maketitle

\documentclass[a4paper,11pt]{article}
\usepackage[a4paper, margin=8em]{geometry}

% usa i pacchetti per la scrittura in italiano
\usepackage[french,italian]{babel}
\usepackage[T1]{fontenc}
\usepackage[utf8]{inputenc}
\frenchspacing 

% usa i pacchetti per la formattazione matematica
\usepackage{amsmath, amssymb, amsthm, amsfonts}

% usa altri pacchetti
\usepackage{gensymb}
\usepackage{hyperref}
\usepackage{standalone}

% imposta il titolo
\title{Appunti Calcolo Numerico}
\author{Luca Seggiani}
\date{2025}

% disegni
\usepackage{pgfplots}
\pgfplotsset{width=10cm,compat=1.9}

% imposta lo stile
% usa helvetica
\usepackage[scaled]{helvet}
% usa palatino
\usepackage{palatino}
% usa un font monospazio guardabile
\usepackage{lmodern}

\renewcommand{\rmdefault}{ppl}
\renewcommand{\sfdefault}{phv}
\renewcommand{\ttdefault}{lmtt}

% disponi il titolo
\makeatletter
\renewcommand{\maketitle} {
	\begin{center} 
		\begin{minipage}[t]{.8\textwidth}
			\textsf{\huge\bfseries \@title} 
		\end{minipage}%
		\begin{minipage}[t]{.2\textwidth}
			\raggedleft \vspace{-1.65em}
			\textsf{\small \@author} \vfill
			\textsf{\small \@date}
		\end{minipage}
		\par
	\end{center}

	\thispagestyle{empty}
	\pagestyle{fancy}
}
\makeatother

% disponi teoremi
\usepackage{tcolorbox}
\newtcolorbox[auto counter, number within=section]{theorem}[2][]{%
	colback=blue!10, 
	colframe=blue!40!black, 
	sharp corners=northwest,
	fonttitle=\sffamily\bfseries, 
	title=Teorema~\thetcbcounter: #2, 
	#1
}

% disponi definizioni
\newtcolorbox[auto counter, number within=section]{definition}[2][]{%
	colback=red!10,
	colframe=red!40!black,
	sharp corners=northwest,
	fonttitle=\sffamily\bfseries,
	title=Definizione~\thetcbcounter: #2,
	#1
}

% disponi problemi
\newtcolorbox[auto counter, number within=section]{problem}[2][]{%
	colback=green!10,
	colframe=green!40!black,
	sharp corners=northwest,
	fonttitle=\sffamily\bfseries,
	title=Problema~\thetcbcounter: #2,
	#1
}

% disponi codice
\usepackage{listings}
\usepackage[table]{xcolor}

\lstdefinestyle{codestyle}{
		backgroundcolor=\color{black!5}, 
		commentstyle=\color{codegreen},
		keywordstyle=\bfseries\color{magenta},
		numberstyle=\sffamily\tiny\color{black!60},
		stringstyle=\color{green!50!black},
		basicstyle=\ttfamily\footnotesize,
		breakatwhitespace=false,         
		breaklines=true,                 
		captionpos=b,                    
		keepspaces=true,                 
		numbers=left,                    
		numbersep=5pt,                  
		showspaces=false,                
		showstringspaces=false,
		showtabs=false,                  
		tabsize=2
}

\lstdefinestyle{shellstyle}{
		backgroundcolor=\color{black!5}, 
		basicstyle=\ttfamily\footnotesize\color{black}, 
		commentstyle=\color{black}, 
		keywordstyle=\color{black},
		numberstyle=\color{black!5},
		stringstyle=\color{black}, 
		showspaces=false,
		showstringspaces=false, 
		showtabs=false, 
		tabsize=2, 
		numbers=none, 
		breaklines=true
}

\lstdefinelanguage{javascript}{
	keywords={typeof, new, true, false, catch, function, return, null, catch, switch, var, if, in, while, do, else, case, break},
	keywordstyle=\color{blue}\bfseries,
	ndkeywords={class, export, boolean, throw, implements, import, this},
	ndkeywordstyle=\color{darkgray}\bfseries,
	identifierstyle=\color{black},
	sensitive=false,
	comment=[l]{//},
	morecomment=[s]{/*}{*/},
	commentstyle=\color{purple}\ttfamily,
	stringstyle=\color{red}\ttfamily,
	morestring=[b]',
	morestring=[b]"
}

% disponi sezioni
\usepackage{titlesec}

\titleformat{\section}
	{\sffamily\Large\bfseries} 
	{\thesection}{1em}{} 
\titleformat{\subsection}
	{\sffamily\large\bfseries}   
	{\thesubsection}{1em}{} 
\titleformat{\subsubsection}
	{\sffamily\normalsize\bfseries} 
	{\thesubsubsection}{1em}{}

% disponi alberi
\usepackage{forest}

\forestset{
	rectstyle/.style={
		for tree={rectangle,draw,font=\large\sffamily}
	},
	roundstyle/.style={
		for tree={circle,draw,font=\large}
	}
}

% disponi algoritmi
\usepackage{algorithm}
\usepackage{algorithmic}
\makeatletter
\renewcommand{\ALG@name}{Algoritmo}
\makeatother

% disponi numeri di pagina
\usepackage{fancyhdr}
\fancyhf{}
\fancyfoot[L]{\sffamily{\thepage}}

\makeatletter
\fancyhead[L]{\raisebox{1ex}[0pt][0pt]{\sffamily{\@title \ \@date}}} 
\fancyhead[R]{\raisebox{1ex}[0pt][0pt]{\sffamily{\@author}}}
\makeatother

\begin{document}

% sezione (data)
\section{Lezione del 24-02-25}

% stili pagina
\thispagestyle{empty}
\pagestyle{fancy}

% testo
\subsection{Introduzione al corso}
Il corso di \textbf{calcolo numerico}, o come viene definito oggi \textit{analisi numerica}, tratta lo studio degli algoritmi per problemi in campi continui (incognite in $\mathbb{R}$, siano queste numeri o funzioni, ecc...) o su grandi moli di dati.

Il programma del corso è così suddiviso:
\begin{enumerate}
	\item Analisi dell'errore per funzioni scalari;
	\item Richiami di algebra lineare (calcolo vettoriale e matriciale, ecc...);
	\item Risoluzione di sistemi lineari, cioè forme $A\mathbf{x} = b$;
	\item Interpolazione e approssimazione di funzioni nel senso dei minimi quadrati;
	\item Metodi per l'integrazione, cioè per forme $\int_a^b f(x) \, dx$;
	\item Equazioni non lineari, cioè ricerca dei punti $g(x) = 0$ per $g(x)$ non lineari;
	\item Problemi agli autovalori, cioè date matrici $A \in \mathbb{C}^{m \times n}$, trovare $(\lambda, \mathbf{x})$ tali che $A\mathbf{x} = \lambda \mathbf{x}$.
\end{enumerate}

\subsubsection{Matematica del continuo}
Abbiamo detto che i valori trattati sono continui, che intendiamo per appartenenti ad $\mathbb{R}$.
Un problema apparente dei numeri reali è che richiedono teoricamente un numero infinito di cifre per la loro rappresentazione.
Si rende quindi necessaria un'approssimazione in modo da ovviare ai problemi:
\begin{itemize}
	\item Rappresentare oggetti matematici con un numero infinito di parametri;
	\item Risolvere problemi che non hanno formule chiuse per la soluzione, ma richiedono approcci iterativi (discesa a gradiente, ecc...) e quindi che richiedono un approssimazione data dall'impossibilità di effettuare infiniti passi.
\end{itemize}

\subsubsection{Errori}
Avremo quindi bisogno di valutare degli \textbf{errori}, che saranno gli \textit{errori di approssimazione} come riportata sopra, uniti agli \textit{errori nei dati} già presenti nella nostra mole di dati.

Una domanda che potremo porci è come questi errori influiscono sul risultato che ci interessa.
Una prima distinzione può essere fra algoritmi \textbf{instabili} e \textbf{stabili}, cioè che \textit{amplificano} l'errore o lo mantengono costante.
Una seconda distinzione può essere sul \textbf{condizionamento} del problema, cioè la tendenza del problema a reagire in maniera drastica a piccole variazioni delle condizioni iniziali.

\subsubsection{Efficienza}
Un'altra considerazione importante è quella dell'\textbf{efficienza} degli algoritmi, cioè il tempo che questi richiedono per convergere ad una soluzione valida (non ottima, in quanto abbiamo visto dobbiamo troncare il numero di passi nel caso di approcci iterativi, che altrimenti potrebbe tendere ad infinito).
Questo viene stimato attraverso il \textit{costo computazionale}, tenendo conto di una certa \textit{accuratezza} che vogliamo stabilire.

\subsection{Rappresentazione dei reali}
Introduciamo la classe dei \textbf{numeri reali in virgola mobile}, atta a rappresentare numeri $x \in \mathbb{R}$ usando parametri che stanno in $\mathbb{N}$, o al limite in $\mathbb{Z}$, in quanto abbiamo visto possiamo gestire numeri di questo tipo in modo esatto nei calcolatori.

\begin{theorem}{Rappresentazione dei reali in virgola mobile}	
Fissata una base $\beta > 1$, con $\beta \in \mathbb{N}$, si può sempre trovare un esponente $e \in \mathbb{Z}$ e una successione di cifre $\{\alpha_j\}_{j = 1,2, ...}$ tali per cui:
$$
x = \mathrm{sgn}(x) \cdot \beta^e \cdot \sum_{j=1}^\infty \alpha_j \beta^{-j}
$$
\end{theorem}

Osserviamo che scelte tipiche per $\beta$ sono 10 (decimale), 2 (binario) e 16 (esadecimale).

Un problema di questa rappresentazione è che non è propriamente \textbf{unica}: ad esempio, potremo scrivere un approssimazione di $\pi$ come $3.14$, o come $0.314 \cdot 10$, equivalentemente.
Inoltre, possono esistere casi di numeri periodici, come $2.\overline{9}$, che sono effettivamente uguali ad altri (in questo caso $3$).

Sfruttiamo allora il seguente teorema, dato senza dimostrazione:
\begin{theorem}{Teorema di rappresentazione}
	Dati $\beta \in \mathbb{N}, \beta > 1$ base e $x \in \mathbb{R}, x \neq 0$, allora esiste ed è unica la rappresentazione $x = \mathrm{sgn}(x) \cdot \beta^e \cdot \sum_{j=1}^\infty \alpha_j \beta^{-j}$ (dal Teorema 1.1) tale che:
	\begin{itemize}
		\item $\alpha_1 \neq 0$;
		\item $!\exists k \in \mathbb{N} : \alpha_j = \beta - 1 \ \forall j > k$.
	\end{itemize}
\end{theorem}

Introducendo queste due limitazioni possiamo quindi ovviare al problema dell'unicità.

Diamo quindi alcune definizioni, riguardo alla rappresentazione appena vista:

\begin{definition}{Rappresentazione in virgola mobile normalizzata}
	L'unica rappresentazione che verifica il Teorema 1.2 si dice \textbf{rappresentazione in virgola mobile normalizzata}.
\end{definition}

e riguardo alla successione $\{\alpha_j\}$:

\begin{definition}{Mantissa}
	Prende il nome di \textbf{mantissa} la serie $\sum_{j=1}^\infty \alpha_j \beta^{-j}$.
\end{definition}

Notiamo che vale:
$$
\frac{1}{\beta} \leq \sum_{j=1}^\infty \alpha_j \beta^{-j} < 1
$$

In quanto:
\begin{itemize}
	\item Per il limite inferiore: 
		$$
			\sum_{j=1}^\infty \alpha_j \beta^{-j} \geq \beta^{-1} = \frac{1}{\beta}
		$$
	\item Per il limite superiore:
		$$
		\sum_{j=1}^\infty \alpha_j \beta^{-j} < (\beta - 1) \sum_{j=1}^\infty \beta^{-j} = (\beta - 1) \left( \frac{1}{1 - \beta^{-1}} - 1 \right) 
		$$ 
		$$
		= (\beta - 1) \frac{\beta^{-1}}{1-\beta^{-1}} = (\beta - 1) \frac{1}{\beta - 1} = 1
		$$
\end{itemize}

Vediamo che la mantissa non è effettivamente rappresentabile nella sua interezza in un calcolatore, in quanto richiederebbe un numero infinito di cifre. 
Si tronca quindi la mantissa, e si considera un range limitato per l'esponente.
Si definisce quindi un sottinsieme:
\begin{definition}{Numeri di macchina}
	Dati:
	$\beta \in \mathbb{N}$, $m \in \mathbb{N}$, $L, U \in \mathbb{Z}$ tali che $L \leq U$, si definisce $F(\beta, m, L, U)$ come:
			$$
			F =\{ x \in \mathbb{R} : x = \mathrm{sgn}(x) \cdot \beta^e \cdot \sum_{j=1}^m \alpha_j \beta^{-j}, \ L \leq e \leq U, \ \alpha_j = \{ 0, ..., \beta - 1 \}, \ \alpha_1 \neq 0 \} \, \cup \, \{0\}
			$$
			detto \textbf{numero di macchina}.
\end{definition}

L'inclusione dello zero è necessario in quanto questo non è compreso nel primo insieme dato.

Notiamo che l'insieme dei numeri di macchina non è equispaziato (ci sono più numeri vicino allo zero).
Presi intervalli $[\beta^{e - 1}, \beta^e)$, questi risulterebbero equispaziati se venissero considerati su una scala logaritmica base $\beta$.
All'interno di questi intervalli, invece, si hanno effettivamente numeri equispaziati, con periodo $\beta^e \cdot \beta^{-m} = \beta^{e - m}$.

\subsubsection{Limiti inferiori e superiori}

Potremmo chiederci quali sono i numeri \textbf{minimi} e \textbf{massimi} rappresentabili.

Osservando la definizione di $F$ si ha che il numero più piccolo possibile è quello che si ha prendendo $\beta =L$, e $\alpha_j = 0$ per ogni $j > 1,$ e $\alpha_1 =1$, quindi:
$$
\beta^L \cdot 1 \cdot \beta^{-1} = \beta^{L - 1}
$$

Il numero più grande si ha invece prendendo $\beta = U$ e $\alpha_j = \beta - 1$, e quindi:
$$
\beta^U (\beta - 1) \sum_{j=1}^m \beta^{-j} = \beta^U \left( \frac{1-\beta^{-m-1}}{1-\beta^{-1}} - 1\right) (\beta - 1)
$$
$$
= \beta^U \left( \frac{1-\beta^{-m}}{\beta - 1} \right) (\beta - 1) = \beta^U (1 - \beta^{-m}) 
$$

Potremmo poi chiederci quanti numeri macchina esistono fissati $\beta, m, L$ e $U$. 
Si hanno intanto due scelte di segno, $(U - L  + 1)$ scelte di esponenti e $(\beta^m - \beta^{m - 1})$ scelte di mantisse (tutti i numeri rappresentabili su $m$ cifre base $\beta$ meno quelli con prima cifra nulla) più lo zero, da cui:
$$
2 \cdot (U - L  + 1) \cdot (\beta^m - \beta^{m - 1}) + 1
$$

Le rappresentazioni più comuni dei numeri macchina sono definite dallo standard IEEE 754, che definisce:

\begin{table}[h!]
	\center \rowcolors{2}{white}{black!10}
	\begin{tabular} { c | c | c | c | c | c }
		\bfseries Precisione & \bfseries $\beta$ & \bfseries $m$ & \bfseries $L$ & \bfseries $U$ & \bfseries Dimensione \\ 
		\hline
		Singola & 2 & 24 & -126 & 127 & 4 byte (32 bit) \\ 
		Doppia & 2 & 53 & -1022 & 1023 & 8 byte (64 bit) \\ 
		Quadrupla & 2 & 113 & -16382 & 16383 & 16 byte (128 bit) \\ 
	\end{tabular}
\end{table}

Casi particolari potrebbero richiedere precisioni più grandi o più lasche.
Ultimamente, in particolare, si sono diffuse rappresentazioni a precisione più bassa (su 16 o addirittura 8 bit), in particolare nel campo delle reti neurali.

Il pacchetto MATLAB utilizza di default la precisione \textbf{doppia} come definita dallo standard IEEE 754.

\subsection{Arrotondamento}
Abbiamo visto come ci stiamo spostando dalla matematica esatta a una serie di approssimazioni.
In generale, vorremo partire da un certo numero reale $x \in \mathbb{R}$, ma non appartenente all'insieme dei numeri macchina ($x \not \in F(\beta, m, L ,U)$).
L'obiettivo è quello di riportare $x$ ad una sua \textit{approssimazione} appartenente ad $F(\beta, m, L, U)$.

Esistono 3 possibili situazioni:
\begin{itemize}
	\item $|x| > \beta^U (1 - \beta^{-m})$, cioè $x$ maggiore del massimo rappresentabile (\textbf{overflow});
	\item $|x| < \beta^{L - 1}$, cioè $x$ minore del minimo rappresentabile (\textbf{underflow});
	\item $\beta^{L - 1} \leq x \leq \beta^U (1 - \beta^{-m})$. Se nei casi precedenti si poteva scegliere, rispettivamente, un $M$ molto grande, o  $\infty$, e $0$, qui si può effettivamente procedere con l'arrotondamento.
\end{itemize}

\begin{definition}{Arrotondamento}
	Un arrotondamento è una funzione $RD: \mathbb{R} \rightarrow F(\beta, m, L, U)$, con $RD(x)$ uno dei numeri di macchina in $F(\beta, m, L, U)$ "vicini" ad $x$.
\end{definition}

Preso un certo reale $x$, avremo un numero di macchina a sinistra è uno a destra, cioè il primo più piccolo e il primo più grande.
Possiamo allora definire i seguenti arrotondamenti:
\begin{itemize}
	\item \textbf{Troncamento} (\textit{round down}): $$ Tr(x) = \lfloor x \rfloor $$
	\item \textbf{Round-up:} $$ Ru(x) = \lceil x \rceil$$
	\item \textbf{Round-to-nearest:} 
		$$
		Rn(x) =
			\begin{cases}
				\lceil x \rceil, \quad \alpha_{m + 1} \geq \frac{\beta}{2} \\ 
				\lfloor x \rfloor, \quad \alpha_{m + 1} < \frac{\beta}{2} \\ 
			\end{cases}
		$$
		con $\alpha_{m + 1}$ la prima cifra che viene scartata dall'arrotondamento. 
\end{itemize}

Notiamo poi che preso il troncamento: 
$$
Tr(x) = \mathrm{sgn}(x) \beta^e \sum_{j = 1}^{m} \alpha_j \beta^{-j}
$$ 
il round up corrispondente sarà: 
$$
Ru(x) = \mathrm{sgn}(x) \beta^e \left( \sum_{j = 1}^{m} \alpha_j \beta^{-j} + \beta^{-m}\right)
$$

\subsubsection{Errore di arrotondamento}
Valgono le disuguaglianze:
\begin{itemize}
	\item $ |Tr(x) - x| \leq \beta^{e - m} $ \\ 
	\item $ |Rn(x) - x| \leq \frac{\beta^{e - m}}{2} $, che è il minimo errore possibile cioè:
		$$
			|Rn(x) - x| = \mathrm{min}_{RD(x)}(RD(x) - x)
		$$
	Per questo motivo da qui in poi assumeremo quindi di prendere sempre $Rn(x)$.
\end{itemize}

\begin{definition}{Errori di arrotondamento}
	Definiamo:
	\begin{itemize}
		\item \textbf{Errore assoluto di arrotondamento:} $x - Rn(x) = \sigma_x$
		\item \textbf{Errore relativo di arrotondamento:} $\frac{x - Rn(x)}{x} = \epsilon_x$
	\end{itemize}
\end{definition}

La definizione di errore relativo è utile per avere un errore pressochè costante su tutta la retta dei reali.
Si ha quindi che:

\begin{itemize}
	\item Riguardo all'errore assoluto: $$|\sigma_x| \leq \frac{\beta^{e - m}}{2}$$
	\item Riguardo all'errore relativo: $$|\epsilon_x| \leq \frac{\beta^{e - m}}{2\beta^{e - 1}} = \frac{1}{2}\beta^{1 - m}$$ visto che $|x| > \beta^{e - 1}$. Notiamo che questo errore non dipende dall'esponente $e$, che è quello che desideravamo.
\end{itemize}

Abbiamo quindi che l'insieme dei numeri in virgola mobile garantiscono un errore relativo limitato in modo uniforme, se non si incombe in overflow o underflow.

\begin{definition}{Precisione}
	La quantità $U = \beta^{1-m}$  viene detta \textbf{precisione di macchina} di una certa rappresentazione a virgola mobile.	
\end{definition}

Ad esempio, nella precisioni doppia e singola, $U \approx 10^{-16}$ e $U \approx 10^{-8}$, che significa rispettivamente prime 15 o prime 7 cifre esatte.

\subsubsection{Dettagli di implementazione}
Prendiamo ad esempio la doppia precisione.
Avremo:
\begin{itemize}
	\item 1 bit di segno;
	\item 52 bit di mantissa (con 1 bit implicito impostato ad 1, per $\alpha_1 = 1$);
	\item 11 bit di esponente in rappresentazione con offset.
\end{itemize}

Riguardo agli altri formati si ha:

\begin{table}[h!]
	\center \rowcolors{2}{white}{black!10}
	\begin{tabular} { c | c | c | c | c}
		\bfseries Precisione & \bfseries Segno & \bfseries Esponente & \bfseries Mantissa & \bfseries U \\ 
		\hline
		Singola & 1 & 8 & 23 & $\approx 10^{-8}$ \\
		Doppia & 1 & 11 & 52 & $\approx 10^{-16}$ \\
		Quadrupla & 1 & 15 & 112 & $\approx 10^{-34}$ \\
	\end{tabular}
\end{table}

\subsection{Numeri sottonormalizzati}
Vediamo una tecnica per la rappresentazione di numeri più piccoli di $\beta^{L - 1}$, implementata nella maggior parte dei pacchetti software moderni (e nelle implementazioni degli standard a virgola mobile disponibili nei processori moderni).
Si ha che se $x \in [0, \beta^{L - 1}]$, allora si assume come prima cifra della mantissa $0$, con esponente fisso ad $L$. 
Questo ci permette di rappresentare più numeri vicino allo zero. 

Si avranno quindi numeri equispaziati fra 0 e $\beta^{L - 1}$ con distanza $\beta^{L - m}$.
Inoltre, sui numeri sottonormalizzati si avrà un errore relativo non limitato da $\frac{1}{2} \beta^{1 - m}$, ma che invece aumenta avvicinandosi a 0. 

I numeri sottonormalizzati sono indicati da un \textbf{valore speciale dell'esponente}, cosa che accade anche per:
\begin{itemize}
	\item I valori $+\infty$ e $-\infty$;
	\item I valori NaN (Not a Number) con relativi codici di errore in mantissa.
\end{itemize}

La presenza di questi valori rende necessaria l'approssimazione delle precisioni per i numeri in virgola mobile.

\end{document}


\documentclass[a4paper,11pt]{article}
\usepackage[a4paper, margin=8em]{geometry}

% usa i pacchetti per la scrittura in italiano
\usepackage[french,italian]{babel}
\usepackage[T1]{fontenc}
\usepackage[utf8]{inputenc}
\frenchspacing 

% usa i pacchetti per la formattazione matematica
\usepackage{amsmath, amssymb, amsthm, amsfonts}

% usa altri pacchetti
\usepackage{gensymb}
\usepackage{hyperref}
\usepackage{standalone}

% imposta il titolo
\title{Appunti Calcolo Numerico}
\author{Luca Seggiani}
\date{2025}

% disegni
\usepackage{pgfplots}
\pgfplotsset{width=10cm,compat=1.9}

% imposta lo stile
% usa helvetica
\usepackage[scaled]{helvet}
% usa palatino
\usepackage{palatino}
% usa un font monospazio guardabile
\usepackage{lmodern}

\renewcommand{\rmdefault}{ppl}
\renewcommand{\sfdefault}{phv}
\renewcommand{\ttdefault}{lmtt}

% disponi il titolo
\makeatletter
\renewcommand{\maketitle} {
	\begin{center} 
		\begin{minipage}[t]{.8\textwidth}
			\textsf{\huge\bfseries \@title} 
		\end{minipage}%
		\begin{minipage}[t]{.2\textwidth}
			\raggedleft \vspace{-1.65em}
			\textsf{\small \@author} \vfill
			\textsf{\small \@date}
		\end{minipage}
		\par
	\end{center}

	\thispagestyle{empty}
	\pagestyle{fancy}
}
\makeatother

% disponi teoremi
\usepackage{tcolorbox}
\newtcolorbox[auto counter, number within=section]{theorem}[2][]{%
	colback=blue!10, 
	colframe=blue!40!black, 
	sharp corners=northwest,
	fonttitle=\sffamily\bfseries, 
	title=Teorema~\thetcbcounter: #2, 
	#1
}

% disponi definizioni
\newtcolorbox[auto counter, number within=section]{definition}[2][]{%
	colback=red!10,
	colframe=red!40!black,
	sharp corners=northwest,
	fonttitle=\sffamily\bfseries,
	title=Definizione~\thetcbcounter: #2,
	#1
}

% disponi problemi
\newtcolorbox[auto counter, number within=section]{problem}[2][]{%
	colback=green!10,
	colframe=green!40!black,
	sharp corners=northwest,
	fonttitle=\sffamily\bfseries,
	title=Problema~\thetcbcounter: #2,
	#1
}

% disponi codice
\usepackage{listings}
\usepackage[table]{xcolor}

\lstdefinestyle{codestyle}{
		backgroundcolor=\color{black!5}, 
		commentstyle=\color{codegreen},
		keywordstyle=\bfseries\color{magenta},
		numberstyle=\sffamily\tiny\color{black!60},
		stringstyle=\color{green!50!black},
		basicstyle=\ttfamily\footnotesize,
		breakatwhitespace=false,         
		breaklines=true,                 
		captionpos=b,                    
		keepspaces=true,                 
		numbers=left,                    
		numbersep=5pt,                  
		showspaces=false,                
		showstringspaces=false,
		showtabs=false,                  
		tabsize=2
}

\lstdefinestyle{shellstyle}{
		backgroundcolor=\color{black!5}, 
		basicstyle=\ttfamily\footnotesize\color{black}, 
		commentstyle=\color{black}, 
		keywordstyle=\color{black},
		numberstyle=\color{black!5},
		stringstyle=\color{black}, 
		showspaces=false,
		showstringspaces=false, 
		showtabs=false, 
		tabsize=2, 
		numbers=none, 
		breaklines=true
}

\lstdefinelanguage{javascript}{
	keywords={typeof, new, true, false, catch, function, return, null, catch, switch, var, if, in, while, do, else, case, break},
	keywordstyle=\color{blue}\bfseries,
	ndkeywords={class, export, boolean, throw, implements, import, this},
	ndkeywordstyle=\color{darkgray}\bfseries,
	identifierstyle=\color{black},
	sensitive=false,
	comment=[l]{//},
	morecomment=[s]{/*}{*/},
	commentstyle=\color{purple}\ttfamily,
	stringstyle=\color{red}\ttfamily,
	morestring=[b]',
	morestring=[b]"
}

% disponi sezioni
\usepackage{titlesec}

\titleformat{\section}
	{\sffamily\Large\bfseries} 
	{\thesection}{1em}{} 
\titleformat{\subsection}
	{\sffamily\large\bfseries}   
	{\thesubsection}{1em}{} 
\titleformat{\subsubsection}
	{\sffamily\normalsize\bfseries} 
	{\thesubsubsection}{1em}{}

% disponi alberi
\usepackage{forest}

\forestset{
	rectstyle/.style={
		for tree={rectangle,draw,font=\large\sffamily}
	},
	roundstyle/.style={
		for tree={circle,draw,font=\large}
	}
}

% disponi algoritmi
\usepackage{algorithm}
\usepackage{algorithmic}
\makeatletter
\renewcommand{\ALG@name}{Algoritmo}
\makeatother

% disponi numeri di pagina
\usepackage{fancyhdr}
\fancyhf{} 
\fancyfoot[L]{\sffamily{\thepage}}

\makeatletter
\fancyhead[L]{\raisebox{1ex}[0pt][0pt]{\sffamily{\@title \ \@date}}} 
\fancyhead[R]{\raisebox{1ex}[0pt][0pt]{\sffamily{\@author}}}
\makeatother

\begin{document}

% sezione (data)
\section{Lezione del 28-02-25}

% stili pagina
\thispagestyle{empty}
\pagestyle{fancy}

% testo
\subsection{Operazioni sui numeri macchina}
Abbiamo introdotto l'insieme dei numeri macchina.
Vediamo adesso come eseguire \textbf{operazioni} fra elementi di questi insiemi.

Notiamo che, di base, dati $x, y \in F(\beta, m, L, U)$, non necessariamente $x \circ y \in F(\beta, m, L, U)$ per le comuni operazioni aritmetiche $+, -, \times, \div$.

Quello che faremo è quindi approssimare tali operazioni, cioè dire:
\begin{itemize}
	\item $x \oplus y = Rn(x + y)$
	\item $x \ominus y = Rn(x - y)$
	\item $x \otimes y = Rn(x \times y)$
	\item $x \oslash y = Rn(x \div y)$
\end{itemize}

Effetto di questa approsimazione è negare proprietà note dei reali, come ad esempio l'associativa:
$$
(x \oplus y) \oplus z \neq x \oplus (y \oplus z)
$$
$$
(x \oplus y) \otimes z \neq x \oplus (y \otimes z)
$$
Cioè, la valutazione di una formula con ordini diversi ma equivalenti in aritmetica esatta può portare a risultati differenti nell'insieme dei numeri di mmacchina.

\subsubsection{Errore nel calcolo di funzione}
Sia $f:\mathbb{R}^m \rightarrow \mathbb{R}$, e si voglia calcolare $f(P_0)$, con $P_0 =\left(x_1^{(0)}, x_2^{(0)}, ...x_m^{(0)}\right) \in \mathbb{R}^m$.
Le operazioni aritmetiche $+, -, \times, \div$ possono essere viste come funzioni di questo tipo.
Ci interroghiamo quindi sulla fonte dell'errore nella loro valutazione:
\begin{enumerate}
	\item Nel caso contenga funzioni irrazionali o trascendenti, $f$ verrà approssimata con una funzione $\overline{f}$ che coinvolge solo operazioni aritmetiche di base $+, -, \times, \div$;
	\item $\overline{f}$ viene tradotta in un \textit{algoritmo} $\overline{f}_a$, ovvero in una formula che coinvolge $\oplus, \ominus, \otimes, \oslash \leftarrow +, -, \times, \div$;
	\item Potrebbe essere che $P_0 \neq F(\beta, m, L, U)$, e quindi viene approssimato a $P_1 = Rn(P_0)$.
\end{enumerate}

Quindi, vogliamo $f(P_0)$ ma possiamo solo approssimarla come $\overline{f}_a(P_1)$.

Ad esempio, poniamo di voler calcolare $e^\pi$.
I passaggi nell'ordine appena visto saranno:
\begin{enumerate}
	\item Approssimamo l'esponenziale al secondo grado dello svillupppo di Taylor:
		$$
			e^x \approx 1 + x + \frac{x^2}{2} = \overline{f}(x)
		$$
	\item Si riporta la $\overline{f}(x)$ a $\overline{f}_a(x)$:
		$$
			1 \oplus \left( x \oplus ( (x \otimes x) \oslash 2) \right)
		$$
	\item Si approssima $\pi$ al numero macchina più vicino:
		$$
			Rn(\pi) = 3.1415 = P_1
		$$
\end{enumerate}

Avremo quindi la formula finale:
$$
1 \oplus \left( P_1 \oplus ( (P_1 \otimes P_1) \oslash 2) \right)
$$

# riporta grafo (o albero) che la rappresenta

\par\medskip

Diamo quindi la definizione di \textbf{errore totale}:
\begin{definition}{Errore totale}
	Data $f : \mathbb{R}^m \rightarrow \mathbb{R}$, un punto $P_0 \in \mathbb{R}^m$ ed un algoritmo $\overline{f}_a$, l'errore totale è dato da:
	$$
		S_f = \overline{f}_a (P_1) - f(P_0), \quad P_1 = Rn(P_0)
	$$
\end{definition}

\subsubsection{Errore di funzioni razionali}
Assumiamo per adesso $f$ \textbf{funzione razionale} ovvero $f$ si può definire con un numero di operazioni in $+, -, \times, \div$.
Assumere funzioni razionali ci permette di prendere $f = \overline{f}$ e $f_a = \overline{f}_a$ (non ci sono irrazionali da riportare ai razionali).
Potremo allora dire:

$$
S_f = f_a(P_1) - f(P_0) = f_a(P_1) - f(P_1) + f(P1) - f(P_0)
$$
che possiamo dividere in:
$$
S_f = S_a + S_d =, \quad S_a = f_a(P_1) - f(P_1), \quad S_d = f(P_1) - f(P_0)
$$
che chiamiamo rispetivamente \textbf{errore totale algoritmico} e \textbf{errore totale inerente}.

Allo stesso modo possiamo definire l'\textbf{errore relativo}:
$$
\epsilon_f = \frac{S_f}{f(P_0)} = \frac{f_a(P_1) - f(P_0)}{f(P_0)} = \frac{f_a(P_1) - f(P_1)}{f(P_0)} + \frac{f(P_1) - f(P_0)}{f(P_0)}
$$
$$
= \frac{f_a(P_1) - f(P_1)}{f(P_1)} \cdot \frac{f(P_1)}{f(P_0)} + \frac{f(P_1) - f(P_0)}{f(P_0)} 
$$
che si divide nuovamente in :
$$
\epsilon_f = \epsilon_a + \epsilon_d, \quad \epsilon_a = \frac{f_a(P_1) - f(P_1)}{f(P_1)}, \quad \epsilon_d = \frac{f(P_1) - f(P_0)}{f(P_0)}
$$
che chiamiamo rispetivamente \textbf{errore relativo algoritmico} e \textbf{errore relativo inerente}.
Questo viene da:
$$
\epsilon_f = [...] = \frac{f_a(P_1) - f(P_1)}{f(P_1)} \cdot \frac{f(P_1)}{f(P_0)} + \frac{f(P_1) - f(P_0)}{f(P_0)} 
$$
si nota che $\frac{f(P_1)}{f(P_0)} = 1 + \frac{f(P_1) - f(P_0)}{f(P_0)}$, e quindi:
$$
= \epsilon_a \cdot \left( 1 + \frac{f(P_1) - f(P_0)}{f(P_0)} \right) + \epsilon_d = \epsilon_a(1 + \epsilon_d) + \epsilon_d = \epsilon_a + \epsilon_d + \epsilon_a \epsilon_d \approx \epsilon_a + \epsilon_d
$$
assumendo $\epsilon_a \epsilon_d \approx 0$.

\par\smallskip

In generale, per limitare $|S_f|$ cercheremo disuguaglianze $|S_a| < tau_1$, $|S_d| < \tau_2$, da cui:
$$
	|S_f| < \tau_1 + \tau_2
$$

\subsubsection{Errore inerente}
Avevamo quindi definito l'errore totale inerente:
$$
S_d = f(P_1) - f(P_0)
$$

Sotto l'ipotesi $f \in C^1(D)$ per $D \subset \mathbb{R}^m$ che contiene $P_0$, si può usare  lo sviluppo di Taylor di $f$ in $P_0$, troncato al primo ordine:
$$
f(P_1) - f(P_0) = f(P_0) + \nabla f(\overline{P})^T (P_1 - P_0) - f(P_0) = \nabla f(\overline{P})^T (P_1 - P_0)
$$
$$
= \sum_{j=1}^m \frac{\partial f}{\partial x_j}(\overline{P}) \cdot \left(x_j^{(1)} - x_j^{(0)}\right)
\approx \sum_{j=1}^m \frac{\partial f}{\partial x_j}{P_0} \cdot \left(x_j^{(1)} - x_j^{(0)}\right)
$$
dove $\overline{P}$ è un punto che sta sul segmento $\overline{P_1 P_0}$.
Da questo potremo dire:
$$
S_d = \sum_{j=1}^m \frac{\partial f}{\partial x_j}{P_0} \cdot S_j
$$
dove $S_j = \left(x_j^{(1)} - x_j^{(0)}\right)$ è l'\textbf{errore di arrotondemento} nella componente $j$ di $P_0$, e $\frac{\partial f}{\partial x_j}{P_0}$ viene detto \textbf{coefficiente di amplificazione}.

\par\smallskip

Per l'errore relativo inerente potremo fare considerazioni simili: # metti qual'è come sopra
$$
\epsilon_d = \frac{ \sum_{j=1}^{m} \frac{\partial f}{\partial x_j} (P_0) \cdot S_j }{f(P_0)} = \sum_{j=1}^m \frac{ x_j^{(1)} - x_j^{(0)} }{x_j^{(0)}} \cdot \frac{\partial f}{\partial x_j} (P_0) \cdot \frac{x_j^{(0)}}{f(P_0)}  
$$
dove $\epsilon_j = \frac{ x_j^{(1)} - x_j^{(0)} }{x_j^{(0)}}$ sarà l'\textbf{errore di arrotondamento relativo} nella componente $j$ di $P_0$ e $P_j = \frac{\partial f}{\partial x_j} (P_0) \cdot \frac{x_j^{(0)}}{f(P_0)}$ viene detto \textbf{coefficiente di amplificazione dell'errore relativo}.

La formula finale sarà quindi:
$$
\epsilon_d = \sum_{j=1}^m \epsilon_j P_j
$$

I problemi in cui si devono calcolare $f$ i cui coefficient di amplificazione degli errori relativi sono grandi in modulo (o ce n'è almeno uno sufficientemente grande) si dicono \textbf{malcondizionati}.
Viceversa, se $|P_j|$ è vicino a $1$ per ogni $j$ il problema si dice \textbf{ben condizionato}, cioè che $\epsilon_d$ è di un ordine di grandezza comparabile a $\max(\epsilon_i)$

Notiamo che il condizionamento di un problema dipende solamente dalla sua struttura matematica. 

\subsubsection{Errori inerenti delle operazioni aritmetiche}
Vediamo gli errori inerenti associati alle 4 operazioni aritmetiche $+, -, \times, \div$:
\begin{table}[H]
	\center \rowcolors{2}{white}{black!10}
	\begin{tabular} { c | c | c }
		\bfseries Operazione & $S_d$ & $\epsilon_d$ \\
		\hline
		$x \oplus y$ & $S_x + S_y$ & $\frac{x}{x+y} \epsilon_x + \frac{y}{x - y} \epsilon_y$ \\ 
		$x \ominus y$ & $S_x - S_y$ & $\frac{x}{x+y} \epsilon_x - \frac{y}{x - y} \epsilon_y$ \\ 
		$x \otimes y$ & $yS_x + xS_y$ & $\epsilon_x + \epsilon_y$ \\ 
		$x \oslash y$ & $\frac{1}{y}S_x - \frac{x}{y^2}S_y$ & $\epsilon_x - \epsilon_y$ \\ 
	\end{tabular}
\end{table}

Notiamo come somme e sottrazioni non amplificano gli errori totali, mentre prodotti e divisioni non amplificano gli errori relativi (riguardo agli errori inerenti).
Questo significa che somme e sottrazioni possono avere errori relativi grandi quando $|x + y| << \min \{ |x|, |y|\}$. 
Questo effetto viene detto \textbf{cancellazione numerica}.

\subsubsection{Errore algoritmico} # hai fatto un casino boia fra assoluto e totale
Avevamo definito un algoritmo $f_a(x)$ di cui vogliamo stimare l'errore algoritmico assoluto $S_a = f_a(P_1) - f(P_1)$.
Assumiamo $P_1 = Rn(P_0) \in F(\beta, m, L, U)$, cioè gli operandi come privi di errori iniziali di arrotondamento. 

L'idea è di seguire l'errore generato dall'algoritmo sul grafo (o albero) che lo rappresenta sgruttando le relazioni perl'errore inerente nelle 4 operazioni aritmetiche.
Prendiamo ad esempio la funzione:
$$
f(x, y, z, w) = x \cdot \left(\frac{y}{z} - w\right)
$$
Avremo i risultati intermedi:
\begin{itemize}
	\item $r_1 = \frac{y}{z}$
	\item $r_2 = r_1 - w$
	\item $r_3 = r_2 \cdot x$
\end{itemize}
Di cui riportiamo il grafo:
\begin{center}
	\begin{forest}
		[$r_3$, roundstyle
			[$x$]
			[$r_2$
				[$w$]
				[$r_1$
					[$y$]
					[$z$]
				]
			]
		]	
	\end{forest}
\end{center} # aggiungi errori formule fallo bellino
dove $\epsilon_3$, $\epsilon_2$ e $\epsilon_1$ rappresentano gli errori di troncamento dei risultati intermedi e $\epsilon_{r3}$, $\epsilon_{r2}$ e $\epsilon_{r1}$ rappresentano gli errori inerenti delle singole operazioni per il calcolo dei risultati intermedi.

Partiamo dalla radice per valutare gli errori:
$$
\epsilon_{r_3} = \epsilon_3 + \epsilon_x + \epsilon_{r_2} = \epsilon_3 + \epsilon_{r_2} 
$$
$$
= \epsilon_3 + \epsilon_2 + \left( \frac{-zw}{y-zw} \right) \cdot \epsilon_w + \frac{y}{y-zw}\cdot\epsilon_{r_1} = \epsilon_3 + \epsilon_2 + \frac{y}{y - zw} \cdot \epsilon_{r_1}
$$
$$
= \epsilon_3 + \epsilon_2 + \frac{y}{y-zw}\left( \epsilon_1 + \epsilon_y - \epsilon_z \right) = \epsilon_3 + \epsilon_2 + \frac{y}{y-zw}\cdot \epsilon_1 = \epsilon_a
$$

Dove per la stima di $\epsilon_3$, $\epsilon_2$ e $\epsilon_1$, vale $\epsilon_i \leq U$ precisione macchina.
Nel caso di errori assoluti vale $|S_i| \leq U \cdot \max(x_i)$ considerata ogni variabile $x_i$.

Chiaramente, diversi algoritmi equivalenti in aritmetica esatta potranno avere errori algoritmici diversi fatte tutte le approssimazioni.

# riporta esempio x^2 - y^2, due modi uno x,y dipendente l'altro |\epsilon_a| < 3 U fisso

Abbiamo visto quindi tenciche per la stima di $\epsilon_a$ e $\epsilon_d$ ($S_a$ e $S_d$), che ci permettono di calcolare $|\epsilon_f| \leq |\epsilon_a| + |\epsilon_d|$ ($|S_f| \leq |S_a| + |S_d|$). 

Un problema classico sarà quello di, data $f$, un algoritmo risolutivo $f_a$ e una stima degli errori $d_{x_i}$, di stimare $S_f$ per $P_0 \in D \subset \mathbb{R}^m$.

Il problema inverso potrebbe essere quello di, dato $\tau > 0$, $f$ e un punto $P_0 \in \mathbb{R}^n$, determinatre un algoritmo ed un valore di precisione macchina $U$ tali per cui $|s_f| < \tau$.

\end{document}


\documentclass[a4paper,11pt]{article}
\usepackage[a4paper, margin=8em]{geometry}

% usa i pacchetti per la scrittura in italiano
\usepackage[french,italian]{babel}
\usepackage[T1]{fontenc}
\usepackage[utf8]{inputenc}
\frenchspacing 

% usa i pacchetti per la formattazione matematica
\usepackage{amsmath, amssymb, amsthm, amsfonts}

% usa altri pacchetti
\usepackage{gensymb}
\usepackage{hyperref}
\usepackage{standalone}

% imposta il titolo
\title{Appunti Calcolo Numerico}
\author{Luca Seggiani}
\date{2025}

% disegni
\usepackage{pgfplots}
\pgfplotsset{width=10cm,compat=1.9}

% imposta lo stile
% usa helvetica
\usepackage[scaled]{helvet}
% usa palatino
\usepackage{palatino}
% usa un font monospazio guardabile
\usepackage{lmodern}

\renewcommand{\rmdefault}{ppl}
\renewcommand{\sfdefault}{phv}
\renewcommand{\ttdefault}{lmtt}

% disponi il titolo
\makeatletter
\renewcommand{\maketitle} {
	\begin{center} 
		\begin{minipage}[t]{.8\textwidth}
			\textsf{\huge\bfseries \@title} 
		\end{minipage}%
		\begin{minipage}[t]{.2\textwidth}
			\raggedleft \vspace{-1.65em}
			\textsf{\small \@author} \vfill
			\textsf{\small \@date}
		\end{minipage}
		\par
	\end{center}

	\thispagestyle{empty}
	\pagestyle{fancy}
}
\makeatother

% disponi teoremi
\usepackage{tcolorbox}
\newtcolorbox[auto counter, number within=section]{theorem}[2][]{%
	colback=blue!10, 
	colframe=blue!40!black, 
	sharp corners=northwest,
	fonttitle=\sffamily\bfseries, 
	title=Teorema~\thetcbcounter: #2, 
	#1
}

% disponi definizioni
\newtcolorbox[auto counter, number within=section]{definition}[2][]{%
	colback=red!10,
	colframe=red!40!black,
	sharp corners=northwest,
	fonttitle=\sffamily\bfseries,
	title=Definizione~\thetcbcounter: #2,
	#1
}

% disponi problemi
\newtcolorbox[auto counter, number within=section]{problem}[2][]{%
	colback=green!10,
	colframe=green!40!black,
	sharp corners=northwest,
	fonttitle=\sffamily\bfseries,
	title=Problema~\thetcbcounter: #2,
	#1
}

% disponi codice
\usepackage{listings}
\usepackage[table]{xcolor}

\lstdefinestyle{codestyle}{
		backgroundcolor=\color{black!5}, 
		commentstyle=\color{codegreen},
		keywordstyle=\bfseries\color{magenta},
		numberstyle=\sffamily\tiny\color{black!60},
		stringstyle=\color{green!50!black},
		basicstyle=\ttfamily\footnotesize,
		breakatwhitespace=false,         
		breaklines=true,                 
		captionpos=b,                    
		keepspaces=true,                 
		numbers=left,                    
		numbersep=5pt,                  
		showspaces=false,                
		showstringspaces=false,
		showtabs=false,                  
		tabsize=2
}

\lstdefinestyle{shellstyle}{
		backgroundcolor=\color{black!5}, 
		basicstyle=\ttfamily\footnotesize\color{black}, 
		commentstyle=\color{black}, 
		keywordstyle=\color{black},
		numberstyle=\color{black!5},
		stringstyle=\color{black}, 
		showspaces=false,
		showstringspaces=false, 
		showtabs=false, 
		tabsize=2, 
		numbers=none, 
		breaklines=true
}

\lstdefinelanguage{javascript}{
	keywords={typeof, new, true, false, catch, function, return, null, catch, switch, var, if, in, while, do, else, case, break},
	keywordstyle=\color{blue}\bfseries,
	ndkeywords={class, export, boolean, throw, implements, import, this},
	ndkeywordstyle=\color{darkgray}\bfseries,
	identifierstyle=\color{black},
	sensitive=false,
	comment=[l]{//},
	morecomment=[s]{/*}{*/},
	commentstyle=\color{purple}\ttfamily,
	stringstyle=\color{red}\ttfamily,
	morestring=[b]',
	morestring=[b]"
}

% disponi sezioni
\usepackage{titlesec}

\titleformat{\section}
	{\sffamily\Large\bfseries} 
	{\thesection}{1em}{} 
\titleformat{\subsection}
	{\sffamily\large\bfseries}   
	{\thesubsection}{1em}{} 
\titleformat{\subsubsection}
	{\sffamily\normalsize\bfseries} 
	{\thesubsubsection}{1em}{}

% disponi alberi
\usepackage{forest}

\forestset{
	rectstyle/.style={
		for tree={rectangle,draw,font=\large\sffamily}
	},
	roundstyle/.style={
		for tree={circle,draw,font=\large}
	}
}

% disponi algoritmi
\usepackage{algorithm}
\usepackage{algorithmic}
\makeatletter
\renewcommand{\ALG@name}{Algoritmo}
\makeatother

% disponi numeri di pagina
\usepackage{fancyhdr}
\fancyhf{} 
\fancyfoot[L]{\sffamily{\thepage}}

\makeatletter
\fancyhead[L]{\raisebox{1ex}[0pt][0pt]{\sffamily{\@title \ \@date}}} 
\fancyhead[R]{\raisebox{1ex}[0pt][0pt]{\sffamily{\@author}}}
\makeatother

\begin{document}

% sezione (data)
\section{Lezione del 03-03-25}

% stili pagina
\thispagestyle{empty}
\pagestyle{fancy}

% testo
\subsection{Riassunto sulla stima dell'errore}
Riassumiamo quindi le regole viste per la stima dell'errore su funzioni razionali.
Avevamo dato la definizione di errore \textbf{assoluto} $\sigma_f$ e errore \textbf{relativo} $\epsilon_f$, entrambi composti da due fattori denominati errore \textbf{algoritmico} e errore \textbf{inerente}, con pedici rispettivamente $a$ e $d$.

\begin{itemize}
	\item 
Riguardo all'errore \textbf{inerente assoluto} avevamo preso su un dominio $D$ la stime:

$$
|\sigma_d| \leq \sum_{j = 1}^m A_j \cdot |\sigma_j|
$$
con $|\sigma^j|$ \textbf{errore di arrotondamento} e $A_j$ \textbf{coefficiente di amplificazione}:
$$
A_j = \max_{P \in D} \left( \frac{\partial f}{\partial x_j}(P) \right)
$$

Per l'errore di arrotondamento avevamo visto potevamo prendere:
$$
|\sigma_j| \leq U \cdot |x_j|
$$
con $U$ precisione macchina.

	\item
Riguardo all'\textbf{errore inerente relativo} avevamo invece preso:
$$
|\epsilon_d|\leq \sum_{j = 1}^m \overline{A}_j \cdot |\epsilon_j|
$$
con
$|\epsilon_j|$ \textbf{errore di arrotondamento relativo} e $\overline{\sigma_j}$ \textbf{coefficiente di amplificazione relativo}:
$$
\overline{A_j} = \max_{P \in D} \left( \frac{x_j \cdot \frac{\partial f}{\partial x_j} (P) }{f(P)} \right)
$$

Per l'errore di arrotondamento relativo potevamo quindi prendere:
$$
|\epsilon_j| \leq U
$$
\end{itemize}

# esempio errore algortmico di $f(x_1, x_2) = (x_1 + 1) x_1 + x_2$

\subsection{Errori di funzioni non razionali}
Abbiamo finora trascurato il caso di funzioni non razionali.
Prendiamo ad esempio di voler calcolare l'errore su funzioni come $e^{\cos(x + y)}$.
In questo caso sara' necessario usare un approssimazione razionale di $f$ che chiamiamo $\overline{f}$, che poi porteremo a $\overline{f}_a$ che usa operazioni macchina detta \textbf{algoritmo}.
In questo caso l'errore sara' dato dall'\textit{errore inerente}, dall'\textit{errore algortimico} e dall'\textbf{errore analitico} $\sigma_{an}$ della funzione, cioe' potremo dire:

$$
\overline{f}_a (P_1) - f(P_0) = \overline{f}_a (P_1) - \overline{f} (P_1) = \overline{f} (P_1) - f(P_1) + f(P_1) - f(P_0) = \sigma_a + \sigma_{an} + \sigma_d
$$
	L'errore inerente sara' calcolato sulla $f$ originale, mentre l'errore analitico sara' calcolato con la nuova $\overline{f}$, e in particolare dipendera' dall'approssimazione razionale che usiamo.

Vediamo per adesso approssimazioni polinomiali attraverso la \textbf{formula di Taylor}. Nel caso scalare si ha:

\begin{theorem}{Formula di Taylor}
	Data $f : \mathbb{R} \rightarrow \mathbb{R}$, $f \in C^1$, allora dato $x_0 \in \mathbb{R}$ si ha:
	$$
	f(x) = \sum_{n = 0}^k \frac{ f^{(n)}(x_0) }{ n! } \cdot (x - x_0)^n + \frac{ f^{(k + 1)} ( \eta ) }{(k + 1)!} (x - x_0)^{k + 1} 
	$$
	Dove:
	$$
		\epsilon_l = \frac{ f^{(k + 1)} ( \eta ) }{(k + 1)!} (x - x_0)^{k + 1}
	$$
	rappresenta l'\textbf{errore di Lagrange} al $k$-esimo grado, con $\eta \in [x_0, x]$ il punto di massimo della $k + 1$-esima derivata di $f$.
\end{theorem}

In questo caso:
$$
\overline{f}(x) = \sum_{n = 0}^k \frac{ f^{(n)}(x_0) }{ n! } \cdot (x - x_0)^n
$$
cioe' la serie di Taylor troncata al $k$-esimo grado sara' una buona approssimazione per $f$, e l'errore analitico sara' dato da:
$$
\sigma_{an} = R(x) = f(x) - T(x, k) 
$$
		con $R(x)$ il resto fra lo sviluppo di Taylor troncato $T(x, k)$ e la funzione stessa $f(x)$.
In questo caso fissato $k$ si potra' dare una stima di errore direttamente dall'errore di Lagrange, cioe':
$$
R(x) = f(x) - T(x, k) \leq \epsilon_l = \frac{ f^{(k + 1)} ( \eta ) }{(k + 1)!} (x - x_0)^{k + 1}
$$ # penso si faccia cosi' pero' riguarda

# esempio errore di $e^x$ al variare di $k$ con Taylor
# esempi di dettagli d'implementazione (vedi lab)

\subsection{Richiami di algebra lineare}
Nella maggior parte dei casi che ci interessano vorremo trattare non di scalari, ma di quantita' vettoriali, ad esempio $x \in \mathbb{C}^n$, con $n > 1$.

\subsubsection{Matrici complesse}
Ci saranno utili le matrici perche' rappresentano direttamente tutte le \textbf{funzioni lineari}. 
Ad esempio, posta $f: \mathbb{C}^n \rightarrow \mathbb{C}^m$ tale che:
\begin{itemize}
	\item $ f(x + y) = f(x) + f(y) $ (addittivita');
	\item $f(\lambda x) = \lambda f(x)$ (omogeneita')
\end{itemize}
detta \textit{funzione lineare} allora $\exists ! A \in \mathbb{C}^{m \times n}$ tale che $f(x) = A x$,  $\forall \in \mathbb{C}^n$.

Nel corso useremo sia matrici in $\mathbb{R}$ che matrici in $\mathbb{C}$, dove l'appartenenza a ciascuno di questi campi dipende dalla appartenenza di essi delle \textbf{entrate} $A_{ij}$ della matrice.
In ogni caso, una matrice reale non sara' che un caso particolare delle matrici complesse.

Si danno poi per scontate le definizioni di matrici:
\begin{itemize}
	\item \textit{Quadrate} ($n = m$):
	\item \textit{Rettangolari} ($n \neq m$);
	\item \textit{Diagonali} (elementi nulli fuori dalla diagonale);
	\item \textit{Triangolari superiori/inferiori} (elementi nulli sotto/sopra la diagonale) 
\end{itemize}

\subsubsection{Indipendenza lineare}
Diamo la definizione di indipendenza linare:
\begin{definition}{}
	Un insieme di vettori $\{x_1, ..., x_s\}$ si dice linearemente indipendente se:
	$$
		x_1 + ... + x_s = 0 \leftrightarrow x_1, ..., x_s = 0
	$$
\end{definition}

Inoltre, se $s = n$ l'insieme $\{ x_1, ...,x_s \}$ si dice \textbf{base} di $\mathbb{C}^n$, e cioe' $\forall y \in \mathbb{C}^m$ $\exists ! \{c1, ..., c_s\}$ tali che $y = \sum_{j=1}^n c_j x_j$.

\subsubsection{Prodotto scalare}
Diamo la definizione di prodotto scalare, generalizzato al campo complesso dal \textbf{prodotto hermitiano} (entrambi \textit{prodotti interni}):
\begin{definition}{Prodotto interno}
	Definiamo il prodotto interno fra due vettori $x, y \in \mathbb{C}^n$ come:
	$$
	<x, y> = \sum_{j = 1}^n x_j \overline{y_j}
	$$
\end{definition}
dove $\overline{y_j}$ rappresenta il \textbf{coniugato} di $y_j$, che chiaramente in $\mathbb{R}$ si riduce a $y_j$ stesso e quindi:
	$$
	<x, y> = \sum_{j = 1}^n x_j y_j
	$$ 

\subsection{Trasposta coniugata}
Definiamo infine la \textbf{trasposta coniugata} di una certa matrice, generalizzata al campo complesso dalla \textbf{matrice hermitiana}:
\begin{definition}{Trasposta coniugata}
	Data una matrice $A \in \mathbb{C}^{n \times m}$, la trasposta coniugata $A^T$ sara':
	$$
		(A^T)_{ij} = A_{ji}
	$$
	e la matrice hermitiana $A^H$ sara': 
	$$
		(A^H)_{ij} = \overline{A_{ji}}
	$$
\end{definition}

\subsubsection{Operazioni matriciali}
Date due matrici $A$ e $B$ con lo stesso numero di righe e colonne si possono definire le operazioni:
\begin{itemize}
	\item \textit{Somma} ($A, B, C \in \mathbb{C}^{m \times n}$, $A + B = C$, $c_{ij} = a_{ij} + b_{ij}$);
	\item \textit{Prodotto} ($A \in \mathbb{C}^{m \times n}, B \in \mathbb{C}^{n \times p}, C \in \mathbb{C}^{m \times n}$, $A \cdot B = C$, $c_{ij} = \sum_{h = 1}^n a_{ih} b_{hj}$ sia in reali che in complessi). 
\end{itemize}

\par\smallskip

Dal punto di vista computazionale, si ha che il prodotto scalare ha complessita' $O(n)$, e la moltiplicazione matriciale ha complessita' $O(m, n)$.

\end{document}


\documentclass[a4paper,11pt]{article}
\usepackage[a4paper, margin=8em]{geometry}

% usa i pacchetti per la scrittura in italiano
\usepackage[french,italian]{babel}
\usepackage[T1]{fontenc}
\usepackage[utf8]{inputenc}
\frenchspacing 

% usa i pacchetti per la formattazione matematica
\usepackage{amsmath, amssymb, amsthm, amsfonts}

% usa altri pacchetti
\usepackage{gensymb}
\usepackage{hyperref}
\usepackage{standalone}

% imposta il titolo
\title{Appunti Calcolo Numerico}
\author{Luca Seggiani}
\date{2025}

% disegni
\usepackage{pgfplots}
\pgfplotsset{width=10cm,compat=1.9}

% imposta lo stile
% usa helvetica
\usepackage[scaled]{helvet}
% usa palatino
\usepackage{palatino}
% usa un font monospazio guardabile
\usepackage{lmodern}

\renewcommand{\rmdefault}{ppl}
\renewcommand{\sfdefault}{phv}
\renewcommand{\ttdefault}{lmtt}

% disponi il titolo
\makeatletter
\renewcommand{\maketitle} {
	\begin{center} 
		\begin{minipage}[t]{.8\textwidth}
			\textsf{\huge\bfseries \@title} 
		\end{minipage}%
		\begin{minipage}[t]{.2\textwidth}
			\raggedleft \vspace{-1.65em}
			\textsf{\small \@author} \vfill
			\textsf{\small \@date}
		\end{minipage}
		\par
	\end{center}

	\thispagestyle{empty}
	\pagestyle{fancy}
}
\makeatother

% disponi teoremi
\usepackage{tcolorbox}
\newtcolorbox[auto counter, number within=section]{theorem}[2][]{%
	colback=blue!10, 
	colframe=blue!40!black, 
	sharp corners=northwest,
	fonttitle=\sffamily\bfseries, 
	title=Teorema~\thetcbcounter: #2, 
	#1
}

% disponi definizioni
\newtcolorbox[auto counter, number within=section]{definition}[2][]{%
	colback=red!10,
	colframe=red!40!black,
	sharp corners=northwest,
	fonttitle=\sffamily\bfseries,
	title=Definizione~\thetcbcounter: #2,
	#1
}

% disponi problemi
\newtcolorbox[auto counter, number within=section]{problem}[2][]{%
	colback=green!10,
	colframe=green!40!black,
	sharp corners=northwest,
	fonttitle=\sffamily\bfseries,
	title=Problema~\thetcbcounter: #2,
	#1
}

% disponi codice
\usepackage{listings}
\usepackage[table]{xcolor}

\lstdefinestyle{codestyle}{
		backgroundcolor=\color{black!5}, 
		commentstyle=\color{codegreen},
		keywordstyle=\bfseries\color{magenta},
		numberstyle=\sffamily\tiny\color{black!60},
		stringstyle=\color{green!50!black},
		basicstyle=\ttfamily\footnotesize,
		breakatwhitespace=false,         
		breaklines=true,                 
		captionpos=b,                    
		keepspaces=true,                 
		numbers=left,                    
		numbersep=5pt,                  
		showspaces=false,                
		showstringspaces=false,
		showtabs=false,                  
		tabsize=2
}

\lstdefinestyle{shellstyle}{
		backgroundcolor=\color{black!5}, 
		basicstyle=\ttfamily\footnotesize\color{black}, 
		commentstyle=\color{black}, 
		keywordstyle=\color{black},
		numberstyle=\color{black!5},
		stringstyle=\color{black}, 
		showspaces=false,
		showstringspaces=false, 
		showtabs=false, 
		tabsize=2, 
		numbers=none, 
		breaklines=true
}

\lstdefinelanguage{javascript}{
	keywords={typeof, new, true, false, catch, function, return, null, catch, switch, var, if, in, while, do, else, case, break},
	keywordstyle=\color{blue}\bfseries,
	ndkeywords={class, export, boolean, throw, implements, import, this},
	ndkeywordstyle=\color{darkgray}\bfseries,
	identifierstyle=\color{black},
	sensitive=false,
	comment=[l]{//},
	morecomment=[s]{/*}{*/},
	commentstyle=\color{purple}\ttfamily,
	stringstyle=\color{red}\ttfamily,
	morestring=[b]',
	morestring=[b]"
}

% disponi sezioni
\usepackage{titlesec}

\titleformat{\section}
	{\sffamily\Large\bfseries} 
	{\thesection}{1em}{} 
\titleformat{\subsection}
	{\sffamily\large\bfseries}   
	{\thesubsection}{1em}{} 
\titleformat{\subsubsection}
	{\sffamily\normalsize\bfseries} 
	{\thesubsubsection}{1em}{}

% disponi alberi
\usepackage{forest}

\forestset{
	rectstyle/.style={
		for tree={rectangle,draw,font=\large\sffamily}
	},
	roundstyle/.style={
		for tree={circle,draw,font=\large}
	}
}

% disponi algoritmi
\usepackage{algorithm}
\usepackage{algorithmic}
\makeatletter
\renewcommand{\ALG@name}{Algoritmo}
\makeatother

% disponi numeri di pagina
\usepackage{fancyhdr}
\fancyhf{} 
\fancyfoot[L]{\sffamily{\thepage}}

\makeatletter
\fancyhead[L]{\raisebox{1ex}[0pt][0pt]{\sffamily{\@title \ \@date}}} 
\fancyhead[R]{\raisebox{1ex}[0pt][0pt]{\sffamily{\@author}}}
\makeatother

\begin{document}

% sezione (data)
\section{Lezione del 07-03-25}

% stili pagina
\thispagestyle{empty}
\pagestyle{fancy}

% testo
Proseguiamo i richiami di algebra lineare.

\subsubsection{Considerazioni sull'efficienza della moltiplicazione matriciale}
# sul suo lab

\subsubsection{Proprietà della moltiplicazione matriciale}
Abbiamo che in genere il prodotto fra matrici non è \textbf{commutativo}, cioè:
$$
A B \neq BA
$$
di contro, vale l'\textbf{associativa}:
$$
(A \cdot B) \cdot C = A \cdot (B\cdot C)
$$
e la \textbf{distributiva}, separatamente ai due lati:
$$
(A + B) \cdot C = A \cdot C + B \cdot C
$$
$$
C \cdot (A + B) = C \cdot A + C \cdot B 
$$

Inoltre, notiamo che quello delle matrici non è un \textit{dominio integrale}: 
$$
A \cdot B = 0 \not\implies A = 0 \vee B = 0
$$

Vediamo ad esempio come sfruttare la proprietà associativa può permetterci di ottenere algoritmi più veloci.
Supponiamo di voler calcolare:
$$
(A \cdot B) \cdot C = A \cdot (B\cdot C)
$$
con $A \in \mathbb{C}^{m \times n}$; $B \in \mathbb{C}^{n \times p}$ , $C \in \mathbb{C}^{p \times q}$.

L'uguaglianza ci darà due metodi:
\begin{itemize}
	\item $(A \cdot B) \cdot C$: 
		# calcolo me lo sono perso
	\item $A \cdot (B\cdot C)$:
\end{itemize}

In generale, quindi, se si hanno matrici con poche righe o poche colonne, è opportuno cercare di mantenere questa properietà più a lungo possibile nei risultati intermedi.

\par\smallskip

Un altra proprietà del prodotto di matrici e che si può vedere il risultato riga per riga o colonna per colonna nei seguenti modi:
$$
A = \begin{pmatrix}
A_1 \\
... \\ 
A_m
\end{pmatrix}, \quad
B = \begin{pmatrix}
	B_1 & ... & B_p
\end{pmatrix}
$$
$$
\implies
A \cdot B = \begin{pmatrix}
	A \cdot B_1 & ... & A \cdot B_p
\end{pmatrix} = \begin{pmatrix}
A_1 B \\ ... \\ A_m B
\end{pmatrix}
$$
Questo ci dice che le colonne (delle righe) di $C = A \times B$ sono combinazioni lineari delle colonne (delle righe) di $A$ (di $B$).

# esercizio su distributiva stima costi asintotici

\subsubsection{Determinante}
Abbiamo visto la definizione di \textbf{determinante} per matrici quadrate:
\begin{definition}{Determinante}
	Si definisce la funzione:
	$$
		\mathrm{det}(A) : \mathbb{C}^{n \times n} \rightarrow \mathbb{C}
	$$
	con:
	$$
		\mathrm{det}(A) =
			\begin{cases}
				a_{11}, \quad n = 1 \\ 
				a_11 a_22 - a_12 a_21, \quad n =2 \\ 
				\sum_{j = 1}^n (-1)^{i + j} \mathrm{det}(A_{ij}), \quad n > 2 \quad \text{(sviluppo di Laplace)}
			\end{cases}
	$$
	determinante.
\end{definition}

Osserviamo che il calcolo del determinante attraverso lo sviluppo di Laplace ha complessità algoritmica $O(n!)$.

\subsubsection{Proprietà del determinante}
Sappiamo che $A$ invertibile $\Leftrightarrow \mathrm{det}(A)$ (cioè $A$ \textbf{singolare}).
Inoltre valgono:
$$
\mathrm{det}(A^T) = \mathrm{det}(A)
$$
$$
\mathrm{det}(A^H) = \overline{\mathrm{det}(A)}
$$

\subsubsection{Rango}
Vediamo poi come ci può servire il determinante delle \textbf{sottomatrici}:
\begin{definition}{Sottomatrice}
	Una sottomatrice di $A$ è una matrice ottenuta prendendo la restrizione di $A$ a un sottoinsieme di righe e di colonne, cioè data $A \in \mathbb{C}^{n \times n}$, $I, J \subseteq \{ 1, ..., n \}$ con $|I| = n_1$ e $|J| = n_2$, sarà allora:
	$$
		A(I, J) \in \mathbb{C}^{n_1 \times n_2}
	$$
	ottenuta incrociando le righe in $I$ con le colonne in $J$.
\end{definition}

Definiamo quindi i \textbf{minori}:
\begin{definition}{Minore}
	Si dice minore di ordine $k$, con $k \in \{ 1, ..., n \}$ il determinante di una sottomatrice quadrata $k \times k$.
\end{definition}

E le sottomatrici \textbf{principali} e \textbf{principali di testa}:
\begin{definition}{Sottomatrice principale}
	Una sottomatrice si dice principale se gli insiemi $I, J$ usati per estrarla sono $I = J$.
\end{definition}
\begin{definition}{Sottomatrice principale di testa}
	Una sottomatrice quadrata di ordine $k$ si dice sottomatrice principale di testa se $I = J = \{ 1, ..., k \}$.
\end{definition}
Allo stesso modo si possono definire \textbf{matrici di coda} (basterà prendere indici da $k$ ad $n$).

Possiamo quindi definire il \textbf{rango} di matrice:
\begin{definition}{Rango}
	Il rango $\mathrm{rank}(A)$ di una matrice $A$ è definito come il massimo numero di colonne (o di righe) linearmente indipendenti, ed è uguale all'ordine massimo dei minori $\neq 0$ in $A$.
\end{definition}

\subsubsection{Proprietà del rango}
Caso particolare del rango sarà chiaramente quello dove prendiamo come sottomatrice la matrice stessa: $\mathrm{det}(A) \neq 0 \Leftrightarrow \mathrm{rank}(A) = n$, e quindi gli insiemi delle colonne e delle righe di $A$ sono tutte linearmente indipendenti.

Si ha poi che:
$$
\mathrm{rank}(A) = \mathrm{dim}(\mathrm{Im}(A))
$$
dove $\mathrm{Im}(A)$ è la dimensione dell'\textbf{immagine} di $A$:
$$
\mathrm{Im}(A) = \left\{ y \in \mathbb{C}^m : y = Ax, \ x \in \mathbb{C}^n \right\}
$$

\subsubsection{Teorema di Binet-Cauchy}
Concludiamo dimostrando il teorema di Binet-Cauchy sul determinante rispetto al prodotto matriciale.
\begin{theorem}{di Binet-Cauchy}
	Prese due matrici, $A \in \mathbb{C}^{n \times n}$ e $B \in \mathbb{C}^{n \times n}$, e $C = A \cdot B$ di dimensioni uguali, sarà:
	$$
	\mathrm{det}(C) = \mathrm{det}(A) \cdot \mathrm{det}(B)
	$$
Nel caso generale avremo $A \in \mathbb{C}^{n \times p}$ e $B \in \mathbb{C}^{p \times n}$, da cui:
$$
\mathrm{det}(C) =
	\begin{cases}
		0, \quad p \leq n \\
		\sum_j A_{[j]} \cdot B_{[j]}
	\end{cases}
$$
dove $A_{[j]}$ e $B_{[j]}$ sono i minori di ordine $n$ relativi ala stessa scelta di indici in $A$ e in $B$.
\end{theorem}

\subsubsection{Matrice inversa}
Diamo la definizione di \textbf{matrice inversa}:
\begin{definition}{Matrice inversa}
	Data $A \in \mathbb{C}^{n \times n}$ tale che $\mathrm{det}(A) \neq 0$, si definisce $A^{-1} \in \mathbb{C}^{n \times n}$ tale che:
	$$
	A \cdot A^{-1} = A^{-1} \cdot A = I
	$$
\end{definition}
Se guardo ad $A$ come una funzione lineare $A : \mathbb{C}^n \rightarrow \mathbb{C}^n$, sarà che:
$$
A : x \rightarrow Ax, \quad A^{-1} : Ax \rightarrow x
$$
questo da:
$$
\left(A^{-1} \circ A\right) = A^{-1} y = A^{-1} A x = Ix = x
$$

# porta questi titoletti come si deve
\subsubsection{Proprietà della matrice inversa}
Vediamo alcune proprietà di $A^{-1}$:
\begin{enumerate}
	\item Ricordiamo di nuovo che $A^{-1}$ esiste se e solo se $\mathrm{det}(A) \neq 0$, cioè equivalentemente se $\mathrm{rank}(A) = n$, o $A$ ha spazi riga e colonna linearmente indipendenti;

		\item
Vediamo poi il calcolo di $\mathrm{det}(A^{-1})$: da $\mathrm{det}(I) = 1$, e:
$$
\det(A \cdot A^{-1}) = \det(A) \cdot \det(A^{-1}), \quad \text{(Binet)}
$$
$$
\implies \det(A^{-1}) = \frac{1}{\det(A)}
$$

\item Vediamo poi che:
$$
(A \cdot B)^{-1} = B^{-1} A^{-1}
$$
per matrici quadrate $A, B \in \mathbb{C}^{n \times n}$;

\item Riguardo alle trasposte e alle Hermitiane vale:
	$$
		(A^T)^{-1} = (A^{-1})^T = A^{-T}
	$$
	$$
		(A^H)^{-1} = (A^{-1})^H = A^{-H}
	$$
	e:
	$$
	(AB)^T = B^T A^T
	$$
	$$
	(AB)^H = B^H A^H
	$$
\end{enumerate}

\subsubsection{Matrici particolari}
Definiamo alcune matrici particolari:
\begin{definition}{Matrici particolari}
	Data $A \in \mathbb{C}^{n \times n}$, si dice di $A$ che è:
	\begin{itemize}
		\item \textbf{Hermitiana:} $A = A^H$; 
		\item \textbf{Antiermitiana:} $A = -A^H$;
		\item \textbf{Unitaria} $A^H A = A A^H = I$, $A^{-1} = A^H$;
		\item \textbf{Normale} $A^H A = A A^H$;
		\item \textbf{Simmetrica} $A = A^T$;
		\item \textbf{Antisimmetrica} $A = -A^T$;
		\item \textbf{Ortogonale} $A^TA = AA^T = I$, $A^T = A^{-1}$
	\end{itemize}
\end{definition}
dove notiamo che \textbf{simmetrica} e \textbf{antisimmetrica} significano \textit{hermitiana} e \textit{antihermitiana} in \mathbb{R}, e \textbf{ortogonale} significa \textit{unitaria} in \mathbb{R}.

Per di più vale:
$$
\{ \text{matrici simmetriche reali} \} \subseteq \{ \text{matrici hermitiane} \} \subseteq \text{matrici normali}
$$
e:
$$\{ \text{matrici ortogonali} \} \subseteq \{ \text{matrici unitarie} \}
$$
dove notiamo che unitarie ed ortogonali hanno l'inversa "facile", nel senso che basta trasporre o trovare l'hermitiana.

\subsubsection{Matrici di permutazione}
Definiamo le \textbf{matrici di permutazione}:
\begin{definition}{Matrice di permutazione}
	$A \in \mathbb{R}^{n \times n}$ si dice matrice di permutaione se $A$ si ottiene dalla matrice di identità permutandone righe e colonne.
\end{definition}

\subsubsection{Proprietà delle matrici di permutazione}
Tutte le matrici di permutazione sono matrici ortogonali (questo si vede dalla loro unimodularità, o osservando che $P^T$ si ottiene dalla permutazione inversa).

Inoltre, vediamo che il prodotto di $A$ con una matrice di permutazione si limita a scambiare righe e colonne in accordo con la permutazione della matrice.
In particolare, $A \cdot P$ dà la permutazione delle colonne, mentre $P \cdot A$ dà la permutazione delle righe.

\subsubsection{Sistemi lineari}
Abbiamo un sistema di $m$ equazioni lineari in $n$ incognite:
\[
	\begin{cases}
		a_{11} x_1 + ... a_{1n}	x_n = b_1 \\ 
		... \\
		a_{m1} x_1 + ... a_{mn}	x_n = b_m
	\end{cases}
\]

Ci è spesso utile riscrivere sistemi di questo tipo in forma matriciale:
$$
A = \begin{pmatrix}
	a_{11} & ... & a_{1n} \\ 
	... & ... & ... \\
	a_{m1} & ... & a_{mn}
\end{pmatrix} \in \mathbb{C}^{m \times n}, \quad
x = \begin{pmatrix}
	x_1 \\ ...\\ x_n
\end{pmatrix} \in \mathbb{C}^{n}, \quad
b = \begin{pmatrix}
	b1 \\ ... \\ b_m
\end{pmatrix} \in \mathbb{C}^{m}
$$

Il problema principe sarà quello, date $A$ e $b$, di trovare $x$.

\subsubsection{Teorema di Rouché-Capelli}
Richiamiamo il teorema:
\begin{theorem}{di Rouché-Capelli}
	$Ax = b$ ammette almeno una soluzione se $\mathrm{rank}(A) = \mathrm{rank}(A | b)$, con $(A|b) \in \mathbb{C}^{m \times (n + 1)}$ la matrice ottenuta aumentando $A$ con $b$.

	Per quanto riguarda l'unicità, supposto $m \geq n$ (sistema \textit{sovradeterminato}):
	\begin{itemize}
		\item Se $\mathrm{rank}(A) = n$ allora la soluzione è unica;
		\item Se $\mathrm{rank}(A) < n$ allora ci sono $\infty$ soluzioni, e l'insieme delle soluzioni forma uno spazio vettoriale affine di dimensinoe $n - \mathrm{rank}(A)$;
	\end{itemize}
\end{theorem}

Nel caso il vettore $b = 0$, il sistema si dice \textbf{omogeneo}, e la soluzione nulla esiste sempre.

Abbiamo poi che:
\begin{definition}{}
	L'insieme delle soluzioni di $Ax = 0$ si chiama Kernel (nucleo) della matrice:
	$$
		\mathrm{ker}(A) = \left\{ x \in \mathbb{R}^n : Ax = 0 \right\}
	$$
\end{definition}
notiamo che il kernel è un sottospazio vettoriale di dimensione $n - \mathrm{rank}(A)$.

Osserviamo infine che nel caso $m = n$, se $\det(A) \neq 0$, cioè $\mathrm{rank}(A) = n$, òa soluzione di $Ax = b$ è unica per ogni $b \in \mathbb{R}^m$ e si scrive:
$$
x = A^{-1} b
$$

Vedremo che in generale calcolare l'inversa non è conveniente rispetto ad altri metodi di approssimazione.

\subsubsection{Regola di Cramer}
Il vettore soluzione si può scrivere ache componente per componente come:
$$
x_j = \frac{\det(A_j)}{\det(A)}
$$
dove $A_j$ è la matrice ottenuta da $A$ sotituendo la colonna $j$ con $b$.

Questa via costa come calcolare $O(n)$ determinanti di matrici $n \times n$, ed è quindi poco conveniente ($O(n^3)$). 

\end{document}

\end{document}