
\documentclass[a4paper,11pt]{article}
\usepackage[a4paper, margin=8em]{geometry}

% usa i pacchetti per la scrittura in italiano
\usepackage[french,italian]{babel}
\usepackage[T1]{fontenc}
\usepackage[utf8]{inputenc}
\frenchspacing 

% usa i pacchetti per la formattazione matematica
\usepackage{amsmath, amssymb, amsthm, amsfonts}

% usa altri pacchetti
\usepackage{gensymb}
\usepackage{hyperref}
\usepackage{standalone}

% imposta il titolo
\title{Appunti Calcolo Numerico}
\author{Luca Seggiani}
\date{2025}

% disegni
\usepackage{pgfplots}
\pgfplotsset{width=10cm,compat=1.9}

% imposta lo stile
% usa helvetica
\usepackage[scaled]{helvet}
% usa palatino
\usepackage{palatino}
% usa un font monospazio guardabile
\usepackage{lmodern}

% tikz in sans
\tikzset{every picture/.style={/utils/exec={\sffamily}}}

\renewcommand{\rmdefault}{ppl}
\renewcommand{\sfdefault}{phv}
\renewcommand{\ttdefault}{lmtt}

% circuiti
\usepackage{circuitikz}
\usetikzlibrary{babel}

% disponi il titolo
\makeatletter
\renewcommand{\maketitle} {
	\begin{center} 
		\begin{minipage}[t]{.8\textwidth}
			\textsf{\huge\bfseries \@title} 
		\end{minipage}%
		\begin{minipage}[t]{.2\textwidth}
			\raggedleft \vspace{-1.65em}
			\textsf{\small \@author} \vfill
			\textsf{\small \@date}
		\end{minipage}
		\par
	\end{center}

	\thispagestyle{empty}
	\pagestyle{fancy}
}
\makeatother

% disponi teoremi
\usepackage{tcolorbox}
\newtcolorbox[auto counter, number within=section]{theorem}[2][]{%
	colback=blue!10, 
	colframe=blue!40!black, 
	sharp corners=northwest,
	fonttitle=\sffamily\bfseries, 
	title=Teorema~\thetcbcounter: #2, 
	#1
}

% disponi definizioni
\newtcolorbox[auto counter, number within=section]{definition}[2][]{%
	colback=red!10,
	colframe=red!40!black,
	sharp corners=northwest,
	fonttitle=\sffamily\bfseries,
	title=Definizione~\thetcbcounter: #2,
	#1
}

% disponi problemi
\newtcolorbox[auto counter, number within=section]{problem}[2][]{%
	colback=green!10,
	colframe=green!40!black,
	sharp corners=northwest,
	fonttitle=\sffamily\bfseries,
	title=Problema~\thetcbcounter: #2,
	#1
}

% disponi codice
\usepackage{listings}
\usepackage[table]{xcolor}

\definecolor{codegreen}{rgb}{0,0.6,0}
\definecolor{codegray}{rgb}{0.5,0.5,0.5}
\definecolor{codepurple}{rgb}{0.58,0,0.82}
\definecolor{backcolour}{rgb}{0.95,0.95,0.92}

\lstdefinestyle{codestyle}{
	backgroundcolor=\color{black!5}, 
	commentstyle=\color{codegreen},
	keywordstyle=\bfseries\color{magenta},
	numberstyle=\sffamily\tiny\color{black!60},
	stringstyle=\color{green!50!black},
	basicstyle=\ttfamily\footnotesize,
	breakatwhitespace=false,         
	breaklines=true,                 
	captionpos=b,                    
	keepspaces=true,                 
	numbers=left,                    
	numbersep=5pt,                  
	showspaces=false,                
	showstringspaces=false,
	showtabs=false,                  
	tabsize=2
}

\lstdefinestyle{shellstyle}{
	backgroundcolor=\color{black!5}, 
	basicstyle=\ttfamily\footnotesize\color{black}, 
	commentstyle=\color{black}, 
	keywordstyle=\color{black},
	numberstyle=\color{black!5},
	stringstyle=\color{black}, 
	showspaces=false,
	showstringspaces=false, 
	showtabs=false, 
	tabsize=2, 
	numbers=none, 
	breaklines=true
}

\lstdefinelanguage{javascript}{
	keywords={typeof, new, true, false, catch, function, return, null, catch, switch, var, if, in, while, do, else, case, break},
	keywordstyle=\color{blue}\bfseries,
	ndkeywords={class, export, boolean, throw, implements, import, this},
	ndkeywordstyle=\color{darkgray}\bfseries,
	identifierstyle=\color{black},
	sensitive=false,
	comment=[l]{//},
	morecomment=[s]{/*}{*/},
	commentstyle=\color{purple}\ttfamily,
	stringstyle=\color{red}\ttfamily,
	morestring=[b]',
	morestring=[b]"
}

% disponi sezioni
\usepackage{titlesec}

\titleformat{\section}
{\sffamily\Large\bfseries} 
{\thesection}{1em}{} 
\titleformat{\subsection}
{\sffamily\large\bfseries}   
{\thesubsection}{1em}{} 
\titleformat{\subsubsection}
{\sffamily\normalsize\bfseries} 
{\thesubsubsection}{1em}{}

% disponi alberi
\usepackage{forest}

\forestset{
	rectstyle/.style={
		for tree={rectangle,draw,font=\large\sffamily}
	},
	roundstyle/.style={
		for tree={circle,draw,font=\large}
	}
}

% disponi algoritmi
\usepackage{algorithm}
\usepackage{algorithmic}
\makeatletter
\renewcommand{\ALG@name}{Algoritmo}
\makeatother

% disponi numeri di pagina
\usepackage{fancyhdr}
\fancyhf{} 
\fancyfoot[L]{\sffamily{\thepage}}

\makeatletter
\fancyhead[L]{\raisebox{1ex}[0pt][0pt]{\sffamily{\@title \ \@date}}} 
\fancyhead[R]{\raisebox{1ex}[0pt][0pt]{\sffamily{\@author}}}
\makeatother

\begin{document}

% sezione (data)
\section{Lezione del 12-05-25}

% stili pagina
\thispagestyle{empty}
\pagestyle{fancy}

% testo
Vediamo alcuni esempi sui medi stazionari ad un punto.

\subsubsection{Esempio: convergenza locale 1}
Prendiamo la funzione:
$$
f(x) = x \log(x) - 1
$$

Innanzitutto dimostriamo che questa ha un'unica radice nell'intervallo $[0, +\infty)$, e poi studiamo la convergenza locale dei seguenti metodi:
\begin{enumerate}
	\item L'espressione esplicita di $x$:
		$$
		\phi_1(x) = \frac{1}{\log(x)}
		$$
	\item L'espressione analoga, ma meno immediata (prendiamo la $x$ da dentro il logaritmo):
		$$
		\log(x) = \frac{1}{x} \implies \phi_2(x) = e^{\frac{1}{x}}
		$$
	\item Ciò che troveremmo applicando la formula usuale:
		$$
		\phi(x) = x - g(x) f(x) \implies \phi_3(x) = \frac{x + 1}{\log(x) + 1}
		$$
		dove $g(x)$ è:
		$$
		g(x) = \frac{1}{\log(x) + 1} 
		$$
		in quanto:
		$$
		\phi_3(x) = x - g(x) f(x) = x - \frac{x \log(x) + 1}{\log(x) + 1} = \frac{x \log(x) + 1 - x \log(x) - 1}{\log(x) + 1} = \frac{x + 1}{\log(x) + 1}
		$$
\end{enumerate}

Vediamo quindi di dimostrare l'unicità della soluzione.
Abbiamo che la derivata prima $f'(x)$ è:
$$
f'(x) = \log(x) + 1 \implies 
\begin{cases}
	f'(x) < 0, \quad x < \frac{1}{e} \\			
	f'(x) > 0, \quad x > \frac{1}{e} \\			
\end{cases}
$$
e si ha un minimo in $\frac{1}{e}$.
Vediamo quindi il limite in $0^+$:
$$
\lim_{x\rightarrow 0} x\ log(x) - 1 = -1
$$

Cioè la funzione parte da $-1$ in $x = 0$, scende fino al minimo $\frac{1}{e}$ e poi sale, incontrando l'asse delle ascisse una volta sola, nel punto $\alpha$ a cui siamo interessati.

Vorremo avere un punto $> \alpha$ per circoscrivere la regione in cui questo si trova.
Per semplificare i calcoli, proviamo $x = e$, da cui:
$$
f(e) = e \log(e) - 1 = e - 1 > 0
$$
da cui individuiamo la regione:
$$
\alpha \in \left[\frac{1}{e}, e\right]
$$

Verifichiamo quindi le convergenze dei vari metodi proposti.
\begin{enumerate}
	\item Prendiamo la derivata:
		$$
		\phi_1'(x) = \left( \frac{1}{\log(x)} \right)' = \frac{1}{x} \left( - \frac{1}{\log^2(x)} \right) = - \frac{1}{x \log^2(x)}
		$$
		e valutiamone il modulo in $\alpha$:
		$$
		|\phi_1'(\alpha)| = \left| \frac{1}{\alpha \log^2(\alpha)} \right| = \left| \frac{1}{\log(\alpha)} \right| = |\alpha| \in \left[\frac{1}{e}, e\right]
		$$
		sfruttando $\alpha \log(\alpha) = 1$.

		Abbiamo quindi che $\alpha$ è potenzialmente grande fino a $e$, quindi non convergente.
		Per convincercene, prendiamo il punto 1:
		$$
		f(1) = \log(1) - 1 = -1
		$$
		da cui sicuramente, in quanto $f(\alpha)$ sarà $0 > -1$ e quindi $\alpha > 1$:
		$$
		|\phi_1'(\alpha)| > 1
		$$
		e il metodo non converge.

	\item Prendiamo la derivata:
		$$
		\phi_2'(x) = \left( e^{\frac{1}{x}} \right)' = -\frac{1}{x^2} e^{\frac{1}{x}} 
		$$
		e valutiamone il modulo in $\alpha$:
		$$
		|\phi_2'(\alpha)| = \left| -\frac{\alpha}{\alpha^2} \right| = -\frac{1}{\alpha}
		$$
		sfruttando $\alpha = e^{\frac{1}{\alpha}}$.
		
		Da questo abbiamo che:
		$$
		|\phi_2'(\alpha)| < 1
		$$
		in quanto $\alpha \in [1, e]$, e quindi il metodo converge localmente, in modo lineare ($\phi_2'(\alpha) \neq 0$).

	\item Prendiamo la derivata:
		$$
		\phi_3'(x) = \left( \frac{x + 1}{\log(x) + 1} \right)' = \frac{( \log(x) + 1) - (x + 1) \frac{1}{x}}{(\log(x) + 1)^2} = \frac{\log(x) - \frac{1}{x}}{(\log(x) + 1)^2}
		$$
		$$
		= \frac{x \log(x) - 1}{x (\log(x) + 1)^2} = \frac{f(x)}{x (\log(x) + 1)^2}
		$$
		da cui si ha $f(x)$ al denominatore, e quindi sicuramente:
		$$
		\phi_3'(\alpha) = 0
		$$
		e si converge con ordine maggiore o uguale al lineare.
		
		Vediamo la derivata seconda per confermare l'ordine di convergenza:
		$$
		\phi_3''(x) = \frac{ (\log(x) + 1)^3 x - (x \log(x) - 1) \left[ (\log(x) + 1)^2 + \frac{2(\log(x) + 1)}{x} \right] }{ x^2 (\log(x) + 1)^4 } 
		$$
		che sostituendo $\alpha$ diventa:
		$$
		\phi_3''(\alpha) = \frac{\alpha (\log(\alpha) + 1)^3}{\alpha^2 (\log(\alpha + 1)^4)} = \frac{1}{\alpha (\log(\alpha) + 1)} \neq 0
		$$
		cioè l'ordine di convergenza è 2.
\end{enumerate}

\subsubsection{Esempio: convergenza locale 2}
Prendiamo adesso un esempio dove ci è data la funzione di punto fisso: 
$$
\phi(x) = 1 + a \log(x) + b \log^2(x), \quad a, b \in \mathbb{R}
$$
con punto fisso $\alpha = 1$.
Siamo interessati a valutarne l'ordine di convergenza al variare dei parametri $a$ e $b$, e determinare quando la convergenza è lineare e monotona.
Prendiamo quindi la derivata:
$$
\phi'(x) = \frac{a}{x} + 2b \frac{\log(x)}{x} \implies \phi'(1) = a
$$
e il metodo converge localmente per $a \in (-1, 1)$.
In particolare, il metodo converge in modo monotono (localmente) quando $a \in (0, 1)$, e per $a = 0$ converge in modo superlineare.

Valutiamo quindi l'ordine di convergenza per $a = 0$:
$$
\phi(x) = 1 + b \log^2(x)
$$
da cui le derivate:
$$
\phi'(x) = 2b \frac{\log(x)}{x} \implies \phi'(\alpha) = 0
$$
$$
\phi''(x) = \frac{2b}{x^2} - 2b \frac{\log(x)}{x^2} = 2b \frac{(1 - \log(x))}{x^2} \implies \phi''(1) = 2b
$$
cioè per $b \neq 0$ la convergenza è quadratica, e per $b = 0$ la convergenza è in un passo ($x_1 = \alpha = 1$ per ogni $x_0$).

Nel caso $a = 0$ per studiare la convergenza monotona dobbiamo studiare il segno di $\phi'(x)$ vicino ad $\alpha = 1$.
\begin{itemize}
	\item[$b > 0$:] vale $\phi'(x) > 0$ su $(1, 1 + \rho)$ con $\rho$ sufficientemente piccolo, quindi si ha convergenza monotona con $x_0 \in (1, 1 + \rho)$.
	\item[$b < 0$:] vale $\phi'(x) > 0$ su $(1 - \rho, 1)$ con $\rho$ sufficientemente piccolo, quindi si ha convergenza monotona con $x_0 \in (1 - \rho, 1)$.
\end{itemize}

\subsection{Metodo di Newton}
Vediamo un particolare metodo stazionario ad un punto, cioè il \textbf{metodo di Newton}:
$$
\phi(x) = x - \frac{f(x)}{f'(x)}
$$
e quindi il passo iterativo è:
\[
	\begin{cases}
		x_0, \quad \text{dato}	\\
		x_{n + 1} = \phi(x_n) =  x_n - \frac{f(x_n)}{f'(x_n)}
	\end{cases}
\]

L'interpretazione geometrica del metodo di Newton è quella classica, cioè si prende la retta tangente che si stacca da $x_i$ e si chiama $x_{i + 1}$ l'intersezione di questa con l'asse delle ascisse. 
Per questo viene detto anche \textit{metodo delle tangenti}.

Il metodo di Newton è solitamente molto veloce, ma può presentare comporamenti non ottimali nel caso di funzioni particolarmente \textit{"schiacciate"}, cioè con derivata $|f'(x)| << 1$ vicino ad $\alpha$, fra cui ad esempio le polinomiali $x^n$ con $n >> 1$.

Per valutare questi punti sfavorevoli diamo la definizione di \textbf{radice di ordine} $\mathbf{r}$:
\begin{definition}{Radice di ordine $\mathbf{r}$}
	Sia $f : [a, b] \rightarrow \mathbb{R}$ continua e $\alpha : f(\alpha) = 0$.
	In questo caso $\alpha$ è detta radice (o zero) di ordine $r > 0$ se vale:
	$$
		\lim_{x \rightarrow \alpha} \frac{f(x)}{(x - \alpha)^r} = c < + \infty \ (c \neq 0)
	$$
\end{definition}
Quindi, se $f(x) \in C^r([a, b])$, allora $\alpha$ è radice di ordine $r$ se:
$$
f(\alpha) = f'(\alpha) = ... = f^{(r - 1)}(\alpha) = 0, \quad f^{(r)} (\alpha) \neq 0
$$
In particolare, se $r = 1$, la radice si dice \textbf{semplice}.

Riguardo al teorema di Newton, vale quindi il teorema:
\begin{theorem}{Convergenza del metodo di Newton}
	Se $\alpha$ è radice semplice, e $f \in C^2([a, b])$, allora il metodo converge localmente con ordine $p \geq 2$.
\end{theorem}
In particolare varrà:
$$
	\lim_{n \rightarrow +\infty} \frac{|x_{n + 1} - \alpha|}{|x_n - \alpha|^2} = \frac{1}{2} \frac{f''(\alpha)}{f'(\alpha)}
$$
per cui:
\[
	\begin{cases}
		f''(\alpha) = 0 \implies p > 2 \\
		f''(\alpha) \neq 0 \implies p = 2
	\end{cases}
\]

Vediamo di dimostrare il teorema.
Inizieremo prendendo la funzione di punto fisso:
$$
\phi(x) = x - \frac{f(x)}{f'(x)} \implies \phi'(x) = 1 - \frac{(f'(x))^2 - f (x) f''(x)}{(f'(x))^2} = \frac{f(x) f''(x)}{(f'(x))^2}
$$
che sostituendo $\alpha$ diventa:
$$
\phi'(\alpha) = \frac{f(\alpha) f''(\alpha)}{(f'(\alpha))^2} = 0
$$
per cui la convergenza è almeno quadratica.

Continuando si ha:
$$
\phi''(x) = \frac{( f'(x) f''(x) + f(x) f'''(x) ) (f'(x))^2 - f(x) f''(x) ( 2f'(x) f''(x) ) }{(f'(x))^4}
$$
che sostituendo $\alpha$ diventa:
$$
\phi''(\alpha) = \frac{f'(\alpha) f''(\alpha) \left( f'(\alpha) \right)^2 }{ ( f'(\alpha) )^4 } = \frac{f''(\alpha)}{f'(\alpha)}
$$
da cui:
\[
	\begin{cases}
		f''(\alpha) \neq 0 \implies p = 2 \\
		f''(\alpha) = 0 \implies p > 2
	\end{cases}
\]

In particolare, si può prendere:
$$
\frac{|x_{n + 1} - \alpha|}{|x_n - \alpha|^2} = \frac{|\phi(x_n) - \phi(\alpha)|}{|x_n - \alpha|^2} 
= \frac{ | \phi(\alpha) + \phi'(\alpha) (x_n - \alpha) + \frac{1}{2} \phi''(\varepsilon) (x_n - \alpha)^2 - \phi(\alpha) | }{ |x_n - \alpha|^2 }
$$
sfruttando lo sviluppo di Taylor di $\phi(x_n)$ attorno a $\alpha$ con errore di Laplace:
$$
\phi(x_n) = \pphi(\alpha) + \phi'(\alpha) (x_n - \alpha) + \frac{1}{2} \phi''(\alpha) (x_n - \alpha)^2
$$
da cui:
$$
= \frac{ | \phi'(\alpha) (x_n - \alpha) + \frac{1}{2} \phi''(\varepsilon) (x_n - \alpha)^2 | }{ |x_n - \alpha|^2 }
= \frac{1}{2} \phi''(\varepsilon) \xrightarrow{n \rightarrow +\infty} \frac{1}{2} \phi''(\alpha) = \frac{1}{2} \frac{f''(\alpha)}{f'(\alpha)}
$$
con $n \rightarrow +\infty$, noto $\varepsilon \in [x_n, \alpha]$, per cui si ha l'ultima tesi del teorema. \qed

Osserviamo che nel metodo di Newton si può guardare direttamente a $f(x)$ e alle sue derivate in $\alpha$ per studiare la convergenza del metodo.
Con i generici metodi di punto fisso dovevamo invece guardare alla funzione di punto fisso $\phi(x)$.

Vediamo quindi il caso delle radici multiple, cioé $f(x)$ in forma:
$$
f(x) = g(x) (x - \alpha)^r, \quad g(\alpha) \neq 0
$$

Si ha che $r \in \mathbb{N}$, $r > 1$ e:
$$
f'(\alpha) = 0
$$
e quindi il metodo di newton si ridefinisce come:
$$
\phi(x) =
	\begin{cases}
		x - \frac{f(x)}{f'(x)}, \quad x \neq \alpha \\ 
		\alpha,  \quad x = \alpha
	\end{cases}
$$

Ricordiamo che:
$$
\phi'(x) = 1 - \frac{(f'(x))^2 - f(x) f''(x)}{(f'(x))^2} = \frac{f(x) f''(x)}{(f'(x))^2}
$$
dove $f'(x)$ compare al denominatore, quindi non si può valutare in $\alpha$.

Prendiamo allora le derivate successive di $f(x)$:
\begin{enumerate}
	\item 
$
f(x) = g(x) (x - \alpha)^r
$
	\item 
$
f'(x) = g'(x) (x - \alpha)^r + r g(x) (x - \alpha)^{r - 1} = (x - \alpha)^{r - 1} ( g'(x)(x - \alpha) + r g(x) )
$
	\item
$
f''(x) = g''(x) (x - \alpha)^r + 2 r g'(x)(x - \alpha)^{r - 1} + r (r - 1) g(x) (x - \alpha)^{r - 1}
$ \\ 
$$
= (x - \alpha)^{r - 2} ( g''(x) (x - \alpha)^2 + 2 r g'(x) (x - \alpha) + r (r - 1) g(x) )
$$
\end{enumerate}
e sostituiamo in $\phi'(x)$:
$$
\phi'(x) = \frac{ g(x) (x - \alpha)^r \cdot (x - \alpha)^{r - 2} ( g''(x) (x - \alpha^2) + r g'(x) (x - \alpha) + r (r - 1) g(x) ) }{ ( (x - \alpha)^{r - 1} (g'(x) (x - \alpha) + r g(x) ) )^2 }
$$
$$
= \frac{g(x) (g''(x) (x - \alpha^2) + 2 r g'(x) (x - \alpha) + r^2 g^2(x) - r g^2(x) }{(g'(x)(x - \alpha) + r g(x))^2}
$$
da cui:
$$
\phi'(\alpha) = \frac{r^2 g^2(\alpha) - r g^2(\alpha)}{r^2 g^2(\alpha)} = 1 - \frac{1}{r} \neq 0
$$
quando $r > 1$, perciò:
$$
|\phi'(\alpha)| = 1 - \frac{1}{r} < 1
$$
e la convergenza è lineare, dove più è alto $r$ più è lenta la convergenza.

Una proprietà è che se si conosce $r$ (l'ordine della radice) si può modificare Newton per ritrovare la convergenza quadratica.
In particolare si può dimostrare che:
$$
x_{n + 1} = x_n - r \cdot \frac{f(x_n)}{f'(x_n)} = \phi(x_n)
$$
è tale per cui $\phi'(\alpha) = 0$, e quindi la convergenza è almeno quadratica.

\end{document}
